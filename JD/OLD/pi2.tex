\section{The principal investigator}


\subsection{Scientific leadership profile}

I am a Research Director at INRIA Sophia-Antipolis (France) where I lead the Geometrica project-team. I was promoted in 2009 at the highest rank (Class Exceptional) for my contributions to research, formation and dissemination. I also had eminent positions within INRIA, most notably  as the chairman of the Evaluation Commitee of the institute. This position gave me  a comprehensive view of the research performed in the eight research centers of INRIA.

\paragraph{Content and impact of major scientific contributions.}

My research field is {\em Computational Geometry}. I have been one of the very first in France to
get interested in what was at the time (in the early eighties) a new emerging field. I founded a french school in Computational Geometry which is now one of the most influential worldwide.
In 1987, I founded the Prisme project-team devoted to the algorithmic aspects of Robotics. Prisme became the birth place of Computational Geometry in France and played a prominent role in shaping the field. Together with my colleagues, I made major contributions on Voronoi diagrams and Delaunay triangulations, randomized algorithms, surface reconstruction and motion planning. This initial research work led to the book Algorithmic Geometry (1997) co-authored by M. Yvinec, which is recognized as one of the reference books in the field.

I then got convinced that Computational Geometry was lacking impact because its beautiful theoretical results were not seconded by efficient implementations. I promoted research on {\em robust algorithms}, exact computing, algebraic issues in geometric computing, and, most importantly, launched the development of the {\em CGAL library} in collaboration with partners in Europe. CGAL is now regarded as the gold standard in Computational Geometry with a huge impact, far beyond Computational Geometry. A start-up company, GeometryFactory, was created in 2003 by my former PhD student A. Fabri to commercialize CGAL. It is fair to say that my group has been and still  is the leading group among the sites involved in the development of CGAL.

In 2003, I founded the Geometrica project-team in replacement of Prisme. The main objective was to develop Nonlinear Computational Geometry. Together with my students and collaborators, we obtained the first provably correct algorithms to {\em mesh curved objects}. Geometrica's research in this area became flourishing with salient contributions, both theoretical and applied. Part of the research agenda was devoted to the development of CGAL components for mesh generation and  shape reconstruction. Those components are now used worldwide in academia and in industry for  various applications in Geometric Modeling, Medical Imaging and Geology.

Building upon our initial investigation in {\em surface reconstruction}, Geometrica short became one of the leading sites in {\em geometric inference} with prominent contributions to the analysis of distance functions, persistent topology, and the development of a theory of geometric sampling. These theoretical foundations led to highly innovative and influential new results and promissing applications in data analysis. This is of utmost importance since data are ubiquitous and produced at an unprecedented rate in most domains of science. 

Since its creation in 2003, Geometrica has been at the cutting edge of research in 3D Geometric Modeling and, together with a few other sites over the world, settled solid foundations and offered efficient and provably correct software tools.  Time has come to go beyond the third dimension and to consider Geometric Modeling in higher dimensions. A major challenge is to make breakthroughs on the algorithmic side of the theory to bypass the curse of dimensionality, which I consider as the main current obstacle for Geometric Modeling in higher dimensions to become reality. This is the topic I started researching intensively with my students and our first results convinced me that we were ready for such a promissing endeavour. 

\paragraph{International recognition and diffusion.}

I am on the editorial board of some of the most prestigious  journals in Computer Science, the Journal of the ACM, Algorithmica, Discrete and Computational Geometry. I have been on the program committee of many international conferences.

I have been a member of the influential Computational Geometry Impact Task Force chaired by B. Chazelle (1996) and a member of the Computational Geometry Steering Committee (1999-2001). I chaired the Symposium on Computational Geometry (the top conference of the field) in 1997 and I co-chaired the program committee of the symposium in 2004.

% I received two highly prestigious prizes, the IBM award in Computer Science  in 1987
% and the EADS award in Information Sciences in 2006.  I was nominated to the Roberval prize for the french version of my book ”Algorithmic Geometry” coauthored by M. Yvinec.

I am the author of over 150 technical publications in international journals and conferences. My h-number is 47 with 7786 citations according to Google Scholar. 

I am the author of two software that have been commercialized at large scale by major companies, one by Siemens (Flying Through, installed on Siemens scanners) and one by Dassault Systems (integrated in Catia V5 (Shape Editor)). 
My group Geometrica is one of the leader teams of the CGAL project and is at the source of successfull developments in CGAL like interval arithmetics, triangulations and meshing packages. 
The CGAL library has become a de facto standard and is now routinely used for experiments, validation and education. Today, with the list of customers of CGAL (including Schlumberger, Total, Leica, Toshiba, Volkswagen, Petrobras, Dassault Systemes, Navteq) and the recent integration of the triangulation packages of CGAL (developed in my group) in the heart of Matlab, the CGAL project has successfully tackled the main challenge of making Computational Geometry algorithms available for academic and industrial applications, as presented in the Computational Geometry impact task force report back in 1996. CGAL is the only project that achieved this goal, with no equivalent in USA or Asia as of today.




\paragraph{Effort and ability to inspire younger researchers.}
I have supervised 24 Ph.D. students. All of them are enjoying successful careers in academia or industry. I am currently advising two Ph.D. students. One of my former student, Andreas Fabri, founded in 2003 GeometryFactory, a startup company that commercializes CGAL.
Several members of my research groups created their own research teams at INRIA on various subjects~: J-P. Merlet (Robotics), F. Cazals (Structural Biology), S. Lazard (Computational Geometry).

\paragraph{Proven ability to productively change research fields and/or establish new interdisciplinary approaches.}

My Ph.D. thesis was on control theory. I completely changed my research field when I joined INRIA and start working on 3D Geometric Modeling and Robotics. I then turned to Computational Geometry, an emerging field that didn't exist in France at that time.  New thematic moves were undertaken when starting the development of CGAL (algebraic issues, computer arithmetic, programming) and when launching the Geometrica project-team (mesh generation, data analysis, higher dimensions).  I promoted applications of Computational Geometry to Structural Biology with the Ph.D. thesis of J. Duquesne and the recruitment of F. Cazals, a former member of Geometrica who is now leading his own group ABS (Algorithms, Biology, Structure).

\newpage

\subsection{Curriculum Vitae}

-- Born in Nice, May 18, 1953, French Nationality, married, 3 children

\paragraph{Education}\mbox{}

[1976]  Diploma  from the Ecole Sup\'erieure d'Electricit\'e

[1979] PhD Thesis diploma, University of Rennes

[1992]  Habilitation diploma, University of Nice

\paragraph{Professional academic experience}\mbox{}

[1980-1986] Researcher  at INRIA Rocquencourt (Robotvis team)

[1987-2003] Head of the PRISME project team at  INRIA Sophia-Antipolis

[2003-2012] Head of the GEOMETRICA project team at  INRIA Sophia-Antipolis

[2001-2005] VP for Science  of INRIA Sophia-Antipolis (500 employees, 30 research groups)

[2005-2009] Chairman of the Evaluation Committee of INRIA (in charge of the evaluation of personnel and research groups of the 8 INRIA centers)

\paragraph{Academic awards and honors}\mbox{}

[1987] IBM award in Computer Science 

[1996] Nominated to the  Roberval prize for the book "G\'eom\'etrie Algorithmique " coauthored by M. Yvinec.

[2006] Nominated Chevalier de l'Ordre National du M\'erite

[2006] Grand prize EADS in Information Sciences (French Academy of Science)

\paragraph{Publications, patents, software} \mbox{}

-- Over 150 publications including 1 text book, 4 edited books, 57 journal articles, 89 refereed international conference articles, 12 book chapters

-- 4 patents

-- 2 software commercialized respectively by Siemens and Dassault Syst\`emes. My research team is the leading site for the development of CGAL, the de facto standard  in Computational Geometry

\paragraph{Editorship} \mbox{}

-- Geometry and Computing, a Springer book series (2007-)

--  Revue d'Intelligence Artificielle (1991-1996)

--  Techniques des sciences informatiques (1991-1992)

-- {\em Algorithmica} (1990-)

--  Theoretical Computer Science (1990-2002)

-- Computational Geometry : Theory and Applications (1991-2005)

-- {\em The Int. J. on Computational Geometry and Applications} (1991-)

--  The Visual Computer (2000-2005)

-- {\em Discrete and Computational Geometry } (2006-)

-- {\em The Journal of Computational Geometry} (2009-)

-- {\em The Journal of the ACM }(2010-)

\paragraph{Supervision of Ph.D. students and postdocs} \mbox{}

I supervised 24 Ph.D. theses and currently supervise 2 Ph.D. theses.

\paragraph{Teaching} \mbox{}

-- Ecole sup\'erieure d'Electricit\'e : Computer Vision

-- University of Nice : Robotics, Computational Geometry

-- ENS (Paris) : Computational Geometry

-- MPRI (Paris) : Mesh Generation,  Computational Topology

-- Tsinghua University (Beijing) : Geometry Processing and Geometric Modeling (2009-2011)

%\paragraph{Start-up companies}

\paragraph{Nomination to scientific councils (2001-)} \mbox{}

-- Scientific Council of the Ecole Normale Sup\'erieure de Lyon (2000-2003)

-- Member of the AERES Board (French Evaluation Agency for
  Research and Higher Education)

-- Member of working groups 1 (Mod\`eles et calcul) and 2
  (Logiciels et systèmes informatiques) of Allist\`ene (Alliance des sciences et technologies du num\'erique)


-- Chair of the Visiting Committee of LIAMA (Beijing, 2010)

-- Member of the Visiting Committee of the Computer Science department of ULB (Free University of Bruxels, 2011)

-- Member of the Visiting Committee of the Geometric Modeling and Scientific Visualization (GMSV) Center of the King Abdullah University of Science and Technology (KAUST, Saudi Arabia, 2012)


\paragraph{Funding ID (to be completed)} \mbox{}

I have been the site leader of 8 European projects and the project leader of  the IST Project ECG ("Effective Computational Geometry") (2001-2004).

I have been the principal investigator of 8 collaborations with french industry.

\newpage

\subsection{10-year track record}

\paragraph{Top 10 publications as senior researcher}\footnote{Citations are according to Google Scholar. For journal articles, I added the citations of the conference version of the article. Discrete and Computational Geometry and the Symposium on Computational Geometry are regarded as the most prestigious journal and conference in Computational Geometry.}  \mbox{}

%1. Triangulations in CGAL. Comput. Geom. Theory Appl.  Vol. 22 (2002) 5-19. Coauthors: O. Devillers, S. Pion, M. Teillaud, M. Yvinec. (26 citations including those of the conference version)

1. J-D. Boissonnat, F. Cazals. Smooth surface reconstruction via natural neighbour interpolation of
distance functions.  Comput. Geom. Theory Appl. Vol. 22 (2002) 185-203.  (421 citations)

2.  J-D. Boissonnat, F. Cazals. Natural neighbour coordinates of points on a surface.  
Comput. Geom. Theory Appl., Vol. 19, No 2-3, July 2001. (74 citations)

3. D. Attali, J-D. Boissonnat, A. Lieutier. Complexity of the Delaunay Triangulation of Points on Surfaces : the 
Smooth Case. 20th  ACM      Symposium on Computational Geometry, 2003. 
(74 citations)

4. D. Attali, J-D. Boissonnat. A Linear Bound on the Complexity of the Delaunay Triangulation of Points on Polyhedral Surfaces.  Discrete and Comp. Geometry 31: 369--384
(2004). (61 citations)

5. J-D. Boissonnat, S. Oudot. Provably good sampling and meshing of surfaces. Graphical Models, 67 (2005) 405-451. (198 citations. Graphical Models Top-Cited Article 2005-2010)
% including those of the conference version

6. D. Attali, J-D. Boissonnat, H. Edelsbrunner. Stability and computation of medial axes~: a state-of-the-art report.
In {\em Mathematical Foundations of Scientific Visualization,
Computer Graphics, and Massive Data Exploration},
T. Moeller,   B. Hamann and B. Russell Ed.,
Springer, series Mathematics and Visualization, 2007. (93 citations)

7. J-D. Boissonnat, D. Cohen-Steiner, G. Vegter. Isotopic implicit surface meshing.  Discrete and Computational Geometry,  39: 138-157,  2008. (55 citations)% including those of the conference version)

8. J-D. Boissonnat, C. Wormser and M. Yvinec. Locally uniform anisotropic meshing. 
24th ACM Symposium on Computational Geometry, SoCG'08.
(16 citations)

9. J-D. Boissonnat, L. Guibas, S. Oudot. Manifold reconstruction in arbitrary dimensions using witness complexes.
Discrete and Comp. Geom. Vol 42, No 1, 2009. (46 citations)

% 8. An efficient implementation of the Delaunay triangulation and
%   its graph in medium dimension.  25th ACM Symposium on Computational
%   Geometry, SoCG'09.  Coauthors: O. Devillers and S. Hornus.

10. J-D. Boissonnat, F. Nielsen, R. Nock. On Bregman Voronoi diagrams.
Discrete and Comp. Geom. (2), 2010. (76 citations)% including those of the conference version)

%10. Triangulating Smooth Submanifolds  with Light Scaffolding.
%Mathematics in Computer Science, 4(4):431-462, 2011. Coauthor: A. Ghosh.


\paragraph{Edited Books and Proceedings}  \mbox{}

--
Algorithmic Foundations of Robotics V, Springer 2004. Coeditors:  J. Burdick, 
K. Goldberg, S. Hutchinson.

-- Effective Computational Geometry for Curves and Surfaces,
  Springer, 2006. Coeditor:  M. Teillaud. I coauthored two chapters of this book.

--  Curves and Surfaces.
Coeditors:  P. Chenin, A. Cohen,  C. Gout, T. Lyche, M-L.  Mazure and L. Schumaker,
Springer Verlag LNCS Vol. 6920, 2012.


--
Geometric Computing, special issue of the 
International Journal of Computational Geometry and Applications, Vol. 11, 
No. 1, 2001.


--
Computational Geometry, Theory and Applications, Vol. 35 No. 1-2, August 2006.
Special issue on the 20th Symposium on Computational
Geometry.  Coeditor:   J. Snoeyink.

-- 
Discrete and Computational Geometry, Vol. 36, No 4, December 2006.
Special issue on the 20th Symposium on Computational
Geometry.  Coeditor:   J. Snoeyink.



\paragraph{Granted patents} \mbox{}

-- Methods and apparatus for planning robotic surgery. 
United States Patent Application 20030109780. Assignee INRIA and
Intuitive Surgical Inc. (2002).  Coauthors: E. Coste-Manière, L. Adhami,
A. Carpentier, G. Guthart.

-- Method and apparatus for fast automatic centerline extraction for virtual 
endoscopy. United States Patent Application  20050033114. Siemens Corporate 
Research (2004). Coauthor B. Geiger.

\paragraph{Keynote presentations (since 2004)}\mbox{}

$\bullet$ International Symposium on Voronoi Diagrams, Tokyo (2004).
$\bullet$  Workshop "The World a Jigsaw: Tessellations in the Sciences", Leiden (2006).
$\bullet$  French Academy of Science (2 talks, 2006). 
$\bullet$  Franco Preparata's schriftfest, Brown university (2007). 
$\bullet$ Seventh conference on "Mathematical Methods for Curves and Surfaces", Toensberg, Norway, 2008. 
$\bullet$  Colloquium on Emerging Trends in Visual Computing (ETVC, Ecole Polytechnique, 2008). 
$\bullet$  22th Sibgrapi, Rio de Janeiro (2009).  
$\bullet$  ATMCS 2012 (Algebra and Topology; Methods, Computation, and Science), Edinburgh (2012).




\paragraph{Organization of international conferences} \mbox{}

-- [2010] Member of the steering committee of the International Conference on Curves and Surfaces (Avignon)

-- [2011]  Co-chair of the Computational Geometric Learning workshop (Institut Henri Poincar\'e, Paris)

-- [2012] Member of the scientific committee of the eighth conference on "Mathematical Methods for Curves and Surfaces" (Oslo)


\paragraph{International prizes/awards/academy memberships} \mbox{}

[2006] Nominated Chevalier de l'Ordre National du M\'erite

[2006] Grand Prize EADS in Information Sciences (French Academy of Science)

[2010] Graphical Models Top-Cited Article 2005-2010 for  my paper with S. Oudot (ref 5 above) 


\paragraph{Membership to editorial board of international journals}   \mbox{}

-- {\em Algorithmica} (1990-)

-- {\em The Int. J. on Computational Geometry and Applications} (1991-)

-- {\em Discrete and Computational Geometry } (2006-)

-- {\em The Journal of Computational Geometry} (2009-)

-- {\em The Journal of the ACM }(2010-)