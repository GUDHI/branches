\section{The principal investigator}


% \subsection{Scientific leadership profile}

% I am a Research Director at INRIA Sophia-Antipolis (France) where I lead the Geometrica project-team. I was promoted in 2009 at the highest rank (Class Exceptional) for my contributions to research, formation and dissemination. I also had eminent positions within INRIA, most notably  as the chairman of the Evaluation Commitee of the institute. This position gave me  a comprehensive view of the research performed in the eight research centers of INRIA.

% \paragraph{Content and impact of major scientific contributions.}

% My research field is {\em Computational Geometry}. I have been one of the very first in France to
% get interested in what was at the time (in the early eighties) a new emerging field. I founded a french school in Computational Geometry which is now one of the most influential worldwide.
% In 1987, I founded the Prisme project-team devoted to the algorithmic aspects of Robotics. Prisme became the birth place of Computational Geometry in France and played a prominent role in shaping the field. Together with my colleagues, I made major contributions on Voronoi diagrams and Delaunay triangulations, randomized algorithms, surface reconstruction and motion planning. This initial research work led to the book Algorithmic Geometry (1997) co-authored by M. Yvinec, which is recognized as one of the reference books in the field.

% I then got convinced that Computational Geometry was lacking impact because its beautiful theoretical results were not seconded by efficient implementations. I promoted research on {\em robust algorithms}, exact computing, algebraic issues in geometric computing, and, most importantly, launched the development of the {\em CGAL library} in collaboration with partners in Europe. CGAL is now regarded as the gold standard in Computational Geometry with a huge impact, far beyond Computational Geometry. A start-up company, GeometryFactory, was created in 2003 by my former PhD student A. Fabri to commercialize CGAL. It is fair to say that my group has been and still  is the leading group among the sites involved in the development of CGAL.

% In 2003, I founded the Geometrica project-team in replacement of Prisme. The main objective was to develop Nonlinear Computational Geometry. Together with my students and collaborators, we obtained the first provably correct algorithms to {\em mesh curved objects}. Geometrica's research in this area became flourishing with salient contributions, both theoretical and applied. Part of the research agenda was devoted to the development of CGAL components for mesh generation and  shape reconstruction. Those components are now used worldwide in academia and in industry for  various applications in Geometric Modeling, Medical Imaging and Geology.

% Building upon our initial investigation in {\em surface reconstruction}, Geometrica short became one of the leading sites in {\em geometric inference} with prominent contributions to the analysis of distance functions, persistent topology, and the development of a theory of geometric sampling. These theoretical foundations led to highly innovative and influential new results and promissing applications in data analysis. This is of utmost importance since data are ubiquitous and produced at an unprecedented rate in most domains of science. 

% Since its creation in 2003, Geometrica has been at the cutting edge of research in 3D Geometric Modeling and, together with a few other sites over the world, settled solid foundations and offered efficient and provably correct software tools.  Time has come to go beyond the third dimension and to consider Geometric Modeling in higher dimensions. A major challenge is to make breakthroughs on the algorithmic side of the theory to bypass the curse of dimensionality, which I consider as the main current obstacle for Geometric Modeling in higher dimensions to become reality. This is the topic I started researching intensively with my students and our first results convinced me that we were ready for such a promissing endeavour. 

% \paragraph{International recognition and diffusion.}

% I am on the editorial board of some of the most prestigious  journals in Computer Science, the Journal of the ACM, Algorithmica, Discrete and Computational Geometry. I have been on the program committee of many international conferences.

% I have been a member of the influential Computational Geometry Impact Task Force chaired by B. Chazelle (1996) and a member of the Computational Geometry Steering Committee (1999-2001). I chaired the Symposium on Computational Geometry (the top conference of the field) in 1997 and I co-chaired the program committee of the symposium in 2004.

% % I received two highly prestigious prizes, the IBM award in Computer Science  in 1987
% % and the EADS award in Information Sciences in 2006.  I was nominated to the Roberval prize for the french version of my book ”Algorithmic Geometry” coauthored by M. Yvinec.






% \paragraph{Effort and ability to inspire younger researchers.}
% I have supervised 24 Ph.D. students. All of them are enjoying successful careers in academia or industry. I am currently advising two Ph.D. students. One of my former student, Andreas Fabri, founded in 2003 GeometryFactory, a startup company that commercializes CGAL.
% Several members of my research groups created their own research teams at INRIA on various subjects~: J-P. Merlet (Robotics), F. Cazals (Structural Biology), S. Lazard (Computational Geometry).

% \paragraph{Proven ability to productively change research fields and/or establish new interdisciplinary approaches.}

% .

% \newpage

\subsection{Curriculum Vitae}

I was born in Nice, May 18, 1953. I am married and have 3 children. I am a french citizen.

\paragraph{Education}\mbox{}

$\bullet$ I graduated from the Ecole Sup\'erieure d'Electricit\'e (Sup\'elec) in 1976.  Sup\'elec is among the top ``Grandes Ecoles'' in France and the reference in the field of electric energy and information sciences. Sup\'elec is on a par with the best departments of electrical and computer engineering of the top American or European universities.
$\bullet$
I obtained my PhD Thesis in Control Theory from the University of Rennes. 
$\bullet$
I obtained the Habilitation diploma (the highest grade at french universities) in Computer Science from the  University of Nice in 1992.

\paragraph{Professional academic experience}\mbox{}

 $\bullet$ I have been a researcher at INRIA since 1980, first in Rocquencourt and since 1986 in Sophia Antipolis.  $\bullet$  I am currently a {\em Research Director} at INRIA Sophia-Antipolis (France) where I lead the Geometrica project-team. I was promoted in 2009 at the highest rank (Class Exceptional) for my contributions to research, formation and dissemination. $\bullet$ I also had eminent positions within INRIA, most notably  
the {\em VP for Science}  at INRIA Sophia-Antipolis (500 employees, 30 research groups) (2001-2005).
I have also been  the {\em chairman of the Evaluation Commitee} of the institute, one of the most important position at INRIA with a key role in the definition and the implementation
of the scientific policy of the institute.  This position gave me  a comprehensive view of the research performed in the eight research centers of INRIA.


% \paragraph{Scientific expertise} \mbox{}
% Computational Geometry, Algorithmic Robotics, Geometric Computing, Mesh Generation, Shape Reconstruction, Geometric Learning

\paragraph{Academic awards and honors}\mbox{}

I received two highly prestigious prizes, the IBM award in Computer Science  in 1987
and the Grand prize EADS in Information Sciences in 2006 (awarded by the French Academy of Science).  I was nominated to the Roberval prize for the french version of my book ”Algorithmic Geometry” coauthored by M. Yvinec. I have also been nominated Chevalier de l'Ordre National du M\'erite in 2006.

\paragraph{Publications, patents} \mbox{}

I am the author of over 150 technical publications including 1 text book, 4 edited books, 57 journal articles, 89 refereed international conference articles, 12 book chapters. My h-number is 47 with 7786 citations according to Google Scholar. 

My publications cover {\em several fields}~: computational geometry, geometric modeling, algorithmic robotics, medical imaging and to a lower extent structural biology and control theory.

My main {\em contributions} are on  mesh generation, surface reconstruction, motion planning, robust geometric computing, randomized algorithms, Voronoi diagrams, Delaunay triangulations, manifold learning.

I am the author of {\em 4 patents}~: 2 on mesh generation (Assignee: Institut Francais du P\'etrole (IFP)), robotic surgery (Assignee: Intuitive Surgical Inc.), virtual endoscopy (Assignee: Siemens Coroporate Research)).

\paragraph{Software} \mbox{}

I am the author of two software that have been commercialized at large scale by major companies, one by Siemens (Flying Through, installed on Siemens scanners) and one by Dassault Systems (integrated in Catia V5 (Shape Editor)). 

My group Geometrica is one of the leader teams of  the CGAL Open Source Project.  The CGAL library  is now regarded as the gold standard in Computational Geometry with a huge impact, both in academia and in industry (e.g., the triangulation package of CGAL, developed within Geometrica, is now integrated in the heart of Matlab)



% -- 2 software commercialized respectively by Siemens and Dassault Syst\`emes. My research team is the leading site for the development of CGAL, the de facto standard  in Computational Geometry

% \paragraph{Editorship} \mbox{}

% I am on the editorial board of some of the most prestigious  journals in Computer Science, the Journal of the ACM, Algorithmica, Discrete and Computational Geometry. I have been on the program committee of many international conferences.



% -- Geometry and Computing, a Springer book series (2007-)

% --  Revue d'Intelligence Artificielle (1991-1996)

% --  Techniques des sciences informatiques (1991-1992)

% -- {\em Algorithmica} (1990-)

% --  Theoretical Computer Science (1990-2002)

% -- Computational Geometry : Theory and Applications (1991-2005)

% -- {\em The Int. J. on Computational Geometry and Applications} (1991-)

% --  The Visual Computer (2000-2005)

% -- {\em Discrete and Computational Geometry } (2006-)

% -- {\em The Journal of Computational Geometry} (2009-)

% -- {\em The Journal of the ACM }(2010-)

\paragraph{Scientific leadership} \mbox{}

%  I have been one of the very first in France to
% get interested in what was at the time (in the early eighties) a new emerging field. I founded a french school in Computational Geometry which is now one of the most influential worldwide.
% In 1987, I founded the Prisme project-team devoted to the algorithmic aspects of Robotics. 
My first group Prisme has been the birth place of Computational Geometry in France and played a prominent role in shaping the field. 
I promoted research on Delaunay triangulations, randomized algorithms, exact computing, and, most importantly, launched 15 years ago the development of the {\em CGAL library} in collaboration with partners in Europe~\cite{cgal}.

In 2003, I founded the Geometrica project-team in replacement of Prisme. The main objective was to develop Nonlinear Computational Geometry. 
 Geometrica's research in this area has been flourishing with seminal contributions on mesh generation and surface reconstruction~\cite{geometrica-ecg-book}. Part of the research agenda was devoted to the development of CGAL components that are now used worldwide in academia and in industry for various applications in geometric modeling, medical imaging and geology.

In 2006, I created an antenna of Geometrica in Saclay (Paris's area)  to work on  the emerging field of {\em geometric inference}.  Together with my colleagues, we made influential contributions to the analysis of distance functions, persistent topology, manifold learning and the development of a theory of geometric sampling.

% These theoretical foundations led to highly innovative and influential new results and promissing applications in data analysis. This is of utmost importance since point set data are ubiquitous and produced at an unprecedented rate in most domains of science. 

% Since its creation in 2003, Geometrica has been at the cutting edge of research in 3D Geometric Modeling and, together with a few other sites over the world, settled solid foundations and offered efficient and provably correct software tools.  Time has come to go beyond the third dimension and to consider Geometric Modeling in higher dimensions. A major challenge is to make breakthroughs on the algorithmic side of the theory to bypass the curse of dimensionality, which I consider as the main current obstacle for Geometric Modeling in higher dimensions to become reality. This is the topic I started researching intensively with my students and our first results convinced me that we were ready for such a promissing endeavour. 


\paragraph{Supervision of Ph.D. students, postdocs and young researchers} \mbox{}

%I supervised 24 Ph.D. theses and currently supervise 2 Ph.D. theses.
I have supervised 24 Ph.D. students. All of them are enjoying successful careers in academia or industry. I am currently advising two Ph.D. students. One of my former students, Andreas Fabri, founded in 2003 GeometryFactory, a startup company that commercializes CGAL.

Five members of my research group successfully defended their Habilitation.  Three created their own research teams at INRIA~: J-P. Merlet (Robotics), F. Cazals (Structural Biology), S. Lazard (Computational Geometry). Another member of the group, P. Alliez (who received a Jr. ERC grant), is in the process of creating his own group on Geometry Processing.

%  I promoted applications of Computational Geometry to Structural Biology with the Ph.D. thesis of J. Duquesne and the recruitment of F. Cazals, a former member of Geometrica who is now leading his own group ABS (Algorithms, Biology, Structure) 

% \paragraph{Teaching} \mbox{}

% $\bullet$  Ecole sup\'erieure d'Electricit\'e : Computer Vision
% $\bullet$  University of Nice : Robotics, Computational Geometry
% $\bullet$  ENS (Paris) : Computational Geometry
% $\bullet$  MPRI (Paris) : Mesh Generation,  Computational Topology
% $\bullet$  Tsinghua University (Beijing) : Geometry Processing and Geometric Modeling (2009-2011)

% %\paragraph{Start-up companies}



\paragraph{Funding ID} \mbox{}

 I have been the site leader of 8 European projects and the project leader of  the IST Project ECG ("Effective Computational Geometry") (2001-2004).

I have been the principal investigator of 8 collaborations with french industry.

I am a site leader of the ICT Fet-Open project Computational Geometric Learning (CGL) which is closely related to this project (http://cglearning.eu/). 
%http://cordis.europa.eu/fp7/ict/fet-open/) portfolio-cglearning_en.html. 

\newpage

\subsection{10-year track record}

\paragraph{Top 10 publications as senior researcher}  \mbox{} 

Citations are according to Google Scholar. For journal articles, I added the citations of the conference version of the article. Discrete and Computational Geometry and the Symposium on Computational Geometry are regarded as the most prestigious journal and conference in Computational Geometry.

%1. Triangulations in CGAL. Comput. Geom. Theory Appl.  Vol. 22 (2002) 5-19. Coauthors: O. Devillers, S. Pion, M. Teillaud, M. Yvinec. (26 citations including those of the conference version)

1. J-D. Boissonnat, F. Cazals. Smooth surface reconstruction via natural neighbour interpolation of
distance functions.  Comput. Geom. Theory Appl. Vol. 22 (2002) 185-203.  (421 citations)

2.  J-D. Boissonnat, F. Cazals. Natural neighbour coordinates of points on a surface.  
Comput. Geom. Theory Appl., Vol. 19, No 2-3, July 2001. (74 citations)

3. D. Attali, J-D. Boissonnat, A. Lieutier. Complexity of the Delaunay Triangulation of Points on Surfaces : the 
Smooth Case. 20th  ACM      Symposium on Computational Geometry, 2003. 
(74 citations)

4. D. Attali, J-D. Boissonnat. A Linear Bound on the Complexity of the Delaunay Triangulation of Points on Polyhedral Surfaces.  Discrete and Comp. Geometry 31: 369--384
(2004). (61 citations)

5. J-D. Boissonnat, S. Oudot. Provably good sampling and meshing of surfaces. Graphical Models, 67 (2005) 405-451. (198 citations. Graphical Models Top-Cited Article 2005-2010)
% including those of the conference version

6. D. Attali, J-D. Boissonnat, H. Edelsbrunner. Stability and computation of medial axes~: a state-of-the-art report.
In {\em Mathematical Foundations of Scientific Visualization,
Computer Graphics, and Massive Data Exploration},
T. Moeller,   B. Hamann and B. Russell Ed.,
Springer, series Mathematics and Visualization, 2007. (93 citations)

7. J-D. Boissonnat, D. Cohen-Steiner, G. Vegter. Isotopic implicit surface meshing.  Discrete and Computational Geometry,  39: 138-157,  2008. (55 citations)% including those of the conference version)

8. J-D. Boissonnat, C. Wormser and M. Yvinec. Locally uniform anisotropic meshing. 
24th ACM Symposium on Computational Geometry, SoCG'08.
(16 citations)

9. J-D. Boissonnat, L. Guibas, S. Oudot. Manifold reconstruction in arbitrary dimensions using witness complexes.
Discrete and Comp. Geom. Vol 42, No 1, 2009. (46 citations)

% 8. An efficient implementation of the Delaunay triangulation and
%   its graph in medium dimension.  25th ACM Symposium on Computational
%   Geometry, SoCG'09.  Coauthors: O. Devillers and S. Hornus.

10. J-D. Boissonnat, F. Nielsen, R. Nock. On Bregman Voronoi diagrams.
Discrete and Comp. Geom. (2), 2010. (76 citations)% including those of the conference version)

%10. Triangulating Smooth Submanifolds  with Light Scaffolding.
%Mathematics in Computer Science, 4(4):431-462, 2011. Coauthor: A. Ghosh.


\paragraph{Edited Books and Proceedings}  \mbox{}

$\bullet$
Algorithmic Foundations of Robotics V, Springer 2004. Coeditors:  J. Burdick, 
K. Goldberg, S. Hutchinson.
$\bullet$ Effective Computational Geometry for Curves and Surfaces,
  Springer, 2006. Coeditor:  M. Teillaud. I coauthored two chapters of this book.
$\bullet$  Curves and Surfaces.
Coeditors:  P. Chenin, A. Cohen,  C. Gout, T. Lyche, M-L.  Mazure and L. Schumaker,
Springer Verlag LNCS Vol. 6920, 2012.
$\bullet$
Geometric Computing, special issue of the 
International Journal of Computational Geometry and Applications, Vol. 11, 
No. 1, 2001.
$\bullet$
Computational Geometry, Theory and Applications, Vol. 35 No. 1-2, August 2006.
Special issue on the 20th Symposium on Computational
Geometry.  %Coeditor:   J. Snoeyink.
$\bullet$ 
Discrete and Computational Geometry, Vol. 36, No 4, December 2006.
Special issue on the 20th Symposium on Computational
Geometry.  %Coeditor:   J. Snoeyink.



\paragraph{Granted patents} \mbox{}

$\bullet$  Methods and apparatus for planning robotic surgery. 
United States Patent Application 20030109780. Assignee INRIA and
Intuitive Surgical Inc. (2002).  Coauthors: E. Coste-Mani\`ere, L. Adhami,
A. Carpentier, G. Guthart.
$\bullet$  Method and apparatus for fast automatic centerline extraction for virtual 
endoscopy. United States Patent Application  20050033114. Siemens Corporate 
Research (2004). Coauthor B. Geiger.

\paragraph{Keynote presentations (since 2004)}\mbox{}

$\bullet$ International Symposium on Voronoi Diagrams, Tokyo (2004).
$\bullet$  Workshop "The World a Jigsaw: Tessellations in the Sciences", Leiden (2006).
$\bullet$  French Academy of Science (2 talks, 2006). 
$\bullet$  Franco Preparata's schriftfest, Brown university (2007). 
$\bullet$ Seventh conference on "Mathematical Methods for Curves and Surfaces", Toensberg, Norway, 2008. 
$\bullet$  Colloquium on Emerging Trends in Visual Computing (ETVC, Ecole Polytechnique, 2008). 
$\bullet$  22th Sibgrapi, Rio de Janeiro (2009).  
$\bullet$  ATMCS 2012 (Algebra and Topology; Methods, Computation, and Science), Edinburgh (2012).

\paragraph{Membership to editorial board of international journals}   \mbox{}


I am on the editorial board of 5 international scientific journals, including two among the most prestigious  journals in Computer Science, the {\em Journal of the ACM} and  {\em Algorithmica}, and the first venue in my field {\em Discrete and Computational Geometry}. 


% I have been on the program committee of many international conferences.
% $\bullet$  {\em Algorithmica} (1990-) $\bullet$  {\em The Int. J. on Computational Geometry and Applications} (1991-)
% $\bullet$  {\em Discrete and Computational Geometry } (2006-)
% $\bullet$  {\em The Journal of Computational Geometry} (2009-)
% $\bullet$  {\em The Journal of the ACM }(2010-)


\paragraph{Organization of international conferences} \mbox{}

%I have been a member of the Computational Geometry Steering Committee (1999-2001). 
%I chaired the Symposium on Computational Geometry (the top conference of the field) in 1997 and

$\bullet$ I co-chaired in 2004 the program committee of the  Symposium on Computational Geometry (SCG), the top conference of the field.
$\bullet$ 
I chaired the Workshop on Algorithmic Foundations of Robotics (WAFR) in 2002.
$\bullet$ 
I have been a member of the steering committee of SCG (1999-2001) and of the scientific committees of the International Conference on Curves and Surfaces (2010) and of  the eighth conference on "Mathematical Methods for Curves and Surfaces" (2012).
$\bullet$ 
I have been on the program committee of  the following international conferences~: Symposium on Geometry Processing (each year since its creation in 2003), 
STACS 2001 (Symposium on Theoretical Aspects of Computer Science),
ESA 2003 (European Symposium on Algorithms),
SCG'04 (Symposium on Computational Geometry)
SMI'05 (Shape Modelling International),
SMP'05 (ACM Symposium on Solid and Physical Modeling),
Curves and Surfaces 2006,
GMP 2012 (Geometric Modeling 
and Processing).

\paragraph{International prizes/awards/academy memberships} \mbox{}
I received the Grand prize EADS in Information Sciences in 2006 (awarded by the French Academy of Science).  I received the Graphical Models Top-Cited Article for the period 2005-2010 for  my paper with S. Oudot (ref 5 above).

\paragraph{Scientific councils and international visiting committees (2001-)} \mbox{}


$\bullet$  Scientific Council of the Ecole Normale Sup\'erieure de Lyon (2000-2003)
$\bullet$  Member of the AERES Board (French Evaluation Agency for
  Research and Higher Education)
$\bullet$  Member of working groups 1 (Mod\`eles et calcul) and 2
  (Logiciels et systèmes informatiques) of Allist\`ene (Alliance des sciences et technologies du num\'erique)
$\bullet$  Chair of the Visiting Committee of LIAMA (Beijing, 2010)
$\bullet$  Member of the Visiting Committee of the Computer Science department of ULB (Free University of Bruxels, 2011)
$\bullet$    Member of the Visiting Committee of the Geometric Modeling and Scientific Visualization (GMSV) Center of the King Abdullah University of Science and Technology (KAUST, Saudi Arabia, 2012)



