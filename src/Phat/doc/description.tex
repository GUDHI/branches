\documentclass[11pt]{article} 
\usepackage{amsmath}
\usepackage{amsfonts} 
\usepackage{amsthm} 
\usepackage{amssymb}
\usepackage{color} 
\usepackage{url} 
\usepackage{graphicx} 





\begin{document}



\centerline{\sc \large Phat bindings to Gudhi library.}
\vspace{.5pc}

The class \emph{Compute\_persistence\_with\_phat} allows computations of 
$\mathbb{Z}_2$ 
persistent homology 
using Phat software \url{https://bitbucket.org/phat-code/phat}.
Phat is a project developed at IST-Austria founded by Ulrich Bauer, Michael Kerber and Jan Reininghaus 
and contributed by Hubert Wagner. The following algorithms from Phat are avialable in Gudhi:
\begin{enumerate}
\item The "standard" algorithm (see~\cite{one}, p.153), available via the method \emph{compute\_persistence\_pairs\_standard\_reduction()}.
\item The "twist" algorithm, as described in~\cite{three} (default algorithm) aviailable via the method \emph{compute\_persistence\_pairs\_twist\_reduction()}.
\item The "chunk" algorithm presented in~\cite{four}  aviailable via the method \emph{compute\_persistence\_pairs\_dualized\_chunk\_reduction()}.
\item The "spectral sequence" algorithm (see~\cite{one}, p.166)  available via the method \emph{compute\_persistence\_pairs\_spectral\_sequence\_reduction()}.
\end{enumerate}

When using this functionality please acknowledge both Gudhi and the Phat contributors. 

\begin{thebibliography}{9}
\bibitem{one} H.Edelsbrunner, J.Harer, \emph{Computational Topology, An Introduction.} American Mathematical Society, 2010, ISBN 0-8218-4925-5.
%\bibitem{two} V.de Silva, D.Morozov, M.Vejdemo-Johansson, \emph{Dualities in persistent (co)homology.} Inverse Problems 27, 2011.
\bibitem{three} C.Chen, M.Kerber, \emph{Persistent Homology Computation With a Twist.} 27th European Workshop on Computational Geometry, 2011.
\bibitem{four} U.Bauer, M.Kerber, J.Reininghaus, \emph{Clear and Compress: Computing Persistent Homology in Chunks.} [http://arxiv.org/pdf/1303.0477.pdf arXiv:1303.0477]
\end{thebibliography}
\end{document}