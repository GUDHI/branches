\subsection*{WP 3:  Robust models for geometric inference and comparison} 


%Remarques a inserer dans l'intro generale du projet (version longue, plus eventuellement version courte):
%
%- La majorite des donnees proviennent de simulations ou de prises mesures.  Elles sont generalement corrompues par du bruit, des outliers, et elles sont souvent parcellaires. Il est donc %important de disposer de techniques permettant de reparer ces defauts, ou bien directement de methodes d'analyse qui n'y soient pas sensibles.
%
%- Le fait de considerer des donnees en grandes dimensions ou dans des espaces "compliques" empeche l'exploration visuelle directe des donnees, mais egalement la validation visuelle des %resultats obtenus par l'analyse. Il y a donc besoin de methodes de visualisation de la structure de telles donnees qui viennent avec des garanties sur la pertinence des resultats, de sorte que %l'on soit assure que ce qui est montre par ces methodes est bien de l'information et non du bruit.



\paragraph{Noise models.}
Assuming that the data are corrupted by noise of small amplitude succesful robust methods have been proposed for the reconstruction and the estimation of topological and geometric  properties of shapes. They are usually based on the study of the topology of the level sets of the distance function to the data that can be related to the topology of the underlying shape when the Hausdorff distance  between the data and the shape is small enough (small amplitude noise). As previously mentioned such an assumption on the noise does not comply with many applications where the data come corrupted by non local noise. Moreover most of the methods relying on the closeness of the data to the shape with respect to the Hausdorff distance are deterministic and do not take into account the statistical nature of the noise.  
In the recent years more statistical approaches allowing to deal with larger classes of noise models have been considered to infer geometric information from data. Some of them intend to remove the part of the data that is far away from the underlying shape and to make use of  the distance function framework \cite{nsw-tvu-2011} but they assume very restrictive noise models. On another hand purely statistical approaches have been proposed for shape approximation that work for large families of noise models (see, e.g., \cite{gpvw-mme-2011,gpvw-mesd-2011}) but they do not come with topological guarantees on the approximated geometric shapes and do not always provide explicite estimates. A major remaining challenge is the design of new unifying frameworks that embrace the statistical approaches and the deterministic methods coming with topological guarantees.  We intend to develop practically efficient tools for robust topological and geometric inference that work with large classes of models of noise. 


\paragraph{Homology inference.}
To comply with the presence of noise and outliers in the data a new paradigm for point cloud data analysis has emerged recently, where point clouds are no
longer treated as mere compact sets but rather as empirical measures. A notion of distance to
such measures has been defined and shown to be stable with respect to perturbations of the
measure \cite{ccsm-gipm-2011}. This distance can easily be computed pointwise in the case of a point cloud (simply
average the squared distances to the k nearest neighbors), but its sublevel-sets, which carry the
geometric information about the measure (or the underlying shape if we consider a model where the data is generated from a measure on the shapes corrupted by some noise), remain hard to compute or approximate. A big challenge now is to find efficient algorithms in arbitrary dimensions to compute or approximate
the topological structure of the sublevel-sets of the distance to a measure, in the same spirit as
what was done in the recent years for distances to compact sets. Such algorithms would naturally
find applications in topological inference in the presence of significant noise and outliers, but
also in other less obvious contexts such as stable clustering. The current bottleneck is that
there exist no equivalents of the union of balls and alpha-shape in the case of the distance to
a measure. Our first goal will be to work out such equivalents that will allow to infer the homology of the underlying shape or more generally the topological persistence of the distance to measure functions.
%To start with, we will focus on
%medium dimensions and use a variant of the mesh-based inference algorithm [14] to approximate
%the sublevel-sets of the distance to a measure and get an idea of their topological structure.



\paragraph{Robust reconstruction.}

Curve and surface reconstruction has been a prominent subect of
research in several fields including computational geometry, computer
graphics, and medical imaging, for more than twenty
years~\cite{dey-csr-2007}. Its challenges are now well-understood, and
countless methods have been proposed, some of which combine
theoretical soundness with practical robustness and efficiency.

The story becomes quite different when the input data is sitting in
higher dimensions. Existing methods fail blatantly because of two
phenomena: the unavoidable corruption of the data, and the curse of
dimensionality. A few timid attempts have been made to tackle these
issues: for instance, multiscale
reconstruction~\cite{geometrica-bgo-09} can handle data with a limited
degree of corruption, while tangential
complexes~\cite{geometrica-7142i} or other projection-based data
structures can circumvent the curse of dimensionality. However, there
is still considerable progress to be made to handle larger classes of
shapes and more general noise models, since currently only the case of
sampled submanifolds of Euclidean spaces with bounded Hausdorff noise
is addressed.

Reconstruction is such a demanding problem that we cannot hope to be
able to reconstruct all sorts of compact sets within only a few
years. A seemingly reasonable goal would be to reconstruct stratified
manifolds, a problem that received some attention in the recent years,
and for which several promising contributions were made.  Bendich and
co-authors~\cite{bendich-PhD,bendich-strat1,bendich-strat2} used
local homology to recover strata from a sampled stratified manifold:
although not quite practical, their approach is based on a sound
theoretical framework. More practical is the approach by Aanjaneya
{\em et al.}~\cite{metric-graphs-reconstruction}, but it is limited to
graphs (1-d manifolds) reconstruction. Can a compromise be made
between generality and practical efficiency of the reconstruction?

In order to handle a wider class of noise models, one made consider
incorporating some statistical techniques into our otherwise purely
topological or geometric approaches. A firs attempt along this line
was made by Caillerie and Michel~\cite{claire-bertrand}, who used
model selection to estimate the {\em most relevant} (in a statistical
sense) scale at which to process the data in multi-scale
reconstruction. Their criterion for measuring the quality of the
reconstruction was oblivious to the topology, being purely based on
the Hausdorff approximation to the underlying object. An interesting
extension would be to incorporate topological constraints into the
method. Another related question would be to try to understand the
relationship that may exist between optimal geometric reconstruction
and optimal topological reconstruction.


\paragraph{Topological signatures for shapes.}

Shape descriptors are used in a variety of applications, including
shape classification, shape retrieval, shape matching, shape
registration, and symmetry detection. In the context of this proposal,
the word {\em shape} must be understood in a very broad sense: for
instance, it can be a point cloud, or a
manifold, or more generally a compact metric space.

Using measured topological quantities to design signatures for shapes
is a relatively new idea. The bottom line of the approach is the
following: given a finite sampling of the shape, build some filtered
simplicial complex on top of the point cloud, and use the topological
structure of this filtration (encoded as a planar diagram called a
{\em persistence diagram}) as a signature for the point
cloud~\cite{ccgmo-ghsssp-09, socg-pbsds-10}. This construction is
well-suited for finite metric spaces, and the obtained signatures are
known to be stable under small perturbations of the spaces in the
Gromov-Hausdorff distance.  

A major remaining challenge is to extend the
construction to infinite metric spaces, and to prove the stability of
its topological structure with respect to perturbations of the space
in the Gromov-Hausdorff distance.

A more fundamental question is how much information about a shape can
be recovered from descriptors. Previous work in shape analysis only
provides lower bounds on a shape distance (e.g. the Gromov-Hausdorff
distance) based on descriptor distance~\cite{survey-on-signatures}. In
other words, equivalent or closely-related shapes do have equal or
similar signatures, but significantly different shapes may also have
similar signatures. It is desirable to go the other way --- to upper
bound shape similarity based on descriptor similarity. This objective
does not seem out of reach in our case: thanks to the virtually
infinite variety of filtrations that can be built on top of a shape,
it is easy to enrich the pool of signatures used for that shape, and
thus restrict the possibility of false positives in the shape
comparison process. It is then a question of how large a family of
filtrations is required to guarantee that different shapes do get
different signatures. And from an algorithmic perspective, it is also
important to identify small samplings of this family of filtrations
that can be used as proxies for a better (if not perfect) assessment
of the similarity between two shapes.

Finally, the difficulty of matching two shapes is
intimately tied to matching a shape to itself --- shapes with many
natural self maps (symmetries) can be difficult to match because of
the ambiguities symmetries create (reflected in duplicate descriptors,
etc.). It may be interesting to define some sort of {\em condition
  number} for a shape, which would capture the intrinsic difficulty of
characteristic or matching against that shape.



