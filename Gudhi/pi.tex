\section{The principal investigator}

\subsection{Curriculum Vitae}

I was born in Nice, May 18, 1953. I am married and have 3 children. I am a french citizen.

\paragraph{Education}\mbox{}

$\bullet$ I graduated from the Ecole Sup\'erieure d'Electricit\'e (Sup\'elec) in 1976.  Sup\'elec is among the top ``Grandes Ecoles'' in France and the reference in the field of electric energy and information sciences. Sup\'elec is on a par with the best departments of electrical and computer engineering of the top American or European universities.
$\bullet$
I obtained my PhD Thesis in Information Theory from the University of Rennes (1979). 
$\bullet$
I obtained the Habilitation diploma (the highest grade at french universities) in Computer Science from the  University of Nice in 1992.

\paragraph{Professional academic experience}\mbox{}

 $\bullet$ I have been a researcher at INRIA since 1980, first in Rocquencourt and since 1986 in Sophia Antipolis.  $\bullet$  I am currently a {\em Research Director} (on par with full professor) at INRIA Sophia-Antipolis (France) where I lead the Geometrica research group. I was promoted in 2009 at the highest rank (Exceptional Class) for my contributions to research, formation and dissemination. $\bullet$ I held important positions within INRIA. I have been 
the {\em VP for Science}  at INRIA Sophia-Antipolis (500 employees, 38 research groups) (2001-2005). 
I have also been  the {\em chairman of the Evaluation Committee} of the institute, one of the most eminent positions at INRIA with a key role in the definition and the implementation
of the scientific policy of the institute (2005-2009).  This position gave me  a comprehensive view of the research performed in the eight research centers of INRIA.


% \paragraph{Scientific expertise} \mbox{}
% Computational Geometry, Algorithmic Robotics, Geometric Computing, Mesh Generation, Shape Reconstruction, Geometric Learning

\paragraph{Scientific leadership} \mbox{}

% I conducted My {\em scientific research} is mainly in computational geometry and topology, geometric modeling, algorithmic robotics and medical imaging.
% My main {\em contributions} are on  mesh generation, surface reconstruction, motion planning, robust geometric computing, randomized algorithms, Voronoi diagrams, Delaunay triangulations, manifold learning,  and to a lower extent on structural biology and control theory.

My first research group at INRIA, {\em Prisme}, has been the birth place of Computational Geometry in France and played a prominent role in shaping the field~\cite{by-ag-98}. 
I contributed and directed research on Voronoi diagrams, Delaunay triangulations, randomized algorithms, motion planning, exact computing, and, most importantly, initiated 15 years ago the development of the {\em CGAL library} in collaboration with partners in Europe (\url{http://www.cgal.org}{}).

In 2003, I founded the {\em Geometrica} research group in replacement of Prisme. The main objective was to develop Nonlinear Computational Geometry. 
 Geometrica's research in this area has been flourishing and well regarded internationally. Together with my students and collaborators, I made seminal contributions on mesh generation and surface reconstruction (see ref. 1-2, 5, 7-8 in Section~\ref{trackrecord}). A very visible output of this research has been the development of CGAL components. These components are now used worldwide in academia and in industry for various applications in geometric modeling, medical imaging and geology.\footnote{see \url{http://www-sop.inria.fr/geometrica/software/cgalmesh/}}

Together with F. Chazal, we established in 2006 a subgroup of Geometrica in {\em Saclay} (Paris's area)  to strengthen our work on  the emerging field of {\em geometric inference}.  This resulted in influential contributions to the analysis of distance functions, persistent homology, manifold learning and the development of geometric inference~\cite{geometrica-7142i,ccsm-gipm-2011}
\newpage


\paragraph{Publications and patents} \mbox{}

I am the author of over 150 technical publications including 1 text book, 57 journal articles, 91 refereed international conference articles, 12 book chapters.  I edited 4 books. My {\em h-number is 47 with 7786 citations} according to Google Scholar. 

I am the author of {\em 4 patents}~: 2 on mesh generation (Assignee: Institut Francais du P\'etrole (IFP)), robotic surgery (Assignee: Intuitive Surgical Inc.), virtual endoscopy (Assignee: Siemens Corporate Research)).

\paragraph{Software} \mbox{}

I am the author of two software that have been commercialized at large scale by major companies, one by Siemens (Flying Through, installed on Siemens scanners) and one by Dassault Systems (integrated in Catia V5 (Shape Editor)). 

My research group Geometrica has been one of the leader teams of  the CGAL Open Source Project since its start (\url{http://www.cgal.org}{}).  The CGAL library  is now regarded as the {\em gold standard in Computational Geometry} with a huge impact, both in academia and in industry.\footnote{For a few identified academic projects using CGAL (including two ERC Starting Grant projects), see
\url{http://www.cgal.org/projects.html}} Notably, the triangulation package of CGAL, developed within Geometrica, is now integrated in the heart of Matlab.

\paragraph{Supervision of Ph.D. students, postdocs and young researchers} \mbox{}

I have supervised 24 Ph.D. students. All of them are enjoying successful careers in academia or industry. I am currently advising two Ph.D. students. One of my former students, Andreas Fabri, founded in 2003 GeometryFactory, a startup company that commercializes CGAL.

Three members of my research group successfully  created their own research groups at INRIA with varied topics~: J-P. Merlet (Robotics), F. Cazals (Structural Biology), S. Lazard (Computational Geometry). Another member of the group, P. Alliez (who received an ERC starting grant), is in the process of creating his own group on geometry processing and modeling of urban scenes.

\paragraph{Academic awards and honors}\mbox{}

$\bullet$ I received two highly prestigious prizes, the {\em IBM award in Computer Science}  in 1987
and the {\em Grand prize EADS in Information Sciences} in 2006 (awarded by the {\em French Academy of Science}).  $\bullet$  I was nominated to the {\em international Roberval prize} for the french version of my book ”Algorithmic Geometry” coauthored by M. Yvinec (The Roverval prize awards the best scientific textbook in french). $\bullet$  I have been received in the {\em National Order of Merit} (Order of State awarded by the President of the French Republic)  in 2006.
My student C. Maria and I received a best paper awards at the European Symp. on Algorithms (ESA 2012).

\paragraph{Funding ID} \mbox{}

 % I have been the site leader of 8 European projects and the project leader of  the IST Project ECG ("Effective Computational Geometry") (2001-2004). % These projects gave a proeminent position to the European community of computational geometry, and led to the successful development of CGAL.

% I have been the principal investigator of 8 collaborations with french industry. The collaboration with Dassault Syst\`emes, a world leader in Geometric Modeling,  is of particular significance for this project %. We have been collaborating  with Dassault Syst\`emes for more than 10 years 
% (commercialization of software, joint publications).

We obtained with my team a budget of external funds of approximately 1.3 million Euros to support our research activities during the period 2007-2011 which divides into 0.5 from European projects,
0.5 from ANR projects (ANR is the french National Research Agency) and 0.3 from industry.

I am a site leader of the {\em ICT Fet-Open project} Computational Geometric Learning (CGL) which is closely related to the Gudhi project (\url{http://cglearning.eu/}). CGL will end in 2013. 
%http://cordis.europa.eu/fp7/ict/fet-open/) portfolio-cglearning_en.html. 

\newpage

\subsection{10-year track record}
\label{trackrecord}

\paragraph{Top 10 publications as senior researcher}  \mbox{} 

{\em Citations are according to Google Scholar. For journal articles, I added the citations of the conference version of the article. Discrete and Computational Geometry and the Symposium on Computational Geometry are regarded as the most prestigious journal and conference in Computational Geometry.}

%1. Triangulations in CGAL. Comput. Geom. Theory Appl.  Vol. 22 (2002) 5-19. Coauthors: O. Devillers, S. Pion, M. Teillaud, M. Yvinec. (26 citations including those of the conference version)

1. J-D. Boissonnat, F. Cazals. Smooth surface reconstruction via natural neighbour interpolation of
distance functions.  Comput. Geom. Theory Appl. Vol. 22 (2002) 185-203.  (421 citations)

2.  J-D. Boissonnat, F. Cazals. Natural neighbour coordinates of points on a surface.  
Comput. Geom. Theory Appl., Vol. 19, No 2-3, July 2001. (74 citations)

3. D. Attali, J-D. Boissonnat, A. Lieutier. Complexity of the Delaunay Triangulation of Points on Surfaces~: the 
Smooth Case. 20th  ACM      Symposium on Computational Geometry, 2003. 
(74 citations)

4. D. Attali, J-D. Boissonnat. A Linear Bound on the Complexity of the Delaunay Triangulation of Points on Polyhedral Surfaces.  Discrete and Comp. Geometry 31: 369--384
(2004). (61 citations)

5. J-D. Boissonnat, S. Oudot. Provably good sampling and meshing of surfaces. Graphical Models, 67 (2005) 405-451. (198 citations. Graphical Models Top-Cited Article 2005-2010)
% including those of the conference version

6. D. Attali, J-D. Boissonnat, H. Edelsbrunner. Stability and computation of medial axes~: a state-of-the-art report.
In {\em Mathematical Foundations of Scientific Visualization,
Computer Graphics, and Massive Data Exploration},
T. Moeller,   B. Hamann and B. Russell Ed.,
Springer, series Mathematics and Visualization, 2007. (93 citations)

7. J-D. Boissonnat, D. Cohen-Steiner, G. Vegter. Isotopic implicit surface meshing.  Discrete and Computational Geometry,  39: 138-157,  2008. (55 citations)% including those of the conference version)

8. J-D. Boissonnat, C. Wormser and M. Yvinec. Locally uniform anisotropic meshing. 
24th ACM Symposium on Computational Geometry, SoCG'08.
(16 citations but,  in my opinion, one of the most important papers in this list. It  led to several significant results related to the proposal, e.g.~\cite{geometrica-7142i}).

9. J-D. Boissonnat, L. Guibas, S. Oudot. Manifold reconstruction in arbitrary dimensions using witness complexes.
Discrete and Comp. Geom. Vol 42, No 1, 2009. (46 citations)

% 8. An efficient implementation of the Delaunay triangulation and
%   its graph in medium dimension.  25th ACM Symposium on Computational
%   Geometry, SoCG'09.  Coauthors: O. Devillers and S. Hornus.

10. J-D. Boissonnat, F. Nielsen, R. Nock. On Bregman Voronoi diagrams.
Discrete and Comp. Geom. (2), 2010. (76 citations)% including those of the conference version)

%10. Triangulating Smooth Submanifolds  with Light Scaffolding.
%Mathematics in Computer Science, 4(4):431-462, 2011. Coauthor: A. Ghosh.

\vspace{-1mm}

\paragraph{Edited Books and Proceedings}  \mbox{}

$\bullet$
Algorithmic Foundations of Robotics V, Springer 2004. Coeditors:  J. Burdick, 
K. Goldberg, S. Hutchinson.
$\bullet$ Effective Computational Geometry for Curves and Surfaces,
  Springer, 2006. Coeditor:  M. Teillaud. I coauthored two chapters of this book.
$\bullet$  Curves and Surfaces.
Coeditors:  P. Chenin, A. Cohen,  C. Gout, T. Lyche, M-L.  Mazure and L. Schumaker,
Springer Verlag LNCS Vol. 6920, 2012.
$\bullet$
Geometric Computing, special issue of the 
International Journal of Computational Geometry and Applications, Vol. 11, 
No. 1, 2001.
$\bullet$
Computational Geometry, Theory and Applications, Vol. 35 No. 1-2, August 2006.
Special issue on the 20th Symposium on Computational
Geometry.  %Coeditor:   J. Snoeyink.
$\bullet$ 
Discrete and Computational Geometry, Vol. 36, No 4, December 2006.
Special issue on the 20th Symposium on Computational
Geometry.  %Coeditor:   J. Snoeyink.

\vspace{-1mm}

\paragraph{Granted patents} \mbox{}

$\bullet$  Methods and apparatus for planning robotic surgery. 
United States Patent Application 20030109780. Assignee INRIA and
Intuitive Surgical Inc. (2002).  Coauthors: E. Coste-Mani\`ere, L. Adhami,
A. Carpentier, G. Guthart.
$\bullet$  Method and apparatus for fast automatic centerline extraction for virtual 
endoscopy. United States Patent Application  20050033114. Siemens Corporate 
Research (2004). Coauthor B. Geiger.

\vspace{-1mm}

\paragraph{Keynote presentations (since 2004)}\mbox{}

$\bullet$ International Symposium on Voronoi Diagrams, Tokyo (2004).
$\bullet$  Workshop "The World a Jigsaw: Tessellations in the Sciences", Leiden (2006).
$\bullet$  French Academy of Science (2 talks, 2006). 
$\bullet$  Franco Preparata's schriftfest, Brown university (2007). 
$\bullet$ Seventh conference on "Mathematical Methods for Curves and Surfaces", Toensberg, Norway, 2008. 
$\bullet$  Colloquium on Emerging Trends in Visual Computing (ETVC, Ecole Polytechnique, 2008). 
$\bullet$  22th Sibgrapi, Rio de Janeiro (2009).  
$\bullet$  ATMCS 2012 (Algebra and Topology; Methods, Computation, and Science), Edinburgh (2012).

\vspace{-1mm}

\paragraph{Membership to editorial board of international journals}   \mbox{}


I am on the editorial board of 5 international scientific journals, including two among the most prestigious  journals in Computer Science, the {\em Journal of the ACM} and  {\em Algorithmica}, and the first venue in my field {\em Discrete and Computational Geometry}. 


% I have been on the program committee of many international conferences.
% $\bullet$  {\em Algorithmica} (1990-) $\bullet$  {\em The Int. J. on Computational Geometry and Applications} (1991-)
% $\bullet$  {\em Discrete and Computational Geometry } (2006-)
% $\bullet$  {\em The Journal of Computational Geometry} (2009-)
% $\bullet$  {\em The Journal of the ACM }(2010-)

\vspace{-1mm}

\paragraph{Organization of international conferences} \mbox{}

%I have been a member of the Computational Geometry Steering Committee (1999-2001). 
%I chaired the Symposium on Computational Geometry (the top conference of the field) in 1997 and

$\bullet$ I co-chaired the program committee of the  Symposium on Computational Geometry (SCG), the top conference of the field,  in 2004. $\bullet$ I chair the Geometry Week in conjonction with SCG 2013.
$\bullet$ 
I chaired the Workshop on Algorithmic Foundations of Robotics (WAFR) in 2002.
$\bullet$ 
I have been a member of the steering committee of SCG  (1999-2001) and of the scientific committees of the International Conference on Curves and Surfaces (2010) and of  the eighth conference on "Mathematical Methods for Curves and Surfaces" (2012).
$\bullet$ 
I have been on the program committee of  the following international conferences~: Symposium on Geometry Processing (each year since its creation in 2003), 
STACS 2001 (Symposium on Theoretical Aspects of Computer Science),
ESA 2003 (European Symposium on Algorithms),
SCG'04 (Symposium on Computational Geometry)
SMI'05 (Shape Modelling International),
SMP'05 (ACM Symposium on Solid and Physical Modeling),
Curves and Surfaces 2006,
GMP 2012 (Geometric Modeling 
and Processing).

\vspace{-1mm}

\paragraph{International prizes/awards/academy memberships} \mbox{}

$\bullet$ I received the Grand prize EADS in Information Sciences in 2006 (awarded by the French Academy of Science).  $\bullet$ I received the Graphical Models Top-Cited Article for the period 2005-2010 for  my paper with S. Oudot (ref. 5 above). I received, together with my student C. Maria, one of the two best paper awards at the European Symposium on Algorithms ESA 2012~\cite{bm-dssc-2012}.

\vspace{-1mm}

\paragraph{Scientific councils and international visiting committees (2001-)} \mbox{}


$\bullet$  Scientific Council of the Ecole Normale Sup\'erieure de Lyon (2000-2003)
$\bullet$  Member of the AERES Board (French Evaluation Agency for
  Research and Higher Education)
% $\bullet$  Member of working groups 1 (Mod\`eles et calcul) and 2
%   (Logiciels et systèmes informatiques) of Allist\`ene (Alliance des sciences et technologies du num\'erique)
$\bullet$  Chair of the Visiting Committee of LIAMA, a  national center for international research of the chinese ministry of science, hosted by the Institute of Automation of the Chinese Academy of Sciences  (Beijing, 2010)
$\bullet$  Member of the Visiting Committee of the Computer Science department of ULB (Free University of Brussels, 2011)
$\bullet$    Chair of the Visiting Committee of the Geometric Modeling and Scientific Visualization  Center of the King Abdullah University of Science and Technology (KAUST, Saudi Arabia, 2012)



