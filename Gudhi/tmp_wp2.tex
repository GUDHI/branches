% -*- LaTeX -*-
%
% 20120205
%


% intrinsic Dts
%

The Delaunay paradigm has proven to be central to the development and
understanding of meshing algorithms, whether the domain of interst is
a full dimensional subset of $\rdee$, or a more general manifold. In
order strengthen the theoretical foundations of anisotropic meshing of
Euclidean domains, and of meshing general Riemannian manifolds, we
believe it is important to develop a deeper understanding of the
Delaunay complex defined by a Riemannian metric. 

We now know that sampling conditions exist which can guarantee that
the intrinsic Delaunay complex is a triangulation of a compact
manifold, and that these conditions depend not just on the sampling
density, but also on a genericity condition. However, these conditions
have only been developed with respect to extrinsic criteria, when the
manifold is supposed to be embedded in $\rdee$. Specifically, the
sampling density condition relates extrinsic distances to the
\emph{reach} of the manifold, a quantity that has no meaning in an
intrinsic context, and the genericity condition relies on an
orthogonal projection onto the tangent space. 

We aim to develop purely intrinsic sampling criteria. We have shown
that if the Delaunay balls are sufficiently \emph{protected}, with
respect to the Euclidean metric, then locally the Delaunay
triangulation will be the same if the Riemenannian metric is close to
the Euclidean one, where the closeness is defined with respect to the
scale of the sampling. This means in particular that the intrinsic
Delaunay triangulation is locally a triangulation. However, the result
does not hold if we instead define the protection directly in terms of
the intrinsic Riemannian metric. It is a significant gap in our
understanding that we cannot express the required genericity condition
in terms of the intrinsic metric. It makes the required sampling
condtions awkward, and it unnaturally constrains our sampling density
requirements: we would require that the density is bounded with
respect to the absolute value of the sectional curvatures (in order to
obtain small metric distortion between local Euclidean metrics),
whereas we expect that a sampling density bounded by the strong
convexity radius should be possible.

After establishing the local criteria we need for correctness, we will
demonstrate that the resulting simplicial complex is homeomorphic to
the manifold. For this we cannot use the established techniques
developed for embedded manifolds, nor arguments based on the Voronoi
diagram itself, since the genericity criteria do not imply that the
closed ball property will be met. Instead we will build from techniques
described by Munkres and Cheeger. The resulting homeomorphism theorem
should be general enough to be used in other contexts.

Finally, we will develop a meshing algorithm based on these
results. This will likely use ideas from controlled perturbation, as
well as the witness complex. The intrinsic Delaunay triangulation is
an abstract simplicial complex. In order to turn it into a mesh, we
need to endow the simplices with a Euclidean structure. The obvious
thing to do is to give the edges the lengths of the geodesics between
the corresponding vertices. We will need to establish that our
genericity conditions are sufficient to ensure that the Cayley-Menger
determinant of the resulting simplices is positive, and thus that the
edge lengths do describe well-defined Euclidean simplices.


% witness cplx algoritms
%

evaluation of a polynomial of degree $m$, vs distance comparisons





% sparse sampling
