% -*- LaTeX -*-
% wp2.tex
% 20120208
%

\newcommand{\man}{\mathcal{M}}
\newcommand{\reel}{\mathbb{R}}
\newcommand{\rdee}{\reel^d}
\renewcommand{\pts}{P}
\newcommand{\mesh}{\hat{M}}

\newcommand{\ramsay}[1]{\framebox{#1}}%{\rred{[[#1]]}}

\section*{Focus Area 2: Triangulation of non Euclidean geometric spaces}

% Geometric spaces, such as the space of solutions associated with a
% nonlinear system, frequently appear as a structure of interest
% associated with a problem.  A detailed and accurate discrete model of
% such a space is the best representation possible for computational
% purposes. A simplicial complex can fill that role;
The challenge is to develop algorithms that can construct simplicial
complexes approximating manifolds and spaces which cannot be
represented as subsets of 3-dimensional Euclidean space, and to do
this efficiently and with guarantees on the approximation qualities of
the output. %Our experience leaves us uniquely positioned to take up
%this challenge.

\paragraph{Intrinsic Delaunay triangulations of Riemaniann manifolds.}

% 1. why is it important
% -- submanifolds 
% -- context where RM appear naturally without an embedding: statistical
% manifolds? discrete metric space, shape space [Mumford] 
The Delaunay paradigm has proven to be central to the development and
understanding of meshing algorithms, whether the domain of interest is
a full dimensional subset of $\rdee$, or a more general manifold. In
order to strengthen the theoretical foundations of anisotropic meshing of
Euclidean domains, and of meshing general Riemannian manifolds, we
intend to develop a deeper understanding of the
Delaunay complex defined by a Riemannian metric.
%
% 2. what is known
% -- anisotropic meshes
% -- Leibon Letscher
% Previously announced sampling criteria~\cite{leibon2000} for intrinsic
% Delaunay triangulations have recently been demonstrated to be
% insufficient~\cite{boissonnat2012stab}; in addition to density
% criteria it is essential that the points be bounded away from
% degenerate configurations. Algorithms for anisotropic meshing already
% implicitly strive to achieve this condition~\cite{bwy-luam-08}.

% 3. workplan
% -- conditions for the existence of fully intrinsic DT on RM
% -- algorithms to construct DT on RM
% -- anisotropic meshes 
% -- meshes for Bregman manifolds,
The conditions for intrinsic Delaunay triangulations have so far only
been established through extrinsic measures on manifolds embedded in
Euclidean space. We plan to establish sampling criteria based only on
intrinsic properties of the manifold. The homeomorphism demonstration
in this abstract setting will require different techniques, but likely
be more generally useful than the results so far developed for
specific substructures of an ambient Delaunay triangulation.  We will
then develop an algorithm to construct the intrinsic Delaunay complex.
Since exact computational of geodesic distances is out of reach in
many cases, in particular when the manifold is only known through a
point sample, we will develop algorithms that are robust with respect to
approximate (intrinsic) distance computations. We will take
inspiration from our recent work on anisotropic mesh generation~\cite{bwy-luam-08}.
% on a controlled
% perturbation of a given point set, rather than refining the point set
% as required in previous algorithms.
We expect that these results will
lead to insight into the meshing of spaces equipped with more general
distance-like measures such as the Bregman divergence.

\paragraph{Manifold reconstruction using Delaunay-like structures.}

% 1. why is it important
% -- alternative to DT
% -- no need for higher arithmetics
% -- usable in any discrete metric space
% -- landmark selection
The tangential complex paved the way for a
manifold reconstruction algorithm that does not depend exponentially
on the ambient dimension \cite{geometrica-7142i}. However the algorithm is still bound to
evaluate Delaunay triangulations of the dimension of the manifold, so
geometric predicates involving polynomials whose degree is that of the
manifold severely limit the dimension of the manifolds that can be
handled.
%
The witness complex~\cite{deSilva2008} is showing potential to be an
effective route to computationally efficient and conceptually simple
Delaunay meshing and reconstruction. Its computation depends on the
evaluation of much simpler geometric predicates than is required by
the Delaunay complex, and it is well defined on discrete metric
spaces, where the Delaunay complex lacks a natural
definition. However, as of now, the complexity of witness complex
based manifold reconstruction is exponential in the ambient dimension
$d$ , and whether it can be made only polynomial in $d$ remains an
open question we want to address by borrowing ideas from the
tangential complex and using insights developed from investigations of
intrinsic Delaunay triangulations, we will also develop algorithms and
guarantees for witness complexes representing manifolds that are
presented only as a discrete metric space.

%
% 2. what is known
% -- de Silva's result
% -- reconstruction in time exp. in d
% -- sufficient conditions for identity with RDT
% The witness complex is built on a set $L$ of landmarks, through
% consultation with a set $W$ of witnesses. If the set $W$ is taken to
% be all of $\rdee$, then it is known~\cite{deSilva2008}, that the
% witness complex is equal to the Delaunay complex on $L$. However, in
% practice $W$ is taken to be a dense finite set. 
% %
% Simple distance comparisons between points in $W$ and points in $L$
% are employed instead of the expensive determinant evaluations required
% by traditional Delaunay algorithms.
% %
% It has
% recently been shown~\cite{boissonnat2011cgl} that if $W$ is a finite
% set sufficiently Hausdorff close to a compact smooth manifold $\man
% \subset \rdee$, then sampling conditions for $L$ exist which ensure
% that the witness complex is equal to the Delaunay complex restricted
% to $\man$.

% 3. workplan
% -- reconstruction in time linear in k
% -- reconstruction in discrete metric spaces
% -- sampling strategy for selecting witnesses
% We will develop an algorithm for reconstructing a witness complex
% homeomorphic to a manifold, that is presented only as a dense point
% cloud. %$W \subset \rdee$. 
% The challenge is to develop a strategy for
% selecting landmark sites so that the required genericity conditions
% are met, and to do this in a way that does not introduce expensive
% geometric predicates. 


\paragraph{Crude models.}

% 1. why is it important
% -- conditions for precise reconstruction unrealistic
% -- simplicial complexes too heavy
% -- find a compromise between dimension reduction techniques and
% topological methods 
The homeomorphism guarantees obtained for reconstructing and
triangulating manifold still generally demand sampling criteria which
are not realistic in practice. Even if the manifold is known to
sufficient precision, the resulting output may be
unwieldy. In order to progress towards practical algorithms with
meaningful guarantees, satisfactory compromises must be found. 
%
% 2. what is known
% -- Collapsing
% -- Reeb graph and skeletons
% -- tree reconstruction [Chazal-Guibas]
%%%%%%
% I don't know if these examples address the issue that I think we
% want to present (I don't know tree recon). In particular, can we
% produce a Reeb graph or skeleton with quantifiable guarantees on the
% topology under significantly weaker sampling conditions than is
% required for full reconstruction? These may be lighter weight output
% structures, but I am not sure that is the issue -- the issue I see
% is how to meaninfully measure the quality of the output
% representation, given the input quality?
%
% One approach has been to forego a full representation of the manifold,
% and instead strive to obtain the associated homology groups. The
% guarantees on the output are then probabilistic, depending on the
% sampling density~\cite{nsw-fhm-2008}, which must nonetheless be high.
% This is the challenging problem at the heart of our proposal: finding
% quality measures that enable some kind of theoretical guarantees even
% in the face of corrupt and insufficient data. \ramsay{what has been
%   done really? statistical stuff: sparse representations ...}

% 3. workplan
% -- transport distance
% -- ??
We plan to explore two avenues to address the problem. On the one
hand, we will strive to attain relaxed, parameterisable, approximation quality
measures that yield a meaningful comparison between the algorithmic
output and the true manifold, given the input data. In the case where
the manifold is known to high precision, and only the output
representation is crude, evaluations based on Gromov-Hausdorff
distance, or Wasserstein-type distances will be investigated.

Another approach, appropriate when crude input data is the only
explicit information about the underlying manifold, is to assume that
the manifold belongs to some restricted family of manifolds which can
be differentiated from each other on the basis of little
information. Thus we wish to be able to prove that the output
represents the ``projection'' of the true manifold into a restricted
space of manifolds.

\paragraph{Stratified manifolds.}

% 1. why is it important
% -- C_8H_{16}
While manifold triangulation and reconstruction in higher dimensions
already represents a challenge for effective practical algorithms,
there is a need to progress to more complicated spaces than
manifolds. Stratified manifolds represent a potentially tractable yet
flexible generalisation that can model many known naturally occurring
structures. Examples include conformation spaces of molecules, such as
that discovered for cyclo-octane~\cite{mtcw-tco-2010}, and also the invariant sets that
appear in dynamical systems.
%
% 2. what is known
% -- surfaces, protecting balls
% -- continuation for bifurcations
% -- boundaries (Munkres)
Methods have been developed for meshing and reconstructing surfaces
with boundaries. Also, algorithms have been proposed for separating
the strata of stratified manifolds~\cite{bendich2007}; the resulting
strata being manifolds with boundaries. 

% 3. workplan
We plan to develop algorithms for meshing and reconstructing manifolds
with boundary with an aim for applications to stratified
manifolds. Thus a complementary task will be to triangulate a manifold
whose boundary has already been triangulated, and thus obtain a means
to reconstruct a stratified manifold by identifying the common
component strata. 