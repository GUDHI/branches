\subsection*{Focus Area 1:  Dimension-sensitive data structures} 

A central tenet in this project is that simplicial complexes are the appropriate data structure for geometry understanding in higher dimensions. However,
the algorithmic theory of simplicial complexes is in its infancy and a vigorous effort has to be undertaken to dispose of small simplicial complexes with good properties.
We will work along the following directions.



\paragraph{Delaunay-like  simplicial complexes.} 
Simplicial complexes have been derived from the Delaunay triangulation with the intent to
define more tractable simplicial complexes. Let us mention the restricted Delaunay triangulation~\cite{he-gtmg-2001}, the tangential Delaunay complex~\cite{geometrica-7142i} and the witness complex~\cite{cds-tewc-2004}. Both the restricted and the tangential Delaunay complexes are embedded in $\R^d$ and have dimension $k$ (under a mild general position assumption).  Hence, those complexes are especially designed for the case where $V$ samples a topological space of small dimension $k$, and are of special interest for meshing and reconstructing manifolds (WP2).  Still, these simplicial complexes are more delicate to compute 
than the Rips complex since they do not simply require to compute interpoint distances but also  critical points of the distance function, which involves algebraic operations whose degree depends exponentially on the dimension. Hence, they are  limited to small $k$.
The witness complex is not subject to this limitation. It is embedded in $\R ^d$ and is remarkably easy to compute in any dimension since the only numerical operations involved in its construction are comparisons of distances. Hence, it can be computed in any discrete metric space as the Rips complex. 

As for the Rips complex, the penalty for its computational simplicity is that it is not clear how the witness complex captures the topology of the sampled manifold. 
% Some equivalences between the various types of simplicial complexes are known. For example,
% the Rips and the \v{C}ech complexes are identical for the $L_{\infty}$ norm and for the Euclidean norm, we have $ \cech ({\alpha}/{2}) \subset \rips (\alpha) \subset \cech (\alpha)$. Related inclusions % properties have been established for other types of simplicial complexes, which 
% have been shown to be of primary importance to infer the homology of manifolds~\cite{co-tpr-2008}.
We want to study this question and, more generally, understand the various properties of these simplicial complexes as well as their relationships. As a  first step in that direction, we established sufficient and quite strong conditions under which the witness complex, the restricted Delaunay triangulation and the tangential complex are identical~\cite{boissonnat2012stab}. 

\paragraph{Non euclidean metric.}
Among the simplicial complexes discussed above, only the Rips and the witness complexes can be constructed on a discrete metric space where only the distances between points are known. 
Replacing the Euclidean distance in the embedding space by the geodesic distance on the object of interest results in smaller Rips complexes while keeping good approximation properties~\cite{dl-clt-2009}. We intend to study Delaunay-like simplicial complexes in the context of Riemaniann geometry. 
As mentionned above, Delaunay-like simplicial complexes are more complicated to compute. Replacing the Euclidean distance by  the geodesic distance would lead to intrinsic Delaunay simplicial complexes. First encouraging results in this direction can be found in our work on anisotropic triangulations~\cite{bwy-luam-08} and on our recent work \cite{boissonnat2012stab}. 


We also intend to extend to Delaunay-like simplicial complexes where the metric is replaced by a divergence measure such as the Bregman divergence or other information theoretic distortion measure which are not true distances and, in particular, are not symmetric nor satisfy the triangular inequality.


\paragraph{Combinatorial and algorithmic complexity.}
A main limitation to the use of simplicial complexes is their combinatorial and algorithmic complexity.  Differently from polytopes, very few results are known. The random flag complex is a noticeable exception~\cite{CambridgeJournals:2077252}. Other types of random abstract complexes as well as geometric simplicial complexes have to be studied from a combinatorial point of view. An especially important question is to obtain complexity bounds for simplicial complexes of well sampled substructures (e.g. submanifolds).  We intend to measure the effect of perturbations (either noise or computed perturbations) on the mathematical properties and combinatorial complexity of those structures, and to develop probabilistic analyses. In addition to their combinatorial complexity, the complexity of algorithms that construct the simplicial complexes is to be precisely analyzed under realistic models. In particular expected complexity and output-sensitive complexity will performed in addition to worst-case analysis. Due to the potential huge size of simplicial complexes, parallel and out-of-core algorithms will also be developed. Efficient algorithms to simplify simplicial complexes while preserving some properties such as their topological type will also be searched.


\paragraph{Compact representation of simplicial complexes.} We are aware of only a few works on the design of data structures for general simplicial complexes. Brisson~\cite{Brisson:1989:RGS:73833.73858} and Lienhardt~\cite{DBLP:journals/ijcga/Lienhardt94} have introduced data structures to represent $d$-dimensional cell complexes, most notably subdivided manifolds. While those data structures have nice algebraic properties, they are very redundant and do not scale to large data sets or high dimensions. More recently, Attali et al.~\cite{Attali2011} have proposed an efficient data structure to represent and simplify flag complexes, a special family of simplicial complexes including the Rips complex.  We intend to develop a tree representation for general simplicial complexes.  Recently, we have experimented with such a data structure
 that seems to outperform previous solutions. Simplicial complexes of 500 millions of simplices have been constructed and stored on a laptop~\cite{bm-dssc-2012}.  Theoretical guarantees and experiments on a large scale are mandatory. In addition, more compact storage could be further obtained by using succinct representations of trees and graphs~\cite{Ferragina:2005:SLT:1097112.1097456,Munro:2002:SRB:586840.586885}. The problem of finding minimal representations of simplicial complexes is widely open beyond the planar case. 











