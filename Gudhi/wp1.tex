\subsection*{Focus Area 1:  Dimension-sensitive data structures} 

A central tenet in this project is that simplicial complexes are the appropriate representation of shapes  for geometry understanding in higher dimensions. However,
the algorithmic theory of simplicial complexes is in its infancy and a vigorous effort has to be undertaken to devise small simplicial complexes with good approximation properties.
We will work along the following directions.



\paragraph{Delaunay-like  simplicial complexes.} 
% Simplicial complexes have recently been derived from the Delaunay triangulation with the intent to
% define more tractable than previously known simplicial complexes. Let us mention the restricted Delaunay triangulation~\cite{he-gtmg-2001}, the tangential Delaunay complex~\cite{geometrica-7142i} and the witness complex~\cite{cds-tewc-2004}. %Both the restricted and the tangential Delaunay 
% If the data live on a $k$-dimensional manifold of $\R ^d$, these complexes are embedded in the ambient space $\R^d$ and their dimension is generically $k$, the intrinsic dimension of the manifold.  Hence, those complexes are of special interest for meshing and reconstructing manifolds (A2).  The only known way to compute the restricted Delaunay complex of general shapes is to first compute a $d$-dimensional Delaunay triangulation, which results in an algorithm whose complexity is exponential in $d$. The tangential Delaunay is much easier to compute 
% since its construction requires to compute $k$-dimensional Delaunay simplices.
The witness complex is easy to compute since only comparisons of interpoint distances are needed. Hence the witness complex can be computed in any discrete metric space as is possible for the Rips complex. 
As for the Rips complex, the penalty for its computational simplicity is that it is not clear how the witness complex captures the topology of the sampled manifold. 
% % Some equivalences between the various types of simplicial complexes are known. For example,
% % the Rips and the \v{C}ech complexes are identical for the $L_{\infty}$ norm and for the Euclidean norm, we have $ \cech ({\alpha}/{2}) \subset \rips (\alpha) \subset \cech (\alpha)$. Related inclusions % properties have been established for other types of simplicial complexes, which 
% % have been shown to be of primary importance to infer the homology of manifolds~\cite{co-tpr-2008}.
We want to study this question and, more generally, understand the various properties of Delaunay-based simplicial complexes as well as their relationships. As a  first step in that direction, we established sufficient (albeit quite restrictive) conditions under which the witness complex, the restricted Delaunay triangulation and the tangential complex are identical~\cite{boissonnat2012stab}. 

\paragraph{Non Euclidean metric.}
% Among the simplicial complexes discussed above, only the Rips and the witness complexes can be constructed on a discrete metric space where only the distances between points are known. 
Replacing the Euclidean distance in the embedding space by the geodesic distance on the object of interest results in smaller complexes while keeping good approximation properties. We intend to study Delaunay-like simplicial complexes in the context of Riemaniann geometry. This includes Riemannian Delaunay triangulations and variants that are easier to compute. First encouraging results in this direction can be found in our work on anisotropic triangulations~\cite{bwy-luam-08} and in our recent work on Delaunay triangulations on Riemannian manifolds~\cite{boissonnat2012stab}. 


We also intend to study Delaunay-like simplicial complexes where the metric is replaced by a divergence measure such as the Bregman divergence or other information theoretic distortion measure which are not true distances and, in particular, are not symmetric nor satisfy the triangular inequality.


\paragraph{Combinatorial and algorithmic complexity.}
A main limitation to the use of simplicial complexes is their combinatorial and algorithmic complexity. %  Apart for some very special cases such as the random flag (abstract) complex~\cite{CambridgeJournals:2077252}, the combinatorics of simplicial complexes are not well known. 
We intend to study random abstract complexes as well as geometric simplicial complexes from a combinatorial and algorithmic point of view. A central question is to obtain {\em complexity bounds} for simplicial complexes of well sampled substructures (e.g. submanifolds).  We intend to measure the effect of {\em perturbations} (either noise or computed perturbations) on the mathematical properties and combinatorial complexity of those structures, and to develop {\em probabilistic analyses}. 
In addition to their combinatorial complexity, the {\em complexity of algorithms} that construct the simplicial complexes is to be precisely analyzed under realistic models. In particular expected complexity and output-sensitive complexity will be performed in addition to worst-case analysis. Due to the potential huge size of simplicial complexes, parallel and out-of-core algorithms will also be developed. % Efficient algorithms to simplify simplicial complexes while preserving some properties such as their topological type will also be searched.


\paragraph{Compact representation of simplicial complexes.}  We intend to develop efficient representations for {\em general} simplicial complexes. 
A major open question is to establish bounds on the minimal size of data structures representing simplicial complexes.
 Recently, we have experimented with a tree structure that can store all the simplices of any simplicial complex in a compact way. Simplicial complexes of 500 million simplices have been constructed and stored on a laptop~\cite{bm-dssc-2012}.  Theoretical guarantees and large scale experimentations are mandatory. In addition, more compact storage will be further obtained by developing succinct representations of trees and graphs~\cite{Ferragina:2005:SLT:1097112.1097456,Munro:2002:SRB:586840.586885}. 
\framebox{Rips filtrations?}










