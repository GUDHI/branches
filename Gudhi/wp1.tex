\subsection*{Focus Area 1:  Intrinsic dimension-sensitive data structures} 

A central tenet in this project is that simplicial complexes are the appropriate representation of shapes for geometry understanding in higher dimensions.  They will be at the heart of all the developments undertaken in all four focus areas.  The challenge is to devise small simplicial complexes with good approximation properties, to analyze their complexity and to propose succinct representations. We will work along the following directions.

% =====



% , and more recent Delaunay-like complexes such as the $\alpha$-complex~\cite{he-ubds-95}, the witness complex~\cite{cds-tewc-2004} and the Delaunay tangential complex~\cite{geometrica-7142i}. These simplicial complexes differ by their power to approximate the geometry or the topology of a shape, its dimension, its homotopy type or its differential properties. Some simplicial complexes are embedded in the ambient space (such as the Delaunay-like simplicial complexes mentionned above) while others, such as the \v{C}ech and the Rips complexes, are not. They also differ by their combinatorial and algorithmic complexities. % Nevertheless, the simplicial complexes required to approximate complicated shapes in high dimensions are huge and their construction is problematic.
% Some simplicial complexes like the \v{C}ech and the $\alpha$-complexes are extremely difficult to compute even in moderate dimensions while others like the Rips complex are easy to compute but their size is so big that they cannot be constructed from real high-dimensional data.  The recent Delaunay-like simplicial complexes offer interesting new compromises between complexity and approximation quality \cite{geometrica-7142i,cds-tewc-2004}.
% %  First investigations led to very promising results, such as the design of new simplicial complexes with good complexity and approximation algorithms that scale well with the dimension 
% Apart for some very special cases such as the random flag (abstract) complex~\cite{CambridgeJournals:2077252}, the combinatorics of simplicial complexes are not well known. 

% =====

\paragraph{Delaunay-like  simplicial complexes.} 
Simplicial complexes have recently been derived from the Delaunay
triangulation with the intent of offering better compromises between
complexity and approximation quality. Two such simplicial complexes are 
of special
interest for meshing and reconstructing manifolds in high dimensional spaces (Focus Area 2), namely
 the witness
complex~\cite{cds-tewc-2004}, and the tangential Delaunay
complex~\cite{geometrica-7142i}.
% . Both complexes are embedded in the
% ambient space and their dimension is generically $k$, the intrinsic
% dimension of the manifold. Those complexes are thus 
The
tangential Delaunay complex is fairly well understood but is more
difficult to compute than the witness complex since it requires the
computation of $k$-dimensional Delaunay triangulations, where $k$ is
the dimension of the manifold.  In contrast, the witness complex is
easy to compute since only comparisons of interpoint distances are
needed. Hence, the witness complex can be computed in any discrete
metric space, as is possible for the Rips complex. %  As for the Rips
% complex, the penalty for its computational simplicity is that it is
% not clear how the witness complex captures the topology of the sampled
% manifold (see Focus Area 2). 

 Both structures uses ideas that can be combined. The tangential complex constructs local triangulations in a way that is reminiscent of dimensionality reduction techniques. Since dimensionality reduction has been flourishing during the last decade, many variations of the basic idea of the tangential complex are conceivable, with the the potential of enlarging its usability. In addition to being easy to compute, the witness complex brings the idea of landmarking, which can also be used in conjonction with the tangential Delaunay complex. We will further look for new
variants combining dimensionality reduction, landmarking and Delaunay triangulations. 
%
% % Some equivalences between the various types of simplicial complexes are known. For example,
% % the Rips and the \v{C}ech complexes are identical for the $L_{\infty}$ norm and for the Euclidean norm, we have $ \cech ({\alpha}/{2}) \subset \rips (\alpha) \subset \cech (\alpha)$. Related inclusions % properties have been established for other types of simplicial complexes, which 
% % have been shown to be of primary importance to infer the homology of manifolds~\cite{co-tpr-2008}.
We want to understand the various properties of Delaunay-based simplicial complexes as well as their relationships. As a  first step in that direction, we established sufficient (albeit quite restrictive) conditions under which the witness complex, the restricted Delaunay triangulation and the tangential complex are identical~\cite{boissonnat2012stab}.

% \paragraph{Non Euclidean metric.}
% % Among the simplicial complexes discussed above, only the Rips and the witness complexes can be constructed on a discrete metric space where only the distances between points are known. 
% Replacing the Euclidean distance in the embedding space by the geodesic distance on the object of interest results in smaller complexes while keeping good approximation properties. We intend to study Delaunay-like simplicial complexes in the context of Riemaniann geometry. This includes Riemannian Delaunay triangulations and variants that are easier to compute. First encouraging results in this direction can be found in our work on anisotropic triangulations~\cite{bwy-luam-08} and in our recent work on Delaunay triangulations on Riemannian manifolds~\cite{boissonnat2012stab}. 


% We also intend to study Delaunay-like simplicial complexes where the metric is replaced by a divergence measure such as the Bregman divergence or other information theoretic distortion measure which are not true distances and, in particular, are not symmetric nor satisfy the triangular inequality.


\paragraph{Combinatorial and algorithmic complexity.}

Since the main limitation to using simplicial complexes is their combinatorial and algorithmic complexity, we will devote time towards better understanding those questions, both theoretically and experimentally.  % We intend to study random abstract complexes as well as geometric simplicial complexes from a combinatorial and an algorithmic point of view.
A central question is to obtain {\em complexity bounds} for simplicial complexes of well sampled substructures such as submanifolds. We will consider various types of random or deterministic simplicial complexes.  We intend to measure the effect of {\em perturbations} (either noise or computed perturbations) on the mathematical properties and combinatorial complexity of those structures, and to develop {\em probabilistic analyses}.  In addition to their combinatorial complexity, the {\em complexity of algorithms} that construct the simplicial complexes is to be precisely analyzed under realistic models. In particular, expected complexity and output-sensitive complexity analyses will be performed in addition to worst-case analysis. % Due to the potential huge size of simplicial complexes,
We will also develop parallel and out-of-core algorithms.% will also be developed. % Efficient algorithms to simplify simplicial complexes while preserving some properties such as their topological type will also be searched.


\paragraph{Compact representation of simplicial complexes.}  We will develop efficient data structures to represent {\em general} simplicial complexes. 
An open question is to establish bounds on the minimal size of data structures representing simplicial complexes. Another major question is to design succinct data structures
in the spirit of what has been done for trees and graphs~\cite{Ferragina:2005:SLT:1097112.1097456,Munro:2002:SRB:586840.586885}. 
 Theoretical guarantees and large scale experimentations on various
 types of simplicial complexes are mandatory. As a first encouraging
 step in this direction, we have experimented with a tree structure
 that can store all the simplices of any simplicial complex in a
 compact way and can support fast queries. Simplicial complexes of 500 million simplices can be routinely constructed and stored on a laptop~\cite{bm-dssc-2012}.  This data structure deserves  full analysis and optimization before it can become the standard representation of simplicial complexes. In addition to designing compact data structures, we will also consider simplifications of simplicial complexes that preserve geometric or topological properties.










