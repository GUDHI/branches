% -*- LaTeX -*-
% wp1.tex
%

\newcommand{\man}{\mathcal{M}}
\newcommand{\reel}{\mathbb{R}}
\newcommand{\rdee}{\reel^d}
\renewcommand{\pts}{P}
\newcommand{\mesh}{\hat{M}}

\newcommand{\ramsay}[1]{\rred{[[#1]]}}

\section*{WP1: Triangulation of manifolds}

We aim to construct simplicial complexes to represent
geometric spaces of interest. As an example of a space of interest we
can consider a model for the space of possible valid data points for a
system which produces data with a much higher apparent dimension than
the number of degrees of freedom  the system provides. Thus we are
considering a space of much lower intrinsic dimension than the ambient
Euclidean space which implicitly hosts the data. It is hoped that the
space of interest can be modeled as a manifold, or at least may be
decomposed into manifold parts. 

The appropriate simplicial representation will depend on the geometric
or topological information that is of interest in the model geometric
structure. Homological information can be extracted from easily
constructed, but very large complexes, or filtrations of complexes,
built from the raw input data, whereas a homeomorphic complex that
accurately captures the geometry of the underlying space requires
strong conditions on the input data, and intricate construction.


\subsection*{Geometric structures of interest}

\paragraph{Model spaces.} 
% Riemannian manifolds
The basic spaces of interest are Riemannian manifolds. These may be
given as submanifolds of an ambient Euclidean space, or they may be
manifest by data given purely in the form of inter-point distances, so
that there is no implicit ambient space.  
% Non-smooth manifolds
A natural extension of this basic model is given by a manifold endowed
with a metric that is not necessarily smooth. 

% manifolds with boundary
% stratified spaces
In the simplest case the manifold is compact and has no boundary,
however the case of manifolds with boundary will be an important one
to handle in preparation for dealing with more complicated composite
structures such as stratified spaces. Stratified spaces can be
decomposed into a sequence of manifolds of increasing
dimension. Although work has been done in extracting homological
information from such spaces~\cite{bendich2007,bendich2010}, there are
no proposed algorithms for more detailed simplicial representations.

% statistical manifolds
Another case of interest involves models where the manifold is endowed
with a geometric structure that is not necessarily defined by a
metric. Such is the case with the so-called statistical manifolds
which represent the parameter space of a family of probability
distributions. Here, a distance-like function called a divergence
takes the place of a true metric.  Voronoi diagrams and Delaunay
complexes can be defined on such
spaces~\cite{onishi1998,boissonnat2010bregvor}, but ...

\paragraph{Simplicial representations.} 
% homeomorphic representations
The 

cocone homeo -- not satisfactory
incomplete homeo \cite{cairns1961,cheng2005}
complexity of construction 

tan cplx improves the situation on both counts
 \cite{boissonnat2010tan-socg}
tan cplx homeo \cite{boissonnat2011tancplx}

wit cplx potential for further improvement in construction cost
witness \cite{boissonnat2011cgl}

Intrinsic Delaunay triangulation
bad intr \cite{leibon2000}
stab \cite{boissonnat2012stab}
tie with
witness complex (intrinsic/extrinsic)

% looser topological and geometric representations

supercomplexes (Rips/\v{C}ech complex)

graphs (Delaunay/Gabriel)

\subsection*{Algorithmic considerations}

\paragraph{Mesh generation and simplification} .\\

Delaunay refinement 

landmark selection (witness cplx) ;  controlled perturbation

other simplification techniques

\paragraph{Quality considerations} .\\

sampling conditions

topological quality

geometric quality

complexity


