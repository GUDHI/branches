% -*- LaTeX -*-
% wp1.tex
%

\newcommand{\man}{\mathcal{M}}
\newcommand{\reel}{\mathbb{R}}
\newcommand{\rdee}{\reel^d}
\renewcommand{\pts}{P}
\newcommand{\mesh}{\hat{M}}

\newcommand{\ramsay}[1]{\rred{[[#1]]}}

\section*{WP1: Triangulation of manifolds}

We aim to construct simplicial complexes to represent
geometric spaces of interest. As an example of a space of interest we
can consider a model for the space of possible valid data points for a
system which produces data with a much higher apparent dimension than
the number of degrees of freedom  the system provides. Thus we are
considering a space of much lower intrinsic dimension than the ambient
Euclidean space which implicitly hosts the data. It is hoped that the
space of interest can be modeled as a manifold, or at least may be
decomposed into manifold parts. 

The appropriate simplicial representation will depend on the geometric
or topological information that is of interest in the model geometric
structure. Homological information can be extracted from easily
constructed, but very large complexes, or filtrations of complexes,
built from the raw input data, whereas a homeomorphic complex that
accurately captures the geometry of the underlying space requires
strong conditions on the input data, and intricate construction.


\subsection*{Geometric structures of interest}

\paragraph{Model spaces.} 
% Riemannian manifolds
The basic spaces of interest are Riemannian manifolds. These may be
given as submanifolds of an ambient Euclidean space, or they may be
manifest by data given purely in the form of inter-point distances, so
that there is no implicit ambient space.  
% Non-smooth manifolds
A natural extension of this basic model is given by a manifold endowed
with a metric that is not necessarily smooth. 

% manifolds with boundary
% stratified spaces
In the simplest case the manifold is compact and has no boundary,
however the case of manifolds with boundary will be an important one
to handle in preparation for dealing with more complicated composite
structures such as stratified spaces. Stratified spaces can be
decomposed into a sequence of manifolds of increasing
dimension. Although work has been done in extracting homological
information from such spaces~\cite{bendich2007,bendich2010}, there are
no proposed algorithms for more detailed simplicial representations.

% statistical manifolds
Another case of interest involves models where the manifold is endowed
with a geometric structure that is not necessarily defined by a
metric. Such is the case with the so-called statistical manifolds
which represent the parameter space of a family of probability
distributions. Here, a distance-like function called a divergence
takes the place of a true metric.  Voronoi diagrams and Delaunay
complexes can be defined on such
spaces~\cite{onishi1998,boissonnat2010bregvor}, but ...

\paragraph{Simplicial representations.} 
% homeomorphic representations
The Delaunay paradigm has been prominent in efforts to design
simplical complexes provably homeomorphic to a smooth manifold.
Methods developed for surface reconstruction were extended to
manifolds of higher dimension~\cite{cheng2005}, producing a simplicial
complex that is a substructure of the Delaunay triangulation of the
ambient space. The principal difficulty encountered was the need to
ensure that the simplicies of the resulting structure were
sufficiently fat, i.e. had sufficiently large volume relative to their
edge lengths. This problem was addressed through the use of weighted
Delaunay triangulations. Although a sketch of a proof of sufficient
conditions for which the resulting structure is homeomorphic to the
underlying manifold has been provided \cite{cairns1961,cheng2005}, the
algorithm is not satisfactory because its complexity is exponential in
the ambient dimension.

The tangential complex~\cite{boissonnat2010tan-socg} was introduced in
order to circumvent the dependence on the ambient Delaunay
structure. Although the tangential complex is also a substructure of
the ambient Delaunay triangulation, its can be constructed without
constructing the latter structure; the resulting algorithm is
exponential in the dimension of the manifold, but only linear in the
ambient dimension, and it is demonstrated to be homeomorphic to the
underlying manifold under sufficiently dense sampling
conditions~\cite{boissonnat2011tancplx}.

The tangential complex brings medium dimensional manifold
reconstruction potentially within the reach of practical
implementation, but the construction still involves a costly weighting
scheme, and the Delaunay condition for each potential simplex must be
checked by a geometric predicate that is exponential in the dimension
of the manifold. A possible means to circumvent these problems lies in
constructing a witness complex as the simplicial representation of the
manifold. The witness complex is constructed by choosing as vertices a
relatively small subset of a dense cloud of sample points. The other
sample points serve to guide the selection of simplicies for the
witness complex, which are chosen based on predicates that only
evaluate squared distances. Recently sampling conditions have been
presented which yield a guarantee that the witness complex is the same
as the tangential Delaunay complex
\cite{boissonnat2011cgl,boissonnat2012stab}. What remains to be
developed is a practical algorithm that can guarantee these sampling
conditions. 

Another structure of interest is the Delaunay complex defined by
intrinsic distances on the manifold. Such a structure is of interest
as a theoretical model for anisotropic meshing algorithms, and also as
a target complex for data that is given in terms of interpoint
distances, without a natural isometric embedding into an ambient
Euclidean space. Previously announced sampling conditions
\cite{leibon2000} for guaranteeing that the intrinsic Delaunay complex
is a triangulation of the manifold have been shown to be insufficient
\cite{boissonnat2012stab}. Although sampling conditions were
demonstrated which guarantee that the intrinsic Delaunay complex
coincides with the tangential Delaunay complex when the manifold is
embedded in Euclidean space, these conditions are extrinsic and do not
address the situation where no such embedding is given. We aim to
develop purely intrinsic sampling conditions which guarantee that the
intrinsic Deluany complex is homeomorphic to the underlying
manifold. The witness complex is also easily defined with respect to
intrinsic distances, and we expect to acheive homeomorphism guarantees
for this structure also.

% % looser topological and geometric representations
% Taking a more relaxed interpretation of ``triangulation'', there are
% many other simplicial structures which are of interest for
% representing manifolds. For topological inference, structures
% such as the Rips-Vietoris and \v{C}hech complexes are popular. However
% these tend to be very large and of much higher dimension than the
% underlying manifold. 

% supercomplexes (Rips/\v{C}ech complex)

% graphs (Delaunay/Gabriel)

\subsection*{Algorithmic considerations}

\paragraph{Mesh generation and simplification} .\\

Delaunay refinement 

landmark selection (witness cplx) ;  controlled perturbation
(predicates exponential in $k$?)

other simplification techniques

\paragraph{Quality considerations} .\\

sampling conditions

topological quality

geometric quality

complexity


