
\subsection*{Focus Area 4 : A software platform for geometry understanding in higher
  dimensions}
We intend to develop an open source software platform of highest
quality % that will provide a comprehensive \framebox{sure?} and robust
% set of tools
for geometric understanding in high dimensions.  The goal
is not to provide code tailored to the numerous potential applications but
rather to provide the central data structures and algorithms that
underly any application in geometry understanding in higher
dimensions.

The development of such a platform will serve to benchmark and
optimize new algorithmic solutions resulting from our theoretical
work. Such development will 
 necessitate a whole line of research on software architecture and
interface design, heuristics and fine-tuning optimization, robustness
and arithmetics issues, and visualization.
%
% \paragraph{The need for a software platform.}
% On one hand, we perceive the development of such a platform as 
% an absolute necessity to serve as a test bench and  evaluation process
% for any new algorithmic solution resulting  from  our theoretical work.
% On the other hand, we perceive the development of such a platform as
% a full research work. Indeed we are convinced that, as robustness
% issues triggered the development of a whole branch of theoretical
% work, known as Geometric Computing, the need for
% highly efficient implementations to walk around the curse of  
% dimensionality,  will participate in the emergence of relevant
% concepts for new algorithmic foundations
% of  geometric understanding in high dimension.
%
% The term platform means that we intend to capitalize upon development
% efforts and encourage contributions from researchers external to the
% project.  At the end of this project, the platform will offer a
% comprehensive and robust set of algorithmic tools for geometry
% processing and analysis in higher dimensional spaces.
%
% We also think about this platform as the choice
% vector for the diffusion of our algorithmic solutions 
% into application domains, such as astrophysics
% or structural biology) 
%  where the need for handling high dimensional
% data crucially arises.  Such a diffusion in various fields
%  will in turn provide for various bench set data 
% and user feedback.
%
% \paragraph{State of the art.}  
% Leaving aside the flourishing field of machine learning
% and well-known successful software for linear algebra,
% linear and quadratic optimization,  only  few implementations handle
% geometry in high dimension. 
% Most of those software use multiscale  grids based
% data structures that somehow adapt
%  to the local  density of data.
% A typical example is the popular ANN software to compute  the approximate
% nearest-neighbor of a query point
% among a  high dimensional point cloud. 
% Qhull is a software that can compute convex hulls and Delaunay
% triangulations in dimensions larger than 3, but it doesn't see much
% development anymore, and as the authors prominently announce on the
% webpage: ``Qhull does not support triangulation of non-convex surfaces,
% mesh generation of non-convex objects, medium-sized inputs in 9-D and
% higher, alpha shapes''. The polymake framework has many features,
% it can handle several types of complexes, build Voronoi diagrams and
% compute advanced topological characteristics of objects like a finite
% representation of the fundamental group. However, it is strongly
% oriented towards an interactive use for mathematical experimentation on
% a given object and not automated, fast and robust data processing.
% % C'est l'impression que j'en ai apres avoir un peu surfe sur differents
% % sites, mais je n'ai pas une confiance absolue dans ce que je dis
% % ci-dessus.
% Only two implementations of persistent homology algorithms are
% currently available. One of them is the PLEX package for Matlab,
% developed by the Computational Topology group at Stanford University.
% The other one is the
% library Dionysus proposed by Dmitriy Morozov. These implementations
% %do not propose parallel nor out-of-core versions. They 
% are known to be
% successful in small dimensions but inefficient as soon as the
% dimension rises.  Their  maintenance, only assumed by the very few authors
% is likely not to be perennial.
%
%\framebox{Continuation methods : Multifario http://www.research.ibm.com/people/h/henderson/Continuation/ContinuationMethods.html}
%
%
% %\paragraph{Methodology.} 
% Two key requirements in the development of this platform will be
% efficiency and theoretical guarantees, including handling of
% robustness issues.
We aim at providing  a full programming environment
% % A common
% % platform is ideal for these goals, as it allows for more
% % resources to be poured into 
% the optimization, bug-fixing and interface
% design than for a single prototype.
% as well as the appropriate infrastructure. This includes a
% comprehensive documentation, a web site, a bug report mechanism and
% communication channels towards developers inside and outside the Gudhi
% project (mailing lists, developer workshops, etc.).
%
%We intend to apply to the development of this software platform the
following the very recipes that made up the success story of the CGAL
library.  

An important aspect of the work is also to educate students and young
researchers and to offer them a multidisciplinary education covering
the mathematical, algorithmic and implementational aspects of the subject.

%\paragraph{Architecture design.}

\paragraph{Data structures.}
The software platform will provide tools to extract simplicial
complexes from point clouds. We will focus our attention on simplicial
complexes that are relevant for geometric understanding, in the first
place Rips,
tangential Delaunay and witness complexes.  The platform will provide
efficient implementations of data structures to encode those
complexes. In particular, the compact data structures developed in
Focus Area 1 will be fully tested and benchmarked against other
encodings.

\paragraph{Basic algorithms.}
The platform will offer a well chosen set of basic algorithms at the
heart of geometry understanding in higher dimensions. It will include
in particular algorithms to {\em sample}, {\em mesh} and {\em
reconstruct smooth and stratified manifolds}, compute the {\em persistent homology} of
simplicial complex filtrations, {\em cluster data}, compute {\em
signatures of shapes} (see Focus Areas 2 and 3). The platform will
also provide some visualization tools.


Parallel implementations as well as out-of-core versions of some
critical algorithms %whenever possible will also be included so as to
will make possible the handling of huge data sets in high dimensions.

\paragraph{Applications.}
The results of the Gudhi project will be used and benchmarked against
real data. We have selected two main applications, one in structural
biology and the other in astrophysics, and intend to
collaborate with renowned experts in those fields. 


% guided by two is of a fundamental nature, our approach
% and, in particular the software platform, will be used in a few selected  prototype
% applications in collaboration with renowned researchers in 

We will establish strong collaborations with Prof. E. Coutsias (University
of New Mexico) and 
F. Cazals, leader of a
research group on computational structural biology at INRIA Sophia
Antipolis. A protein or any macromolecule with $n$ atoms is a flexible
system with $3n$ degrees of freedom. Dynamic molecular
simulations or Monte Carlo simulations are able to sample the {\em energy
landscape of a molecular system}. A better understanding of the
sampled conformation space of the molecule is likely to
foster our understanding of transitions between stable states,
whence of molecular functions.
We also intend to study the persistent homology
of the level sets of those molecular energy landscapes. A key question
 is to derive collective coordinates describing large amplitude -
low frequency deformations, a process reminiscent of dimensionality
reduction. We aim to identify collective coordinates incorporating
the homological constraints --- a topic which has not yet been
addressed \cite{djw-el-2003}. 

We will also collaborate with R. van de Weijgaert, an astrophysicist from
Groningen University, aiming at understanding the  phase space dynamics of
{\em cosmic structure formation}. In principle $n$-body simulations are
$6D$ datasets: each particle has 3 spatial coordinates and 3 velocity components.
In fact, lately the interest in the phase-space structure
of cosmological datasets has increased significantly. The idea is that matter of evolving cosmic
structure defines an intricate strongly non-uniform distribution in phase-space,
marked by matter streams which define ever more complicated
manifolds. At any location in space, there may be far more than 1 stream, and this
can be far better understood by looking at $6$-dimensional phase space. This is of high
interest in the search for dark matter. Regretfully, the current observational
cosmological datasets -- the galaxy surveys -- do only provide $3D$ information.
However, a very interesting higher dimensional dataset will be produced by
the European Gaia satellite in the coming years: after its schedule launch of 2013, this satellite
will make an unprecedented accurate map of the structure and distribution of
stars in our own the Galaxy and will shed new light on its formation
history. 

In addition to these two major applications, we foresee other
applications in neurosciences, medical imaging and dynamical
systems. We will leap at opportunities arising from our new results
and tools.




\paragraph{Datasets.}
To benchmark our algorithms, it is important to have access to datasets
that are representative of realistic tasks. Some datasets exist in the
machine learning community (e.g. http://archive.ics.uci.edu/ml/). We
will also use datasets from structural biology and astrophysics.  We
will create a public repository for new datasets, including synthetic
datasets with controlled parameters that will be found useful for
benchmarking against our data structures and algorithms. Providing easy
access to such data sets is an important service to the research
community of geometry understanding in higher dimensions and will help
build a standard for benchmarking algorithms in the field.


\paragraph{Infrastructure and diffusion.}
The development infrastructure will include a web site, 
comprehensive documentation, an svn repository, and regular running of
test suites on several hardware architectures.  We will set up an
editorial board in charge of the review of new packages
submitted for inclusion in the platform.  One strategic goal is to
give to the platform sufficient momentum that it will  attract
interest of a large community of researchers in computational geometry
and topology,
and various applicative fields, and serve as a standard for
experimental research. This is important for the development of new
tools as well as for the perennial maintenance of the software. In
addition to a vigorous effort towards code diffusion, we will set up
appropriate communication channels towards developers inside and
outside the Gudhi project, including mailing lists, developer
workshops, publications and talks at conferences in applied domains.















