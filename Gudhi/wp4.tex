
\subsection*{WP4 : A software platform for geometric understanding in high
  dimension}

We intend to develop an open source  software platform that will provide a
comprehensive and robust set of tools for geometric understanding in
high
dimension. 

\paragraph{The need for a software platform.}
On one hand, we perceive the development of such a platform as 
an absolute necessity to serve as a test bench and  evaluation process
for any new algorithmic solution resulting  from  our theoretical work.
On the other hand, we perceive the development of such a platform as
a full research work. Indeed we are convinced that, as robustness
issues triggered the development of a whole branch of theoretical
work, known as Geometric Computing, the need for
highly efficient implementations to turn around the curse of  
dimensionality,  will participate in the emergence of relevant
concepts for new algorithmic foundations
of  geometric understanding in high dimension.


The term platform means that we intend to capitalize upon development
efforts and encourage contributions from researchers external to the
project.  At the end of this project, the platform will offer a
comprehensive and robust set of algorithmic tools for geometry
processing and analysis in higher dimensional spaces.

We also think about this platform as the choice
vector for the diffusion of our algorithmic solutions 
into application domains, such as astrophysics
or structural biology) 
 where the need for handling high dimensional
data crucially arises.  Such a diffusion in various fields
 will in turn provide for various bench set data 
and user feedback.



\paragraph{State of the art.}  
Leaving aside the flourishing field of machine learning
and well-known successful software for linear algebra,
linear and quadratic optimization,  only  few implementations handle
geometry in high dimension. 
Most of those software use multiscale  grids based
data structures that somehow adapt
 to the local  density of data.
A typical example is the popular ANN software to compute  the approximate
nearest-neighbor of a query point
among a  high dimensional point cloud. 
Qhull is a software that can compute convex hulls and Delaunay
triangulations in dimensions larger than 3, but it doesn't see much
development anymore, and as the authors prominently announce on the
webpage: ``Qhull does not support triangulation of non-convex surfaces,
mesh generation of non-convex objects, medium-sized inputs in 9-D and
higher, alpha shapes''. The polymake framework has many features,
it can handle several types of complexes, build Voronoi diagrams and
compute advanced topological characteristics of objects like a finite
representation of the fundamental group. However, it is strongly
oriented towards an interactive use for mathematical experimentation on
a given object and not automated, fast and robust data processing.
% C'est l'impression que j'en ai apres avoir un peu surfe sur differents
% sites, mais je n'ai pas une confiance absolue dans ce que je dis
% ci-dessus.
Only two implementations of persistent homology algorithms are
currently available. One of them is the PLEX package for Matlab,
developed by the Computational Topology group at Stanford University.
The other one is the
library Dionysus proposed by Dmitriy Morozov. These implementations
%do not propose parallel nor out-of-core versions. They 
are known to be
successful in small dimensions but inefficient as soon as the
dimension rises.  Their  maintenance, only assumed by the very few authors
is likely not to be perennial.

%\framebox{Continuation methods : Multifario http://www.research.ibm.com/people/h/henderson/Continuation/ContinuationMethods.html}


\paragraph{Methodology.} 
Two key requirements in the development of this platform will be
efficiency and theoretical guarantees, including handling of
robustness issues. A common
platform is ideal for these goals, as it allows for more
resources to be poured into the optimization, bug-fixing and interface
design than for a single prototype.

 We intend to apply to the development of this software platform 
the very  recipes that made up the success story
of the CGAL library. 
The development work will be based on a strong infrastructure
including a web site, a svn repository and  daily running of testsuites on several  hardware architectures.
Above all we will set up  an editorial board assuming a  serious review of the 
specifications of packages proposed for inclusion in the platform.
We forecast that the platform will reach a sufficient critical mass
to be of interest for a large community of researchers in
computational geometry and neighboring applicative fields,
which in turn will ensure the long range perennial maintenance
of the software.



\paragraph{Planned developments} 
\begin{itemize}
\item The software platform will provide tools to extract from clouds of points the
simplicial complexes that are relevant for geometric understanding,
Rips, tangential Delaunay  and witness complexes to begin with. 
\item The platform will provide efficient data structures to handle those
complexes. We will in particular  focus on parsimonious data
structures, aiming for a partially implicit representation of those simplicial
complexes, thus avoiding the full space cost entailed by the curse of
dimensionality. 
\item The platform is meant to offer state of the art algorithmic tools for geometric
understanding,
including in particular algorithms to
\begin{itemize}
\item  mesh or reconstruct manifolds
\item  compute the persistent homology of a simplicial complex filtration 
\item cluster data
\item compute signatures of shapes
\item visualization tools \framebox{?}
\end{itemize}
\item The platform will aim at providing parallel implementations as well
as out-of-core versions whenever possible to make possible the
handling of huge data sets in high dimensions.
\end{itemize}

\framebox{Datasets}

\framebox{Diffusion}
 Based on our experience with CGAL, we will undertake a vigorous action towards code diffusion in applied domains.

%  It will namely include tools to extract from
% point clouds various simplicial complexes and filtrations,
%  like Rips, Cech or witness
% complexes, that are relavant for data analysis and topological feature
% extraction. It will provide data structures to handle those simplicial
% complexes.
% We will focus on parsimonious data structures that will contribute
% to turn around the curse of dimensionality. Particularly promising
% are data structure involving a  partial implicit representation of those simplicial
% complexes. The platform will also offer 
% robust and efficient implementations for persistent homology algorithms
% and other state of the art algorithms arising from
% work in this ERC and from neighboring groups. 

% \thispagestyle{empty}
% \pagestyle{myheadings}
% \markboth{\tcg{Titre ou acronyme de l'ADT}}{\tcg{Titre ou acronyme de l'ADT}}

% %\renewcommand{\thefootnote}{\fnsymbol{footnote}}
% %\setcounter{footnote}{1}
% %\renewcommand{\thefootnote}{\arabic{footnote}}
% %\setcounter{footnote}{0}

% \begin{center}
% {\LARGE\bf
% \tcg{Titre de l'ADT\footnote{
% Tout ce qui est en \tcg{vert} explique ce qui est attendu et est \`a remplacer
% par le texte appropri\'e \`a votre ADT. {\bf Merci d'enlever tout le \tcg{vert} au
% moment de la soumission.}
% }
% }}\\[1ex]
% \Large\bf
% Campagne ADT 2012\\
% \end{center}

% \renewcommand{\thefootnote}{\arabic{footnote}}
% %\setcounter{footnote}{0}













