% -*- LaTeX -*-

%\begin{abstract}
 %  After sound, images and videos, 3D models are everywhere and used for multimedia, video games, numerical simulations, manufacturing, computer-aided medicine, culturage heritage and other fields. In parallel to this new multimedia revolution during the past decade, exceptional progress was made in elaborating the theoretical and algorithmic foundations of 3D geometric modeling, and providing efficient algorithms and codes for applications such as surface reconstruction, mesh generation and point cloud processing.  Extending those techniques to higher dimensions is a grand challenge with a huge potential impact in science and engineering but is currently extremely limited and asks for new algorithmic  breakthrough. This project aims at settling the algorithmic foundations of geometric modeling in higher dimensions and to propose a well-principled ground-breaking software platform allowing technological advances for varied applications in science and engineering.


\paragraph{Proposal summary.} 


The central focus of this proposal is the computer analysis of
geometric structures, which we refer to as {\em geometry
  understanding}.  The need for analyzing geometric structures is
ubiquitous in science and has become an essential part of {\em scientific
computing and data analysis}. Geometry understanding is by no means
limited to 3-dimensions and many applications in physics, biology and
engineering require a keen understanding of the geometry of a variety
of higher dimensional spaces. Let us mention phase space in particle
physics, invariant manifolds in dynamical systems, configuration
spaces of mechanical systems, conformational spaces of molecules,
image manifolds, or shape spaces.
Data analysis  is another place where understanding
geometry in high-dimensional spaces is needed to
capture concise information from the underlying structure of the data.  
Since data are produced at an unprecedented rate in all
sciences, understanding the geometry of high-dimensional data %point clouds
has become a core task in science and engineering.

The central goal of this proposal is to settle the {\em algorithmic
foundations} of geometry understanding in {\em dimensions higher than
3}.  We intend to develop {\em scalable representations}, and {\em
practical algorithms} to approximate highly nonlinear shapes, and to
infer geometric and topological properties from data subject to
significant {\em defects} and under {\em realistic conditions}.
As is common in many applications across science and engineering, we
will assume that the objects of interest can be modeled as {\em
  low-dimensional manifolds} embedded in possibly high-dimensional
spaces. By exploiting the {\em intrinsic properties} of the objects,
we will produce intrinsic dimension-sensitive data structures and algorithms
that will break the current computational
bottleneck.

To reach these objectives, the guiding principle will be to foster a
symbiotic relationship between theory and practice, and to address
{\em fundamental research} issues along three parallel advancing
fronts. We will simultaneously develop {\em mathematical approaches}
providing theoretical guarantees, {\em effective algorithms} that are
amenable to theoretical analysis and rigorous experimental validation,
and {\em perennial software} development.  We will undertake the
development of a high-quality open source {\em software platform} to
implement the most important geometric data structures and algorithms
at the heart of geometry understanding in higher dimensions. The
platform will be a unique vehicle towards researchers from other
fields and will serve as a {\em basis for groundbreaking advances} in
scientific computing and data analysis.



\paragraph{Keywords :} Computational geometry, computational topology,
manifold learning, homology inference, simplicial complexes,
scientific computing, data analysis