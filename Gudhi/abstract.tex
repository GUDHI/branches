% -*- LaTeX -*-

\paragraph{Proposal summary.} 

The central goal of this proposal is to settle the algorithmic
foundations of geometry understanding in dimensions higher than 3.  We
coin the term {\em geometry understanding} to encompass a collection
%of tasks including the computer representation and the approximation
of tasks including the approximation and digital representation
of geometric structures, and the inference of geometric or topological
properties of sampled
shapes.  

The need to understand geometric structures is ubiquitous in science
and has become an essential part of {\em scientific computing} and
{\em data analysis}. 
Geometry understanding is by no means limited to
three dimensional settings, and many applications in physics, biology, and
engineering require a keen understanding of the geometry of a variety
of {\em higher dimensional spaces}. Let us mention phase space in particle
physics, invariant manifolds in dynamical systems, configuration
spaces of mechanical systems, conformational spaces of molecules,
image manifolds, and shape spaces, to name a few.  {\em Data
  analysis} is a major area where understanding
geometry in high dimensional spaces is  needed to
 capture concise information from the underlying structure of
 data. Our approach is complementary to {\em manifold learning}
 techniques and aims at  developing an effective theory for {\em geometric and
 topological data analysis.}

% The central goal of this proposal is to settle the {\em algorithmic
%   foundations} of geometry understanding in {\em dimensions higher
%   than 3}.
We intend to develop {\em scalable representations} in the
form of {\em simplicial complexes}, and {\em practical algorithms} to
approximate highly nonlinear shapes, and to infer geometric and
topological properties from data subject to significant {\em defects}
and under {\em realistic conditions}.  As is common in many
applications across science and engineering, we will assume that the
objects of interest can be modeled as {\em low-dimensional manifolds}
embedded in possibly high-dimensional spaces. By exploiting the {\em
  intrinsic properties} of the objects, we will produce intrinsic
dimension-sensitive data structures and algorithms that will break the
current computational bottleneck.

To reach these objectives, our unique approach  will be to foster a
symbiotic relationship between theory and practice, and to address
{\em fundamental research} issues along three parallel advancing
fronts. We will simultaneously develop {\em mathematical approaches}
providing theoretical guarantees, {\em effective algorithms} that are
amenable to theoretical analysis and rigorous experimental validation,
and {\em perennial software} development.  We will undertake the
development of a high-quality open source {\em software platform} to
implement the most important geometric data structures and algorithms
at the heart of geometry understanding in higher dimensions. The
platform will be a unique vehicle to reach out to researchers from other
fields and will serve as a {\em basis for groundbreaking advances} in
scientific computing and data analysis.

\paragraph{Keywords :} Computational geometry, computational topology,
manifold learning, homology inference, simplicial complexes,
scientific computing, data analysis
