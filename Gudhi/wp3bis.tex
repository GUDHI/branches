%\paragraph{Noise models.}
Assuming that the data are corrupted by noise of small amplitude succesful robust methods have been proposed for the reconstruction and the estimation of topological and geometric  properties of shapes. They are usually based on the study of the topology of the level sets of the distance function to the data that can be related to the topology of the underlying shape when the Hausdorff distance  between the data and the shape is small enough (small amplitude noise). As previously mentioned such an assumption on the noise does not comply with many applications where the data come corrupted by non local noise. 
%Moreover most of the methods relying on the closeness of the data to the shape with respect to the Hausdorff distance are deterministic and do not take into account the statistical nature of the %noise.  
Recently more statistical approaches allowing to deal with larger classes of noise models have been considered to infer geometric information from data. Some of them intend to remove the part of the data that is far away from the underlying shape and to make use of  the distance function framework \cite{nsw-tvu-2011} but they assume very restrictive noise models. On another hand purely statistical approaches have been proposed for shape approximation that work for large families of noise models (see, e.g., \cite{gpvw-mme-2011,gpvw-mesd-2011}) but they do not come with topological guarantees on the approximated geometric shapes and do not always provide explicite estimates. A major remaining challenge is the design of new unifying frameworks that embrace the statistical approaches and the deterministic methods coming with topological guarantees.  
This a goal we set to ourselves in this project for three different central problems: homology inference, reconstruction, and signature design.


%We intend to develop practically efficient tools for robust topological and geometric inference that work with large classes of models of noise. 


\paragraph{Homology inference.}
To comply with the presence of noise and outliers in the data a new paradigm for point cloud data analysis has emerged recently, where point clouds are no
longer treated as mere compact sets but rather as empirical measures. A notion of distance to
such measures has been defined and shown to be stable with respect to perturbations of the
measure \cite{ccsm-gipm-2011}. This distance can easily be computed pointwise in the case of a point cloud (simply
average the squared distances to the k nearest neighbors), but its sublevel-sets, which carry the
geometric information about the measure (or the underlying shape if we consider a model where the data is generated from a measure on the shapes corrupted by some noise), remain hard to compute or approximate. A big challenge now is to find efficient algorithms in arbitrary dimensions to compute or approximate
the topological structure of the sublevel-sets of the distance to a measure, in the same spirit as
what was done in the recent years for distances to compact sets. Such algorithms would naturally
find applications in topological inference in the presence of significant noise and outliers, but
also in other less obvious contexts such as stable clustering. The current bottleneck is that
there exist no equivalents of the union of balls and alpha-shape in the case of the distance to
a measure. Our first goal will be to work out such equivalents that will allow to infer the homology of the underlying shape or more generally the topological persistence of the distance to measure functions.
%To start with, we will focus on
%medium dimensions and use a variant of the mesh-based inference algorithm [14] to approximate
%the sublevel-sets of the distance to a measure and get an idea of their topological structure.