\section{Research proposal}

The central goal of this proposal is to settle the {\em algorithmic foundations of geometry understanding in higher dimensions}.  We aim at processing general highly nonlinear geometries with nontrivial topologies that can be modeled as  {\em low-dimensional manifolds} embedded in possibly high-dimensional spaces. 

\subsection{State-of-the-art and objectives}
%The last decade has seen tremendous progress in the understanding of geometry in high-dimensional spaces. 

\paragraph{Dimensionality reduction.} A widely accepted assumption in scientific computing and data analysis is that the objects of interest can be modeled as {\em low-dimensional manifolds}, even if they are embedded in high-dimensional ambient spaces. Low here is to be understood with respect to the ambient dimension and may be significantly higher than 3. This powerful assumption is supported by the fact that data are most of the time associated with physical systems that have relatively few parameters.  This assumption is valid accross science and engineering and is at the heart of dimensionality reduction and manifold learning.
%Examples can be found in fields as varied as chemistry~\cite{mtcw-tco-2010}, %neurosciences~\cite{} and  image processing~\cite{cids-lbsni-2008}.

{\em Dimensionality reduction} techniques intend to infer the intrinsic dimensionality of the data, as well as to provide structure-preserving mappings of the data into lower-dimensional spaces. Nonlinear dimensionality reduction techniques~\cite{lv-nldr-2007} are capable of discovering nonlinear structures in the data and have been successfully applied to analyze data in a wide number of applications.
Nevertheless, the methods come with no or very limited guarantees. For example, Isomap~\cite{tsl-isomap-2000} provides a correct embedding only if the manifold is isometric to a convex open set of $\R ^k$, where $k$ is the estimated dimension of the manifold, and LLE~\cite{rs-lle-2000} can only reconstruct topological balls. {\em Topological methods} are complementary to dimension reduction methods. They intend to better approximate manifolds by constructing piecewise linear (PL) approximations in the form of simplicial complexes.

%However, they are only guaranteed to work on data sets sampled from manifolds with low curvature and trivial topology.

\paragraph{Simplicial complexes.}  Simplicial complexes can be defined in any dimension and can be used as piecewise linear approximations of general shapes.
Such representations have been extensively used in $\R ^3$ where surfaces or volumes are approximated by 2-dimensional or 3-dimensional simplicial complexes~\cite{geometrica-ecg-book}. 

Various types of simplicial complexes have been proposed such as the \v{C}ech and the Rips complexes, and more recent Delaunay-like complexes such as the $\alpha$-complex~\cite{he-ubds-95}, the witness complex~\cite{cds-tewc-2004} and the Delaunay tangential complex~\cite{geometrica-7142i}. These simplicial complexes differ by their power to approximate the geometry or the topology of a shape, its dimension, its homotopy type or its differential properties. Some simplicial complexes are embedded in the ambient space (such as the Delaunay-like simplicial complexes mentionned above) while others, such as the \v{C}ech and the Rips complexes, are not. They also differ by their combinatorial and algorithmic complexities. % Nevertheless, the simplicial complexes required to approximate complicated shapes in high dimensions are huge and their construction is problematic.
Some simplicial complexes like the \v{C}ech and the $\alpha$-complexes are extremely difficult to compute even in moderate dimensions while others like the Rips complex are easy to compute but their size is so big that they can be constructed from real high-dimensional data.  The recent Delaunay-like simplicial complexes offer interesting new compromises between complexity and approximation quality \cite{geometrica-7142i,cds-tewc-2004}.
%  First investigations led to very promising results, such as the design of new simplicial complexes with good complexity and approximation algorithms that scale well with the dimension 

{\bf Objective 1}:  to extend  current knowledge on simplicial complexes, most notably on their combinatorial and algorithmic properties, and   to design new  {\em intrinsic dimension-sensitive data  structures} to construct and store such simplicial complexes.

\paragraph{Geometric approximation.}
In low dimensions, computing an approximation of a given geometric object is a well-understood problem and good approximations can be efficiently constructed~\cite{geometrica-bcmrv-ms-06,he-gtmg-2001}.  The situation is quite different in higher dimensions.  Although the mathematical literature on triangulation of manifolds is abundant, few effective algorithms have been proposed and tested.  To analyze {dynamical systems} in science or engineering, higher-dimensional {\em continuation methods} have been proposed to mesh solution manifolds~\cite{mh-mpc-2002}. These methods are however lacking guarantees and are restricted in practice to low dimensional manifolds. 
We made recently some progress and proposed a provably correct algorithm to mesh smooth submanifolds~\cite{boissonnat2010meshing}.

Another fundamental problem is {\em manifold reconstruction}.  In low dimensions, effective reconstruction techniques exist that can provide a faithful approximation of a geometric structure from point samples~\cite{dey-csr-2007}. % Further processing makes it possible to study their topological and geometric properties.
Extending those techniques to higher dimensions is a major challenge.  First, the data often suffers from significant defects, including sparsity, noise, and outliers, violating sampling conditions required by extant methods. The problem is further compounded by the rapid growth in complexity of the data structures used for reconstruction as the dimensionality of the ambient space increases, which makes them intractable in high dimensions.%  since their complexity depends exponentially on the ambient dimension
 Despite some very recent results~\cite{geometrica-7142i}, designing practical algorithms that can reconstruct submanifolds of high-dimensional spaces under mild sampling conditions remains widely open.  

{\bf Objective 2 :}  to   develop new algorithms to {\em triangulate non Euclidean geometric spaces}, and to mesh or reconstruct manifolds equipped with various metrics.

\paragraph{Topological inference.}
Since computing precise approximations is currently only possible under strong assumptions that may not be met in some applications, we can look for cruder approximations 
that can still  uncover some of the properties of the structures underlying the data.
%
%is not always mandatory. In some applications,   useful approximations can be obtained at a lower computational cost.
% In {\em robotics}, the goal is to capture the connectivity of configuration spaces and to search paths. Randomized techniques have been proved to be quite successful in constructing graphs to approximate the space of free configurations of robots~\cite{sml-pa-2006}. Those techniques are however limited to simple mechanical systems with no redundancy, no loop nor kinematic constraints. \framebox{check}
%In {\em topological data analysis}, 
%
A prominent example is homology that can be computed  without a precise reconstruction and under less restrictive conditions~\cite{geometrica-ccl09,nsw-fhm-2008}. The rapidly growing theory of {\em persistent homology}~\cite{eh-ph-2008,rg-bptd-2008} was recently introduced as a powerful tool for the study of the topological invariants of sampled shapes. The approach consists of building a simplicial complex whose elements are filtered by some user-defined function. The filtration is then used to remove topological noise and to report the stable topological features.  These advances in computational topology attracted interest in the mathematical community and in several fields like neurosciences, computer vision or sensor networks~\cite{cids-lbsni-2008,rg-bptd-2008}. Again, the bottleneck that still prevents applications from benefiting of the full potential of these new methods is the lack of efficient data structures and algorithms to construct simplicial complexes and filtrations in high dimensions, and the lack of methods that are provably stable with respect to  noisy and defect-laden inputs.


% \paragraph{Stable models.} 
% When dealing with approximation and samples, one needs stability results to ensure that the quantities that are computed are good approximations of the real ones. This is especially true in higher-dimensions where data are usually corrupted by various types of noise.  When the noise is of small amplitude, methods have been proposed to robustly estimate topological and geometric properties of shapes~\cite{geometrica-ccl09,nsw-tvu-2011}. 
%  The recent and fast developing theory of {\em persistent homology} provides a powerful tool to study  the homology of sampled spaces~\cite{eh-ph-2008}.
% However, in  many applications the noise is non local and the previous methods fail.
% Recently,  larger families of noise models  have been considered and statistical approaches  have been proposed to approximate shape under  those models~\cite{gpvw-mme-2011}. These methods however do not provide topological guarantees on the approximation and do not always provide explicit estimates \framebox{check} . A major challenge is thus to design  unifying frameworks that embrace statistical approaches and deterministic methods, and offer topological guarantees.   Statistical techniques are also needed to automatically select the relevant scales at which the geometry of data should be considered. \framebox{un peu sec}

{\bf Objective 3 :} to study new {\em robust models for homology inference, comparison and  clustering}  and to provide the crucial  algorithms for topological data analysis.

\paragraph{Theory versus practice.} 
Geometric and topological methods are well behind dimensionality reduction techniques in terms of 
software development and applications.  
%
% Breaking the computational bottleneck is now the main issue.  Settling the {\em algorithmic foundations} of geometry understanding in
% higher dimensions is a grand challenge of great theoretical and practical significance.
%
To go beyong these low-dimensional examples, one needs efficient and robust software to construct and manipulate simplicial complexes in dimensions higher than 3.  Only a very few such software exist.  {\em Qhull} can compute convex hulls and Delaunay triangulations in moderate dimensions, but is of little use in the context of geometry understanding since it only constructs full-dimensional triangulations. {\em Multifario} is a set of subroutines and data structures dedicated to {\em meshing} manifolds that occur in dynamical systems using a multiple parameter continuation approach. The software can approximate 2 and 3-manifolds embedded in higher dimensional spaces. No theoretical guarantees nor extensive experiments are reported.  {\em Polymake} can handle several types of complexes, build Voronoi diagrams and compute advanced topological characteristics of objects like a finite representation of the fundamental group. It is however more oriented towards an interactive use for mathematical experimentation
rather than suited to an automated use at large scale.%, fast and robust data processing.
% C'est l'impression que j'en ai apres avoir un peu surfe sur differents
% sites, mais je n'ai pas une confiance absolue dans ce que je dis
% ci-dessus.

Several libraries exist for homology computation. RedHom computes Betti numbers and torsion coefficients of cubical sets, simplicial complexes and general, regular CW complexes using geometric reductions/coreductions and discrete Morse theory (http://redhom.ii.uj.edu.pl/).
Two implementations of {\em persistent homology} algorithms are currently available, PLEX a package for Matlab (http://comptop.stanford.edu/u/programs/jplex/), and  the Dionysus library (http://www.mrzv.org/software/dionysus/). They both offer the construction of several types of simplicial complexes. PLEX has been  successfully applied in low dimensions~\cite{fpgo-airc-2009,rg-bptd-2008,mtcw-tco-2010}. Dionysus offers advanced
functionalities (such as persistent cohomology computation and 
Zigzag persistent homology).
%Dionysus constructs $\alpha$-shapes, Cech and Rips complexes. Various options to compromize between memory and efficiency
%Plex : rips, landmark selection , ex k=1, d=25

{\bf Objective 4 :}   to develop a {\em  software platform for geometric understanding in high dimensions} that will provide the software environment for experimenting with our new data structures and algorithms, to integrate them in a coherent library of interoperable modules, and to diffuse our results to applied fields. 

\subsection{Research roadmap} 


% This proposal addresses {\em fundamental
%   research} issues, and its results are expected to serve as a basis
% for groundbreaking advances for {\em applications in scientific computing
% and data analysis}. 
% A major outcome of the project will be a
% high-quality open source software {\em platform} of components
% implementing the main results. 
%
To reach these objectives, the guiding principle  will be to simultaneously
develop {\em mathematical approaches} providing theoretical
guarantees, {\em effective algorithms} that are amenable to both
theoretical analysis and rigorous experimental validation, and {\em
  perennial software} development.


The proposal is structured into four areas that focus on the four objectives mentionned above, which we now describe in more detail. 


\subsection*{Focus Area 1:  Dimension-sensitive data structures} 

A central tenet in this project is that simplicial complexes are the appropriate data structure for geometry understanding in higher dimensions. However,
the algorithmic theory of simplicial complexes is in its infancy and a vigorous effort has to be undertaken to dispose of small simplicial complexes with good approximation properties.
We will work along the following directions.



\paragraph{Delaunay-like  simplicial complexes.} 
Simplicial complexes have recently been derived from the Delaunay triangulation with the intent to
define more tractable simplicial complexes. Let us mention the restricted Delaunay triangulation~\cite{he-gtmg-2001}, the tangential Delaunay complex~\cite{geometrica-7142i} and the witness complex~\cite{cds-tewc-2004}. %Both the restricted and the tangential Delaunay 
If the data live on a $k$-dimensional manifold of $\R ^d$, these complexes are embedded in the ambient space $\R^d$ and their dimension is $k$, the intrinsic dimension of the manifold.  Hence, those complexes are of special interest for meshing and reconstructing manifolds (A2).  The only known way to compute the restricted Delaunay complex of general shapes is to compute first a $d$-dimensional Delaunay triangulation, which results in an algorithm whose complexity is exponential in $d$. The tangential Delaunay is much easier to compute 
since its construction requires to compute $k$-dimensional Delaunay simplices.
The witness complex is even easier to to compute since only comparisons of interpoint distances are needed. Hence the witness complex can be computed in any discrete metric space as the Rips complex. 
As for the Rips complex, the penalty for its computational simplicity is that it is not clear how the witness complex captures the topology of the sampled manifold. 
% Some equivalences between the various types of simplicial complexes are known. For example,
% the Rips and the \v{C}ech complexes are identical for the $L_{\infty}$ norm and for the Euclidean norm, we have $ \cech ({\alpha}/{2}) \subset \rips (\alpha) \subset \cech (\alpha)$. Related inclusions % properties have been established for other types of simplicial complexes, which 
% have been shown to be of primary importance to infer the homology of manifolds~\cite{co-tpr-2008}.

We want to study this question and, more generally, understand the various properties of Delaunay-based simplicial complexes as well as their relationships. As a  first step in that direction, we established sufficient (but unfortunately quite strong) conditions under which the witness complex, the restricted Delaunay triangulation and the tangential complex are identical~\cite{boissonnat2012stab}. 

\paragraph{Non Euclidean metric.}
% Among the simplicial complexes discussed above, only the Rips and the witness complexes can be constructed on a discrete metric space where only the distances between points are known. 
Replacing the Euclidean distance in the embedding space by the geodesic distance on the object of interest results in smaller Rips complexes while keeping good approximation properties~\cite{dl-clt-2009}. The same will We intend to study Delaunay-like simplicial complexes in the context of Riemaniann geometry. 
As mentionned above, Delaunay-like simplicial complexes are more complicated to compute. Replacing the Euclidean distance by  the geodesic distance would lead to intrinsic Delaunay simplicial complexes. First encouraging results in this direction can be found in our work on anisotropic triangulations~\cite{bwy-luam-08} and in our recent work on Delaunay triangulations on Riemannian manifolds~\cite{boissonnat2012stab}. 


We also intend to study Delaunay-like simplicial complexes where the metric is replaced by a divergence measure such as the Bregman divergence or other information theoretic distortion measure which are not true distances and, in particular, are not symmetric nor satisfy the triangular inequality.


\paragraph{Combinatorial and algorithmic complexity.}
A main limitation to the use of simplicial complexes is their combinatorial and algorithmic complexity.  Apart from the random flag (abstract) complex~\cite{CambridgeJournals:2077252}, the combinatorics of simplicial complexes are not well known. Other types of random abstract complexes as well as geometric simplicial complexes have to be studied from a combinatorial point of view. An especially important question is to obtain {\em complexity bounds} for simplicial complexes of well sampled substructures (e.g. submanifolds).  We intend to measure the effect of {\em perturbations} (either noise or computed perturbations) on the mathematical properties and combinatorial complexity of those structures, and to develop {\em probabilistic analyses}. 
In addition to their combinatorial complexity, the {\em complexity of algorithms} that construct the simplicial complexes is to be precisely analyzed under realistic models. In particular expected complexity and output-sensitive complexity will performed in addition to worst-case analysis. Due to the potential huge size of simplicial complexes, parallel and out-of-core algorithms will also be developed. % Efficient algorithms to simplify simplicial complexes while preserving some properties such as their topological type will also be searched.


\paragraph{Compact representation of simplicial complexes.} We are aware of only a few works on the design of data structures for general simplicial complexes. Brisson~\cite{Brisson:1989:RGS:73833.73858} and Lienhardt~\cite{DBLP:journals/ijcga/Lienhardt94} have introduced data structures to represent $d$-dimensional cell complexes, most notably subdivided manifolds. While those data structures have nice algebraic properties, they are very redundant and do not scale to large data sets or high dimensions. More recently, Attali et al.~\cite{Attali2011} have proposed an efficient data structure to represent and simplify flag complexes, a special family of simplicial complexes including the Rips complex.  We intend to develop representations for {\em general} simplicial complexes. 
A major open question is to establish bounds on the minimal size of data structures representing simplicial complexes.
 Recently, we have experimented with a tree structure that can represent any simplicial complex and seems to outperform previous solutions. Simplicial complexes of 500 millions of simplices have been constructed and stored on a common laptop~\cite{bm-dssc-2012}.  Theoretical guarantees and experimentations on a large scale are mandatory. In addition, more compact storage could be further obtained by using succinct representations of trees and graphs~\cite{Ferragina:2005:SLT:1097112.1097456,Munro:2002:SRB:586840.586885}. 
\framebox{Rips filtrations?}











% -*- LaTeX -*-
% wp2.tex
% 20120208
%

\newcommand{\man}{\mathcal{M}}
\newcommand{\reel}{\mathbb{R}}
\newcommand{\rdee}{\reel^d}
\renewcommand{\pts}{P}
\newcommand{\mesh}{\hat{M}}

\newcommand{\ramsay}[1]{\rred{[[#1]]}}

\section*{Focus Area 2: Triangulation of non Euclidean geometric spaces}

\paragraph{Delaunay-like triangulations of Riemaniann manifolds.}

% 1. why is it important
% -- submanifolds 
% -- context where RM appear naturally without an embedding: statistical
% manifolds? discrete metric space, shape space [Mumford] 
The Delaunay paradigm has proven to be central to the development and
understanding of meshing algorithms, whether the domain of interst is
a full dimensional subset of $\rdee$, or a more general manifold. In
order strengthen the theoretical foundations of anisotropic meshing of
Euclidean domains, and of meshing general Riemannian manifolds, we
believe it is important to develop a deeper understanding of the
Delaunay complex defined by a Riemannian metric. %\ramsay{applications}
%
% 2. what is known
% -- anisotropic meshes
% -- Leibon Letscher
Previously announced sampling criteria~\cite{leibon2000} for intrinsic
Delaunay triangulations have recently been demonstrated to be
insufficient~\cite{boissonnat2012stab}; in addition to density
criteria it is essential that the points be bounded away from
degenerate configurations. Algorithms for anisotropic meshing already
implicitly strive to achieve this condition~\cite{bwy-luam-08}.

% 3. workplan
% -- conditions for the existence of fully intrinsic DT on RM
% -- algorithms to construct DT on RM
% -- anisotropic meshes 
% -- meshes for Bregman manifolds,
The conditions for intrinsic Delaunay triangulations have so far only
been established through extrinsic measures on manifolds embedded in
Euclidean space. We plan to establish sampling criteria based only on
intrinsic properties of the manifold. The homeomorphism demonstration
in this abstract setting will require different techniques, but likely
be more generally useful than the results so far developed for
specific substructures of an ambient Delaunay triangulation.  We will
then develop an algorithm to construct the intrinsic Delaunay complex.

Building on these ideas, we intend to develop a high dimensional
anisotropic meshing algorithm which constructs a mesh on a controlled
perturbation of a given point set, rather than refining the point set
as required in previous algorithms. We expect that these results will
lead to insight into the meshing of spaces equipped with more general
distance-like measures such as the Bregman divergence.

\paragraph{Witness complex.}

% 1. why is it important
% -- alternative to DT
% -- no need for higher arithmetics
% -- usable in any discrete metric space
% -- landmark selection
The witness complex~\cite{deSilva2008} is showing potential to be an
effective route to computationally efficient and conceptually simple
Delaunay meshing and reconstruction. Its computation depends on the
evaluation of much simpler geometric predicates than is required by
the Delaunay complex, and it is well defined on discrete metric
spaces, where the Delaunay complex lacks a natural definition. 
%
% 2. what is known
% -- de Silva's result
% -- reconstruction in time exp. in d
% -- sufficient conditions for identity with RDT
The witness complex is built on a set $L$ of landmarks, through
consultation with a set $W$ of witnesses. If the set $W$ is taken to
be all of $\rdee$, then it is known~\cite{deSilva2008}, that the
witness complex is equal to the Delaunay complex on $L$. However, in
practice $W$ is taken to be a dense finite set. Whereas the Delaunay
complex has a complexity of $\mathcal{O}(L^d)$, current algorithms for
the construction of the witness complex have a complexity of
$\mathcal{O}(W+L^d)$, \ramsay{?? a surcharge that we hope will be offset
  by the simplicity of the geometric predicates involved.}  It has
recently been shown~\cite{boissonnat2011cgl} that if $W$ is a finite
set sufficiently Hausdorff close to a compact smooth manifold $\man
\subset \rdee$, then sampling conditions for $L$ exist which ensure
that the witness complex is equal to the Delaunay complex restricted
to $\man$.

% 3. workplan
% -- reconstruction in time linear in k
% -- reconstruction in discrete metric spaces
% -- sampling strategy for selecting witnesses
We will develop an algorithm for reconstructing a witness complex
homeomorphic to a manifold, that is presented only as a dense point
cloud $W \subset \rdee$. The challenge is to develop a strategy for
selecting landmark sites so that the required genericity conditions
are met, and to do this in a way that does not introduce expensive
geometric predicates. The complexity of the algorith should depend
exponentially on the dimension of the manifold, but only linearly in
the ambient dimension. Using insights developed from investigations of
intrinsic Delaunay triangulations, we will also develop algorithms and
guarantees for witness complexes representing manifolds that are
presented only as a discrete metric space.

\paragraph{Crude models.}

% 1. why is it important
% -- conditions for precise reconstruction unrealistic
% -- simplicial complexes too heavy
% -- find a compromise between dimension reduction techniques and
% topological methods 
The homeomorphism guarantees obtained for reconstructing and
triangulating manifold still generally demand sampling criteria which
are not realistic in practice. Even if the manifold is known to
sufficient precision, the resulting output may be
unwieldy. In order to progress towards practical algorithms with
meaningful guarantees, satisfactory compromises must be found. 
%
% 2. what is known
% -- Collapsing
% -- Reeb graph and skeletons
% -- tree reconstruction [Chazal-Guibas]
%%%%%%
% I don't know if these examples address the issue that I think we
% want to present (I don't know tree recon). In particular, can we
% produce a Reeb graph or skeleton with quantifiable guarantees on the
% topology under significantly weaker sampling conditions than is
% required for full reconstruction? These may be lighter weight output
% structures, but I am not sure that is the issue -- the issue I see
% is how to meaninfully measure the quality of the output
% representation, given the input quality?
One approach has been to forego a full representation of the manifold,
and instead strive to obtain the associated homology groups. The
guarantees on the output are then probabilistic, depending on the
sampling density~\cite{nsw-fhm-2008}, which must nonetheless be high.

% 3. workplan
% -- transport distance
% -- ??
We plan to explore two avenues to address this situation. On the one
hand we will strive to attain relaxed, parameterisable, approximation quality
measures that yield a meaningful comparison between the algorithmic
output and the true manifold, given the input data. In the case where
the manifold is known to high precision, and only the output
representation is crude, evaluations based on Gromov-Hausdorff
distance, or Wasserstein-type distances will be interesting.

Another approach, appropriate when crude input data is the only
explicit information about the underlying manifold, is to assume that
the manifold belongs to some restricted family of manifolds which can
be differentiated from each other on the basis of little
information. Thus we wish to be able to claim that the output
represents the ``projection'' of the true manifold into a restricted
space of manifolds.

\paragraph{Stratified manifolds.}

% 1. why is it important
% -- C_8H_{16}
While manifold triangulation and reconstruction in higher dimensions
already represents a challenge for effective practical algorithms,
there is a need to progress to more complicated spaces than
manifolds. Stratified manifolds represent a potentially tractable
yet flexible generalisation 
that can model many known naturally occurring structures. Examples
include conformation spaces of molecules, such as that discovered for
cyclo-octane, and also the invariant sets that appear in dynamical
systems. 
%
% 2. what is known
% -- surfaces, protecting balls
% -- continuation for bifurcations
% -- boundaries (Munkres)
Methods have been developed for meshing and reconstructing surfaces
with boundaries. Also, algorithms have been proposed for separating
the strata of stratified manifolds~\cite{bendich2007}; the resulting
strata being manifolds with boundary.

% 3. workplan
We plan to develop algorithms for meshing and reconstructing manifolds
with boundary with an aim for applications to stratified
manifolds. Thus a complementary task will be to triangulate a manifold
whose boundary has already been triangulated, and thus obtain a means
to reconstruct a stratified manifold by identifying the common
component strata.

\subsection*{Focus Area 3:  Robust models for geometric and topological inference}%, clustering and comparison}


%Remarques a inserer dans l'intro generale du projet (version longue, plus eventuellement version courte):
%
%- La majorite des donnees proviennent de simulations ou de prises mesures.  Elles sont generalement corrompues par du bruit, des outliers, et elles sont souvent parcellaires. Il est donc %important de disposer de techniques permettant de reparer ces defauts, ou bien directement de methodes d'analyse qui n'y soient pas sensibles.
%
%- Le fait de considerer des donnees en grandes dimensions ou dans des espaces "compliques" empeche l'exploration visuelle directe des donnees, mais egalement la validation visuelle des %resultats obtenus par l'analyse. Il y a donc besoin de methodes de visualisation de la structure de telles donnees qui viennent avec des garanties sur la pertinence des resultats, de sorte que %l'on soit assure que ce qui est montre par ces methodes est bien de l'information et non du bruit.



%\paragraph{Noise models.}
% Assuming that the data are corrupted by noise of small amplitude successful robust methods have been proposed for the estimation of topological and geometric  properties of shapes. They are usually based on the study of the topology of the level sets of the distance function to the data that can be related to the topology of the underlying shape when the Hausdorff distance  between the data and the shape is small enough (small amplitude noise). As previously mentioned such an assumption on the noise does not comply with many applications where the data come corrupted by non local noise. 
% %Moreover most of the methods relying on the closeness of the data to the shape with respect to the Hausdorff distance are deterministic and do not take into account the statistical nature of the %noise.  
% Recently more statistical approaches allowing to deal with larger classes of noise models have been considered to infer geometric information from data. Some of them intend to remove the part of the data that is far away from the underlying shape and to make use of  the distance function framework \cite{nsw-tvu-2011} but they assume very restrictive noise models. On another hand purely statistical approaches have been proposed for shape approximation that work for large families of noise models (see, e.g., \cite{gpvw-mme-2011,gpvw-mesd-2011}) but they do not come with topological guarantees on the approximated geometric shapes and do not always provide explicit estimates. A major remaining challenge is the design of new unifying frameworks that embrace the statistical approaches and the deterministic methods coming with topological guarantees.  
% This is a goal we set to ourselves in this project for three different central problems: homology inference, clustering, and signature design.


%We intend to develop practically efficient tools for robust topological and geometric inference that work with large classes of models of noise. 


The goal is to infer geometric and topological properties from defect-laden data.
 A major challenge is to combine  statistical approaches relying on powerful models of noise and deterministic methods coming with topological guarantees.  
% This is a goal we set to ourselves in this project for three different central problems: homology inference, clustering, and signature design.

\framebox{on ne parle plus de stat ensuite}


\paragraph{Homology inference.}
Building on the distance function approach efficient and guaranteed algorithms have been developed to infer the homology of general shapes from \v{C}ech or Rips complexes built on top of the data when they are sampled at a small Hausdorff distance of the considered shape \cite{co-tpr-2008}. These algorithms rely on the idea that the topology of the shape is carried by the topology of some unions of balls centered on the data (i.e. the sublevel sets of the distance function to the data) that can itself be related to the topology of the $\alpha$-shape \framebox{not defined before. really needed?} and Rips complexes.  

To comply with the presence of noise and outliers in the data we intend to explore different approaches inspired from these algorithms.  We will in particular focus on a new paradigm for point cloud data analysis that has emerged recently, where point clouds are no longer treated as mere compact sets but rather as empirical measures. A notion of distance to such measures has been defined and shown to be stable with respect to perturbations of the measure \cite{ccsm-gipm-2011}. This distance can easily be computed pointwise in the case of a point cloud (through averaging the squared distances to the $k$ nearest neighbors), but its sublevel-sets, which carry the geometric information about the measure (or the underlying shape if we consider a model where the data is generated from a measure on the shapes corrupted by some noise), remain hard to compute or approximate.

A related challenge is to find efficient algorithms to compute or approximate,  in arbitrary dimensions, the topological structure of the sublevel-sets of the distance to a measure, in the same spirit as what was done in the recent years for distances to compact sets. Such algorithms would naturally find applications in topological inference in the presence of significant noise and outliers, but also in other less obvious contexts such as stable clustering. The current bottleneck is that there exist no equivalents of the union of balls and $\alpha$-shape in the case of the distance to a measure. Our first goal will be to work out such equivalents that will allow to infer the homology of the underlying shape or more generally the topological persistence of the distance to measure functions.
%To start with, we will focus on
%medium dimensions and use a variant of the mesh-based inference algorithm [14] to approximate
%the sublevel-sets of the distance to a measure and get an idea of their topological structure.

\paragraph{Clustering with a geometric prior.} \framebox{add ref}
Clustering may be viewed as the most basic homology inference problem, since it consists in inferring the connected components in the data set. Typical methods for clustering data sets in high dimensions, {\it e.g.,} spectral clustering, work well under three specific assumptions. First, the clusters should be sufficiently connected, for example, the second eigenvalue of their graph Laplacian should be large enough. Second, they should be well separated, that is, the interpoint distances between different clusters should be large enough on average. Third, the clusters should be balanced enough. We intend to develop methods that would take advantage of situations where the clusters have additional properties, such as being nearly convex or smooth, which seems reasonable in the large number of cases where manifold learning techniques apply. A promising strategy would be to use geometric regularity measures stemming from the geometric sampling theory recently introduced in \cite{geometrica-ccl09}. One challenge is to design these regularity measures in such a way that they are both easy to compute and amenable to classical clustering strategies. We expect the resulting clustering schemes to outperform classical spectral clustering when the data exhibit some form of geometric regularity. Byproducts of this effort could also lead to efficient algorithms for assessing the degree of geometric regularity present in real data. 



\paragraph{Topological signatures for shapes.}

% Shape descriptors are used in a variety of applications, including
% shape classification, shape retrieval, shape matching, shape
% registration, and symmetry detection. % In the context of this proposal,
% the word {\em shape} must be understood in a very broad sense: for
% instance, it can be a point cloud, or a
% manifold, or more generally a compact metric space.

Using measured topological quantities to design signatures for shapes
is a relatively new idea. The bottom line of the approach is the
following: given a finite sampling of the shape, build some filtered
simplicial complex on top of the point cloud, and use the topological
structure of this filtration (encoded as a planar diagram called a
{\em persistence diagram}) as a signature for the point
cloud~\cite{ccgmo-ghsssp-09, socg-pbsds-10}. This construction is
well-suited to finite metric spaces, and the obtained signatures are
known to be stable under small perturbations of the spaces in the
Gromov-Hausdorff distance.  
A major remaining challenge is to extend the
construction to {\em infinite metric spaces} \framebox{motivation?}, and to prove the stability of
its topological structure with respect to perturbations of the space
in the Gromov-Hausdorff distance.

A more fundamental question is how much information about a shape can
be recovered from descriptors. As discussed in the state of the art,
% Most existing work in shape analysis
% only provides lower bounds on a shape distance (e.g. the
% Gromov-Hausdorff distance) based on descriptor distance. There are
% very few exceptions to this rule, and the upper bounds on shape
% distances these provide come with very loose guarantees on the
% error~\cite{bbk-gmds-06,ms-gh-05}. D
deriving tight upper bounds on shape distances based on shape descriptors is
still widely open, and we intend to tackle the problem as
follows: thanks to the virtually infinite variety of filtrations that
can be built on top of a shape, it is easy to enrich the pool of
signatures used for that shape, and thus restrict the possibility of
false positives in the shape comparison process. It is then a question
of how large a family of filtrations is required to guarantee that
different shapes do get different signatures. From an algorithmic
perspective, the problem comes down to identifying small samplings of this
family of filtrations that can be used as proxies for a better (if not
perfect) assessment of the similarity between two shapes.  As we
intend to consider our signatures on large sets of shapes, it will
also be important to design algorithms to efficiently compute these
signatures and compare them.% \framebox{shall we put a ref to
%  another WP where fast bottleneck distance computation/approximation
%  would be considered?}

Finally, the difficulty of matching two shapes is
intimately tied to matching a shape to itself --- shapes with many
natural self maps (symmetries) can be difficult to match because of
the ambiguities symmetries create (reflected in duplicate descriptors,
etc.). It may be interesting to define the analog of a {\em condition
  number} for a shape, which would capture the intrinsic difficulty of
characteristic or matching against that shape.




\subsection*{WP 4 :  Platform for geometric modelling in higher dimensions}
Development of a C++ platform that implements the best algorithms developed during the project or by external collaborators. Set up datasets for controlled experiments in support of WP 1-3, and more generally, research on Geometric Modeling in higher dimensions.  Diffuse the software and promote its use in fields like Machine Learning, Data Analysis, Numerical Simulation, Visualization, Structural Biology and others. 

A small but significant part of the project will be dedicated to gathering datasets. These data sets will be made publicly available. % It is important to compare the algorithms on datasets that are representative of realistic tasks.
They will include both synthetic datasets and datasets coming from the application domains listed above. This will be done in collaboration with expert groups with whom we are in close contact.




\section{Resources}

\paragraph{Research environment.}
The PI and his research team, Geometrica, are part of INRIA, the french national institute for research in mathematics and informatics. Part of the group, including the PI, is located in the INRIA research center in Sophia Antipolis  (500 employees and 38 research groups) and part of the group is hosted by the INRIA research center in Saclay in Paris's area (26 research teams). Two members of the research center in Sophia Antipolis are members of the French Academy of Science, two have received an ERC advanced grant and two have received an ERC junior grant. Because of its location in Saclay, the group benefits from tight collaborations with the Ecole Normale Sup\'erieure and the Ecole Polytechnique. In particular, 4 members of the group teach at these prestigious institutions. Geometrica currently includes 10 permanent researchers,  2 postdoctoral researchers, 10 Ph.D. students, and 1 research engineer. 

\paragraph{The team members.}
The team members who are directly involved in this proposal are the PI (J-D. Boissonnat) and 2 permanent researchers of the Geometrica team : Fr\'ed\'eric Chazal and Mariette Yvinec.  J-D. Boissonnat will conduct and supervise the research activities of Gudhi and will be involved in the project for at least 70\% of his time.  Fr\'ed\'eric Chazal and Mariette Yvinec will each devote 20\% of their time to this project to co-supervise with the PI the research and implementation work of the students, postdocs and engineers to be engaged in this project. Fr\'ed\'eric Chazal, located in Saclay,  is a world expert in geometric inference and computational topology. Mariette Yvinec, located in Sophia Antipolis,  is a member of the CGAL Editorial Board. She  will be bring her unique expertise in geometric computing. Other members of Geometrica, not financially supported by this project, will also contribute to the ideas and expertise of Gudhi.

\paragraph{External team members.} ??

\paragraph{Available resources.} Our European ICT Fet-Open project Computational Geometric Learning (CG-Learning) will still be active until november 2013 for a maximum remaining amount of ??? Euros and will ensure a smooth start of Gudhi.  The ANR Pr\'esage (??? Keuros) will provide additional resources for the Geometrica activities in probabilistic techniques in geometry.

The Geometrica team is equipped with numerous PCs and has access to a large PC cluster owned by INRIA Sophia Antipolis.

\paragraph{Requested resources: personnal costs.}\mbox{}\\
-- 70\% of PI's salary over 5 years with a 70\% commitment of his time\\
-- 20\% of 2 PI's close collaborators over 5 years with a 20\% commitment of their time\\
-- 2 full-time post-doctoral researchers during the 5 years of the project\\
-- 2 full-time research engineers during 2.5 years covering the 5 years of the project\\
-- 1 full-time Ph.D. student during years 1, 2, 3\\
-- 1 full-time Ph.D. student during years 2, 3, 4\\
-- 1 full-time Ph.D. student during years 3, 4, 5\\
-- 4 men-months of invited professors in years 1-5.

\paragraph{Requested resources: other direct costs.}\mbox{}\\
-- travel \\
-- hardware : a multicore computer 5 KEuros

The rest of the costs consists of eligible indirect costs, at the rate of 20\% of the direct costs. \framebox{?}
The grand total amounts to ???? Euros over a period of 5 years, as detailed in the table below.

\framebox{include Table}

The funded 4 PhD students will have their research devoted to the fundamental aspects of FA1, FA2 and FA3 of this proposal, and one on the applications. There will be a lot of synergy between their works, in particular in relation with the development of the platform. The funded research engineers  will help stabilize the software modules, as well as for the construction of new datasets to be made available to the scientific community.

We expect several researchers among our current partners (in particular in the CG-Learning project) to visit us each year and participate to Gudhi. We will also welcome talents from new groups who could bring a complementary expertise to the success of Gudhi. These visits will be funded in part by the ”invited professors” budget above, and in part by INRIA and other resources.
We will also organize a workshop early after the beginning of Gudhi to make it known to the international community, and to help attracting talented scientists for the success of Gudhi.


%\bibliographystyle{alpha}
\bibliographystyle{apalike}
\bibliography{erc}


