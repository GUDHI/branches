\documentclass[11pt,psfig]{article}

\usepackage{latexsym}
\usepackage{url}
% \usepackage{hyperref}
\usepackage[T1]{fontenc}
\usepackage{graphicx}
% \usepackage[francais]{babel}
\usepackage[latin1]{inputenc}
\usepackage{times}

\textwidth=6in
\textheight=8.5in
\topmargin=0.0in
\oddsidemargin=0.25in
\evensidemargin=0in

\newtheorem{Definition}{Definition}[section]
\newtheorem{PseudoDefinition}[Definition]{Pseudo-Definition}
\newtheorem{Proposition}[Definition]{Proposition}
\newtheorem{Corollary}[Definition]{Corollary}
\newtheorem{Theorem}[Definition]{Theorem}
\newtheorem{Example}[Definition]{Example}
\newtheorem{Lemma}[Definition]{Lemma}
\newtheorem{Notation}[Definition]{Notation}



\begin{document}
%\input{psfig}
%environment.tex

\newcommand{\comment}[1]{

COMMENTS \\ #1 END-COMMENTS

}
% \newcommand{\comment}[1]{}

%This is used to put some text into personal comments.  
%	\comment{blabla ... blabla}
%
%For semi-final versions, 
% 1) un-% the last line, and
% 2) % all other lines


%stuff for lemmas, thms, defs, etc.  for THE BOOK

% in the following bactch of definitions two things happen.
% first, temporaryxxx are environments which start with bold Lemma or Theorem,
%	etc., and use a shared counter.
% second, the environments lemma, thm, etc. are just the same, except
%	unlike raw Latex, the text of the environment is roman instead of
%	italic

%the guys above share one counter, and start over with each section

%note that deff and nott (for `notation') are spelled strangely

% cf. p. 174 of manual

\oddsidemargin=0in
\evensidemargin=0in

\newtheorem{temporarylemma}{Lemma}[section]
\newtheorem{temporarythm}[temporarylemma]{Theorem}
\newtheorem{temporaryprop}[temporarylemma]{Proposition}
\newtheorem{temporaryconj}[temporarylemma]{Conjecture}
\newtheorem{temporarycor}[temporarylemma]{Corollary}
\newtheorem{temporaryalg}[temporarylemma]{Algorithm}
\newtheorem{temporaryproc}[temporarylemma]{Procedure}
\newtheorem{temporaryremark}[temporarylemma]{Remark}
\newtheorem{temporaryexamp}[temporarylemma]{Example}


\newenvironment{lemma}{\begin{temporarylemma} 
	\rm}{\end{temporarylemma}}
\newenvironment{thm}{\begin{temporarythm} 
	\rm}{\end{temporarythm}}
\newenvironment{prop}{\begin{temporaryprop} 
	\rm}{\end{temporaryprop}}
\newenvironment{conj}{\begin{temporaryconj} 
	\rm}{\end{temporaryconj}}
\newenvironment{cor}{\begin{temporarycor} 
	\rm}{\end{temporarycor}}
\newenvironment{alg}{\begin{temporaryalg} 
	\parbox{4in}{  } \rm}{$\Box$ \end{temporaryalg}}
\newenvironment{proc}{\begin{temporaryproc} 
	\rm}{$\Box$ \end{temporaryproc}}
\newenvironment{remark}{\begin{temporaryremark} 
	\rm}{$\Box$ \end{temporaryremark}}


% for examples, we place a horizontal bar above
% and below the example, using the new \sep command
% Unfortunately, one of the horizontal lines might
% be orphaned -- either at the bottom of a page or the top of a page.
% (Various attempts by Rick to remedy this problem failed...)

\newcommand{\sep}{\rule{12.7cm}{0.1mm}}
\newenvironment{examp}{\noindent \sep \begin{temporaryexamp} 
		\rm}{\newline \sep \end{temporaryexamp}}

% the following is for homework exercises; they
% are numbered within chapter, not section

% \newtheorem{temporaryexer}{Exercise}[chapter]
%\newtheorem{tempclubexer}[temporaryexer]{$\clubsuit$Exercise}
%\newtheorem{tempdiamondexer}[temporaryexer]{$\diamondsuit$Exercise}
%\newtheorem{tempheartexer}[temporaryexer]{$\heartsuit$Exercise}
%\newtheorem{tempspadeexer}[temporaryexer]{$\spadesuit$Exercise}
%\newenvironment{exer}{\begin{temporaryexer} \rm}{\end{temporaryexer}}
%\newenvironment{clubexer}{\begin{tempclubexer} \rm}{\end{tempclubexer}}
%\newenvironment{diamondexer}{\begin{tempdiamondexer} \rm}{\end{tempdiamondexer}}
%\newenvironment{heartexer}{\begin{tempheartexer} \rm}{\end{tempheartexer}}
%\newenvironment{spadeexer}{\begin{tempspadeexer} \rm}{\end{tempspadeexer}}

\newenvironment{deff}{\medskip \noindent \bf Definition: \rm}{\medskip}
\newenvironment{nott}{\medskip \noindent \bf Notation: \rm}{\medskip}
\newenvironment{fact}{\medskip \noindent \bf Fact: \rm}{\medskip}
\newenvironment{claim}{\medskip \noindent \bf Claim: \rm}{\medskip}

\newenvironment{proof}{\medskip \noindent \bf Proof: \rm}{$\Box$ \medskip}
\newenvironment{proofnobox}{\medskip \noindent \bf Proof: \rm}{\medskip}
\newenvironment{crux}{\medskip \noindent \bf Crux: \rm}{$\Box$ \medskip}
\newenvironment{cruxnobox}{\medskip \noindent \bf Crux: \rm}{\medskip}



% another useful place in manual is p. 57 and p. 173, 

% --------------------------------------

% some symbols we use

\newcommand{\mvd}{\mbox{$\ \rightarrow \hspace*{-.5cm} \rightarrow \ $}}
% for ``multivalued dependencies''

\newcommand{\jd}{\Join\!\!}
% for ``join dependencies'' -- gets the spacing right --
% however, use: $\models \, \jd[X,Y,Z]$ ...

% --------------------------------------

\newcommand{\reminder}[1]{ [[[ \marginpar{\mbox{$<==$}} #1 ]]] }






\newenvironment{Def} {\begin{Definition} \rm} {\end{Definition}} 
\newenvironment{The} {\begin{Theorem} \rm} {\end{Theorem}} 
\newenvironment{Exa} {\begin{Example} \rm} {\end{Example}} 
\newenvironment{Pro} {\begin{Proposition} \rm} {\end{Proposition}} 
\newenvironment{Cor} {\begin{Corollary} \rm} {\end{Corollary}} 
\newenvironment{Lem} {\begin{Lemma} \rm} {\end{Lemma}} 
\newenvironment{Not} {\begin{Notation} \rm} {\end{Notation}} 

\newcommand{\pp}{\vspace{6 mm}}

\newcommand{\begindef}{\begin{Definition} \rm}
\newcommand{\ben}{\begin{enumerate}}
\newcommand{\een}{\end{enumerate}}
\newcommand{\bit}{\begin{itemize}}
\newcommand{\eit}{\end{itemize}}
\newcommand{\bve}{\begin{verbatim}}
\newcommand{\eve}{\end{verbatim}}
\newcommand{\btab}{\begin{tabbing}}
\newcommand{\etab}{\end{tabbing}}

\newcommand{\sect}{\section}
\newcommand{\ssect}{\subsection}
\newcommand{\sssect}{\subsubsection}

\newcommand{\bigR}{$\cal R$}
\newcommand{\bigS}{$\cal S$}
\newcommand{\bigT}{$\cal T$}
% \newcommand{\SS}{\cal S}

\newcommand{\So}{\setlength{\hspace{-0.2mm}\unitlength}{1mm}\begin{picture}(3.5,5)(0,1.3)\put(1,0){\line(0,1){5}}\put(1.6,0){\line(0,1){5}}\put(1,0){\line(1,0){1.5}}\put(1,5){\line(1,0){1.5}}\end{picture}\hspace{-1mm}}
\newcommand{\Sc}{\setlength{\unitlength}{1mm}\hspace{-2.5mm}\begin{picture}(3.5,5)(0,1.3)\put(3.5,0){\line(0,1){5}}\put(2.9,0){\line(0,1){5}}\put(3.5,0){\line(-1,0){1.5}}\put(3.5,5){\line(-1,0){1.5}}\end{picture}\hspace{0.5mm}}
\newcommand{\EMo}{{\Large\bf\protect[}}
\newcommand{\EMc}{{\Large\bf\protect]}}
\newcommand{\sem}[2]{\So #1\Sc$\!\,_{\mbox{\small #2}}$}

% \newcommand{\mvd}{\mbox{$\ \rightarrow \hspace*{-.5cm} \rightarrow \ $}}

\newcommand{\esp}{\vspace*{1cm}}
\setlength{\parindent}{0.7cm}
\setlength{\parskip}{0.3cm}

\vspace*{1cm}


\begin{center}
\Huge{Webdam} \\
\vspace*{1cm}
\Large{Foundations of Web Data Management}\\
\vspace*{1cm}
\Large{Technical Description}\\
\vspace*{1cm}
\Large{ERC Advanced Grant} \\
\Large{7th Framework Programme} 

\end{center}

\vspace*{1cm}
Principal Investigator: Serge Abiteboul,
\url{http://www-rocq.inria.fr/~abitebou} 

Host institution: INRIA Saclay--�le-de-France

Date of this document: 2008

\newcommand{\victor}[1]{ }
\newcommand{\serge}[1]{ }

%% \newcommand{\victor}{{\bf FIX }}
%% \newcommand{\serge}{{\bf Serge }}

% {mh.pautrat@orange.fr}

% \renewcommand{\thesection}{}
% \renewcommand{\thesubsection}{}
% \renewcommand{\thesubsubsection}{}

\newpage

\pagestyle{myheadings}
\markboth
{WebDam -- Technical Description -- 2008}
{WebDam -- Technical Description -- 2008}


% {\begin{center}
\fbox{\parbox{12cm}{\noindent {\bf Abstract: } We propose to develop a
formal model for Web data management. This model will open new
horizons for the development of the Web in a well-principled way,
enhancing its functionality, performance, and
reliability. Specifically, the goal is to develop a universally
accepted formal framework for describing complex and flexible
interacting Web applications featuring notably data exchange, sharing,
integration, querying and updating. We also propose to develop formal
foundations that will enable peers to concurrently reason about global
data management activities, cooperate in solving specific tasks and
support services with desired quality of service.  Although the
proposal addresses fundamental issues, its goal is to serve as the
basis for ground-breaking future software development for Web data
management.  }}\end{center}

% UPDATE A1

% Fondements de la gestion de donn�es du Web

% L'objectif du projet Webdam est de d�velopper un cadre formel
% universellement accept� pour d�crire des applications Web mettant en
% jeu des interactions complexes et flexibles, avec notamment de
% l'�change, du partage, de l'int�gration, de l'interrogation et des
% mises � jour de donn�es. Nous proposons �galement de d�velopper les
% fondements formels qui permettront � des syst�mes autonomes de
% raisonner ensemble sur une gestion globale de donn�es, de coop�rer
% pour r�soudre des t�ches sp�cifiques et procurer des services avec les
% qualit�s de service souhait�es. Bien que ce projet aborde des
% questions fondamentales, son objectif est aussi de servir de base pour
% des d�veloppemeents futurs de logiciels pour la gestion des donn�es du
% Web, faciliter le d�veloppement du Web, am�liorer ses fonctionnalit�s,
% ses performances et sa fiabilit�.}

% \newpage

\section{Extended synopsis of the project proposal}

\begin{center}
\fbox{\parbox{12cm}{\noindent {\bf Abstract: } We propose to develop a
formal model for Web data management. This model will open new
horizons for the development of the Web in a well-principled way,
enhancing its functionality, performance, and
reliability. Specifically, the goal is to develop a universally
accepted formal framework for describing complex and flexible
interacting Web applications featuring notably data exchange, sharing,
integration, querying and updating. We also propose to develop formal
foundations that will enable peers to concurrently reason about global
data management activities, cooperate in solving specific tasks and
support services with desired quality of service.  Although the
proposal addresses fundamental issues, its goal is to serve as the
basis for ground-breaking future software development for Web data
management.  }}\end{center}

% UPDATE A1

% Fondements de la gestion de donn�es du Web

% L'objectif du projet Webdam est de d�velopper un cadre formel
% universellement accept� pour d�crire des applications Web mettant en
% jeu des interactions complexes et flexibles, avec notamment de
% l'�change, du partage, de l'int�gration, de l'interrogation et des
% mises � jour de donn�es. Nous proposons �galement de d�velopper les
% fondements formels qui permettront � des syst�mes autonomes de
% raisonner ensemble sur une gestion globale de donn�es, de coop�rer
% pour r�soudre des t�ches sp�cifiques et procurer des services avec les
% qualit�s de service souhait�es. Bien que ce projet aborde des
% questions fondamentales, son objectif est aussi de servir de base pour
% des d�veloppemeents futurs de logiciels pour la gestion des donn�es du
% Web, faciliter le d�veloppement du Web, am�liorer ses fonctionnalit�s,
% ses performances et sa fiabilit�.

One can see in computers and the network, a revolution comparable to
that of writing. Writing freed us from the need to memorize
information \cite{Serres07}. The new technology is freeing us (to some
extent) from the need to reason about information and will enable us
to focus our energy to imagination and creativity. The Web with data
deployed on millions of machines flourishes at the core of this new
revolution. 

The Web was first seen as a universal access to an ocean of documents.
It is progressing slowly (by Web-time measure) towards a world-wide
repository of knowledge, i.e., towards the Semantic Web
\cite{TBL01}. We are also observing the birth of Web 2.0
\cite{OReilly05}, that stresses social networks and community
building. Then came proposals for Web 3.0 up to Web Googol.0.  These
initiatives, totally lacking scientific grounding, have ``invented''
concepts already studied in research labs and sometimes even present
in software products.  However, in spite of their obvious weaknesses,
these proposals highlight the fact that the Web is meant to support
functionalities that are beyond a simplistic query/answer paradigm.
Indeed, with the Web as the basis of human sharing of information and
human interaction,
\begin{quotation}
\noindent
{\em Web data management is becoming the cornerstone of human
activities.} 
\end{quotation}

In spite of its importance, the management of data on the Web suffers
from a number of weaknesses that can be blamed to its recent birth and
% BUZZ
its too rapid growth. The goal of the proposal is to develop a
mathematical model for Web data management and formal foundations to
reason about Web applications. Indeed, we propose to develop the
analog of the formal model that has successfully underlied relational
database systems. Building on it, we propose to develop the necessary
reasoning capabilities for controlling data exchange, sharing,
integration, querying and updating, in a Web environment.
    
\paragraph{Time to stop hacking the Web }

For data management, the turn of the century has seen the maturing of
the relational database industry (with notably Oracle, IBM DB2 and MS
SQL Server) and of the corresponding scientific field, at the border
of logic and complexity theory. Since the 60's, industry and academic
research have progressed in tandem, with strong interactions that have
been instrumental in their respective successes. Today, with the
preeminence of the Web, new challenges are facing the data management
arena. New ideas are emerging and transferred into products very
rapidly. The scientific community has difficulties following the pace
and to a large extend, it is lagging behind. We claim that the quick
and dirty solutions, often used today, cannot provide long-term
solutions and do not scale to the Web of the future. We argue that,
perhaps even more than for relational systems, academia has an
essential role to play in shaping the new field. Indeed, the project
we propose tackles key technological issues that present the severe
risk of slowing down Web data management. Solutions even designed by
fantastic hackers are bound to fail due to the complexity of the
problem and its scale. We therefore propose to develop a formal model
that is needed for developing efficient, powerful and reliable
software for Web data management.
   
\paragraph{Limiting the scope: semantics}
A main issue for the Web is semantics. For instance, it is simply not
possible to use some newly discovered resource without understanding
first its semantics. The study of Web semantics leads to fascinating
challenges of an AI nature such as knowledge representation, knowledge
extraction from text, or reasoning with knowledge. Such issues will
not be central here. We are focusing instead on the data layer that
will help support semantic layers. To illustrate the distinction, for
instance, we are not concerned here with how a system analyses a Web
service to understand how to use it (an AI problem) but rather, with
how it will actually take the best advantage of the service to obtain
data (a DB problem). Of course, the boundary between the two domains
is fuzzy and the author of the proposal has worked on a number of
projects at their frontier. However, the techniques they require are
rather different, which justifies limiting the scope of the proposal
to the data management side. Clearly, semantics will still be very
present in the sense that the data management layer should be powerful
enough to satisfy the needs of semantic layers. So, for instance, the
formal model will include the means to describe static
properties of the data (e.g., with tree automata) as well as
behavioral properties (e.g., with temporal logic formulas).

\paragraph{A unifying model} 
An underlying philosophy of the proposal is that we need an unified
formal model for noncentralized data management.  So, we are concerned
here with all distributed applications that {\em possibly} manage huge
volumes of data with {\em possibly} large number of autonomous and
heterogeneous computers, typically without centralized authority.  We
believe that the foundational work that is needed should be applicable
to such a wide range of applications. The prime motivation for using a
single model across all Web applications is that it will enable all
kinds of Web applications to exchange data. Another positive aspect of
genericity, is that it will force us to stay away from being too
directly influenced by particular technological biases, and focus on
rich mathematical notions.  The Webdam proposal is thus extremely
ambitious because it targets the world-wide acceptance of a unique
model for data management on the Web.

Note that we did not use the term ``distributed data management''
because for many it comes with limitations to small numbers of data
servers, often with a centralized authority.  We were also reluctant
to use the term ``peer-to-peer'' because for many its meaning is
restricted to very large numbers of machines and high churn rates. In
Webdam, we address distributed and peer-to-peer data management
scenario, even if each such context may come equipped with its specific
optimization strategies and its specific issues. We prefer the term
``Web data management'' to stress that the Web is the testbed for this
new technology. Observe that we are neither excluding smaller scale
devices (such as RFIDs) nor smaller scale networks (such as home
networks).

Since distributed databases have been around for more than twenty
years, one may wonder at this point whether some such foundations
already exist.  It turns out that this is not the case. While he
taught distributed databases at Stanford, the principal investigator
realized how informal, imprecise and incomplete was the material on
the topic.  For teaching relational databases, one can rely on a clean
and solid formal model \cite{AHV,Ullman}. But when distribution is
considered, the rare existing books, e.g. \cite{OV99}, propose ad hoc
extensions, developed for the purpose of explaining particular aspects
and specific techniques.  A formal model is missing and to develop one
is a main goal of Webdam. Indeed, as a by-product, Webdam should
bring improvements to course materials for distributed databases, and
provide the basis for new courses on Web data management.

The first goal of the proposal is thus to develop a mathematical model
for Web data management. We next discuss the second goal, that is, to
develop formal foundations to reason about Web applications.

\paragraph{Automatic reasoning}

In many Web applications, notably industrial applications, we are
interested in achieving some desired quality of service level. For
instance, we may want to allow several users working on the same data
collection without disrupting each other (concurrency control) or to
support recovery from peer failures. Other examples of a less
classical database nature involve guaranteeing full awareness of
changes occurring in sites of interest (monitoring) or checking that
data only come from trusted sites (provenance). The current style of
development of Web applications makes it infeasible to obtain such
guarantees. Techniques developed for the relational model are not
directly applicable. For instance, concurrently control techniques are
typically assuming a central authority which rarely exists in a Web
context. Two aspects are making the problem even more challenging: (i)
the application design must be extremely flexible to adapt to the
intrinsic anarchist style of the Web and (ii) there is typically no
database administrator in Web applications and often not even any
computer specialist to design and maintain the application.  For
instance, most cooperative Web sites (e.g., based on Wordpress or
similar systems) integrating data from many sources (e.g., RSS feeds),
are developed by non computer experts.  Because of (i), we cannot
impose any a-priori design such as ACID for relational
transactions. And, because of (ii), we have to favor approaches that
do not require actually writing any complex code, with application
developers specifying declaratively requirements, and the code being
generated from them. As a consequence of the two, to still be
manageable, the system has to be able to reason about an application,
analyze the current or future run and possibly adapt based on the
results of the analysis. Automatic reasoning (taken here in a very
large sense) is unavoidable in this setting to support all activities
from querying and optimization, to updates, tuning, recovery,
monitoring, etc. We will precise further on, the kind of reasoning
that we mean.

\paragraph{High-risk, high gain}

What are the challenges and chances for achieving such an ambitious
goal with Webdam?

\paragraph{Not enough theoretical}
There is a risk to stay too close to the current Web technology.  If
we are too influenced by technological details of a non fundamental
nature, we will obtain a too complex model, biased by the current
technology and its limitations. As a consequence, it will be
impossible to lay the appropriate formal foundations.  In recent
applied works, the principal investigator could observe the limits of
ad hoc approaches and the crucial need for solid foundations for Web
data management. He is thus much aware of this pitfall. His experience
and that of a number of talented theoreticians who will be associated
to the endeavor, should allow avoiding it.

\paragraph{Too theoretical}
Of course, there is the somewhat opposite risk of developing beautiful
theoretical techniques with little impact outside of academic circles.
The record of the principal investigator is a strong indication that
this will not be the case. Indeed, he has for the last ten years
always been involved with works with transfer to industry. In
particular, his work around semistructured data has had important
impact notably on standards for XML query languages. Foundational work
here is not a goal for its own sake (which would already have been a
fair motivation) but is meant as a sound basis for future software
development.

\paragraph{All or nothing?}
% We are convinced that we can develop in the time frame desired
% foundations that should serve as the basis for needed new standards.
% Our experience in data management and more specifically on Web data
% management and the results we have already obtained in that field
% substantiate our claim.  
With top quality researchers and the ambition to succeed, Webdam will
serve as the catalyst for excellent European research on the topic.
Even if Webdam does not reach its full goal of providing a
comprehensive, universally accepted framework for Web data management,
it should provide substantial progress towards this goal.

\paragraph{An opportunity for Europe}
Does Europe have a chance to succeed in this strategic field that will
surely be very competitive?  In the early 80ths (when the PI finished
his PhD), Europe was almost absent from data management research. For
data management systems, Europe has regularly closed the gap with the
US and now hosts international quality groups in many countries, one
of them being the Gemo team.  Database theory developed in Europe
starting from a few pioneers, e.g., Paredaens from Belgium and the PI
in France.  One can now say that the center of gravity has shifted
from the US to Europe in this area.  ACM PODS has had 4 European
Program Chairs since 2000 and the past {\em 7} winners of the Best
Paper Awards had one European co-author.  Success can be achieved only
by driving the development of sophisticated mathematical models with
concrete technological goals.  With a strong presence in both database
theory and data management systems, Webdam and more generally Europe
are ideally placed to carry out such a program.

We will see in the next section how this can be achieved using a
methodology based on the following principles:
\begin{enumerate}

\item developing mathematical foundations combining techniques from
databases and verification,

\item building on the results and experience already achieved, 

\item bringing together local talents and researchers from an already
existing network of collaborators, that will be extended during the
project,

\item relying on simplicity and carefully chosen limitations in
the model of computation we use, 

\item getting inspiration from real Web applications and validating
results with prototypes.
\end{enumerate}

To conclude this section, we want to stress that:
\begin{quotation}
\noindent
{\em for obvious economical, social and political reasons, the
management of the data of the Web is of strategic importance. }
\end{quotation}We therefore strongly believe
that it is essential to fund ambitious projects in this area.

We next discuss the state of the art and the objectives, considering
in turns the two main facets, namely formal model and reasoning. We
then present the main traits of the proposal, briefly sketching the
expected scientific contributions and the milestones. More technical
details on some of these aspects may be found in a CIDR vision paper,
written with Neoklis Polyzotis \cite{AP}. Finally we discuss
resources.

\section{State-of-the-art and objectives}

\subsection{A first scientific shift in data management: trees,
functions}

With the Web, we have shifted from the management of very structured
data (i.e. relations) in centralized databases to semistructured data
(from text to trees and graphs) in autonomous distributed
systems. Even if most Web data still comes from relational databases,
they are typically viewed on the Web as trees, XML or HTML.  A main
aspect of the information is its lack of regularity coming from the
heterogeneity of the Web and from the multiplicity of sources and
authors. It is also most importantly distributed, and dynamic. Another
essential aspect of modern data management is its scale in terms both
of quantity of data (e.g., see the billions of pages indexed by
Google) and of number of peers (e.g. millions of peers for P2P music
sharing).  In Webdam, we will build on the standards of the Web,
namely XML \cite{xml} and Web services
\cite{ws}. These choices are somewhat not negotiable on the Web. They
happen to also be scientifically sound. Indeed, these standards may be
abstracted in a clean and elegant manner using trees and functions.

\paragraph{XML}
XML, the Web standard for data exchange, is based on unranked,
labeled, ordered trees. Depending on the needs, we will sometimes see
the trees as unordered to bring the model closer to standard logics
that are set-oriented as well as to lower the complexity. Labeled
trees are much more appropriate to the Web than relational structures
that are too rigid and constraining. A most natural tool for such
trees is tree automata \cite{tata} that have strong connections with
monadic second-order logic. Standard query languages for XML, XPATH
and XQuery may be seen (with some stretch of imagination) as logics
for such trees.  Theoretical aspects of XML have been actively
investigated during the last few years, with tree automata and monadic
second-order logic playing central roles. The results that were
obtained are important for us and they will serve as basis for our
investigation. They miss the target of being a formal model for Web
data management - our primary concern - for two main reasons. First,
they do not directly address distribution.  Second, they are
tree-based and not graph oriented, as more essential for the
Web. Finally, they address static issues and miss the dynamic nature
and interactivity of Web needs. For instance, tree automata are a
fantastic tool for static centralized trees, whereas we have to manage
evolving distributed graphs.

\paragraph{The playground: the Web and Web services}
One of the key features of the Web are Web services, that is the
possibility to activate a computation on a distant site and obtain
data from it or more generally interact with that data (e.g. apply
updates). In some sense, HTML, the hypergraph structure of the Web, and
Web search engines, may be seen as very limited forms of read-only
services and Web publication as primitive write services. Web services
are now giving birth to a very active area of software
development. Data exchanges between peers are captured by calls to
complex Web services taking into account the {\em logic} of these data
exchanges typically specified intentionally using logical
formulas. Like XML coming with a family of ``standards'' (XML schema,
XQuery, Xpointer, etc.), Web services come with a family of
``standards'', in particular, WSDL for specifying the signatures of
services, BPEL for specifying their sequencing, UDDI for
publishing/discovering services. In this proposal, the focus is on
services for managing information, e.g., query/update services or data
monitoring, that is on data intensive services rather than on services
performing intensive computations as in scientific grids.  More
precisely, we see a Web service here as a function that just
happens to be supported elsewhere, with some side effect there and
eventually returning a piece or a flow of data.

We have already worked for several years on a formal model based on
XML and Web services that will serve as a first step towards the Webdam
model.

\paragraph{ActiveXML}

ActiveXML \cite{axml} is simply XML with embedded Web services, i.e.,
with intentional information. So the fundamental idea is to place views,
i.e. possibly external information, at the center of the picture.  The
notion of documents with function calls is in the spirit of
object-oriented languages.  The combination of trees and functions
opens an array of new questions that are central to Web data
management, some we started addressing, for instance:
\begin{enumerate}
\item Querying intentional data: evaluating queries when part of the
data is elsewhere.
\item Incremental maintenance: maintaining a materialized view when
some functions bring in continuously data.
\item Casting: choosing which functions to call to force such a tree
to match a desired type.
\item Distribution: distributing a tree between several peers in the
style of Ldap to distribute both the information and the processing
load.
\end{enumerate}
Two essential lessons we have learned from the ActiveXML project, are
the following: (i) function calls should be asynchronous and (ii) data
exchange should be based on data streams. But most importantly we have
acquired experience in developing Web applications and on the various
facets of the problem.

The ActiveXML model is more complex than the relational model because
it encompasses in a nutshell most of the various themes of 20 years of
database research, most notably: deductive databases \cite{AHV},
object databases \cite{odmg} and active databases
\cite{WidomCeri96}. This together with the simplicity of the
definition and experiments with real Web applications are, we believe,
indications that we are on the right track. Another indication is that
independent proposals seem to reach similar shores from different
angles. For instance, starting from XQuery, accesses to several
sources are considered in \cite{DXQ}.  Although the approach may seem
very different, similar issues are encountered.

In its current state, the ActiveXML model is too basic, say like the
core relational model of the early days, before dependency and
concurrency control theory matured. As previously mentioned, the
setting is much more complex so more work is needed to achieve the
goal of a general model for Web data management with the proper
accompanying theory. Building on the results already obtained with
ActiveXML, with the experience in designing applications with that
language, we are now ready for a crucial part of the work in Webdam:
developing a simple and solid mathematical model for distributed data
management and building the accompanying theory to obtain the desired
formal foundations.  We will sketch in the next section some
directions to improve the model and questions to solve about this
model.

\subsection{A second shift in data management: deduction}

Let us reconsider the origin of the success of the relational model. A
relational database system is essentially a First-Order Logic machine
placed on everyone's desk to manage data.  More precisely, a
non-specialist can specify some needs, declaratively, in first-order
logic terms. The system then compiles such a ``logical query'' into an
``algebraic query plan'' that is optimized then evaluated. Relational
systems therefore perform automatic reasoning to handle queries and
views, e.g., to rewrite queries into equivalent ones to optimize
them. Reasoning is also present in many other aspects of relational
systems, e.g., dependencies (logical formulas over the data that the
system should enforce), transactions and concurrency control, triggers
(i.e. active rules).  In relational systems, such reasoning is in some
sense ``hard-wired''.  For instance, several transactions are allowed
to access/update simultaneously the same database. The system is in
charge of verifying that they do not interact improperly. To do that,
the system assumes some laws governing the interactions and implements
an algorithm (some reasoning) to check that these laws are not
violated.  The laws are decided in advance (e.g., ACID properties) and
the reasoning is encoded in algorithms, the correctness of which has
been proved in advance (e.g., 2phase-locking).

Similarly, we want an intelligent interface between a human being and
the network that nowadays stores the data we use.  On the Web,
logic is needed for the declarative specification of the application,
but it is also needed to describe the laws governing applications
(the equivalent to the ``hard-wired'' logic of relational systems)
since much more freedom is desired and Web data management cannot live
with inflexible preconceived laws.  The distribution also brings
fundamental differences.  The reasoning is now performed by different,
typically heterogeneous and autonomous systems.  This entails that
reasoning must become a first class citizen in the sense that one peer
may have to delegate some reasoning task to another one and that they
may have to exchange (partial) proofs.  Also, reasoning will typically
be much less hard-wired than in relational systems, since we need to
support complex interactions between peers governed by
application-dependent laws that are not known in advance. As a
consequence, the system must rely on more sophisticated reasoning for
instance to optimize queries in a distributed setting (between
autonomous peers) or perform change control in a very dynamic
distributed environment.

Deduction and reasoning have been around in relational databases,
e.g. with deductive databases. The novel shift is that deduction is
now used within a much wider spectrum of functionalities.

Reasoning about applications is also particularly essential when one
wants to compose Web services to support some complex task
(orchestration) or when several services cooperate towards a
particular goal (choreography).  One may want to verify that a
particular contract is enforced, some quality of service guaranteed.
Current Web service technology is way too limited in terms of
functionalities. Tools for verifying properties of Web services, e.g.
their composability, are too primitive. These issues are not
particular to data management, but as we will see, the fact that the
focus is on data management, does bring fundamental differences.

We briefly consider next relevant existing technologies.

Web knowledge representation languages such as RDF \cite{rdf}, SPARQL
\cite{sparql} or Owl \cite{owl} are essential components of the Web,
in particular, for data integration.  They adress issues that somewhat
complement the core issues we consider.  They often ignore
distribution and when they provide rich means for describing changes
and dynamic behavior, little is known on the automatic verification of
dynamic properties.  Since the developments of knowledge
representation models and systems are important for the semantic Web,
we will have to consider their interaction with the techniques for
data management we envision.  However, research on Web knowledge
representation is not at the core of Webdam.

In databases, reasoning about data and queries has a long history. In
particular, the beautiful theory of dependencies has emerged and query
equivalence and optimization have been studied extensively.  Most
results were obtained for the relational model even if recent
developments have been concerned with tree data and XML.  Results in
this areas are relevant, e.g. the chase, a general technique used in
many contexts for inferring data properties. Unfortunately, most of the
results deal with query processing (most of the time for monotone
queries) and ignore data changes, an essential aspect in our setting.

Information retrieval and search technology are also relevant for the
Web, because of the number of available resources and the difficulty
to choose between them.  For that aspect, we will rely on results
obtained by the ERC Advanced Grant Proposal SeCo on Search Computing.
We are considering collaborations between the two projects. 

For reasoning about Web applications, three other computer science
areas are very relevant: model checking \cite{mc}, automata
\cite{tata} and temporal logic \cite{tl}.  These areas are now very
mature with impressive achievements. In particular, the automatic
verification of temporal properties of (even distributed) programs has
made enormous progress.  Unfortunately, works in these directions
rarely take data into consideration.  Automatic program analysis is
typically limited to finite state systems whereas the presence of data
immediately leads to infinite states.  Researchers in these areas are
aware of the limitations and have considered recently incorporating
data.  For instance, there has been some recent work on integrating
some aspects of data into automata, logics, and model checking, see,
e.g., \cite{Neven&Schwentick&Vianu04,JL07,BMSSD06,B+07}.  One can in
general consider extensions of model checking techniques to infinite
state systems. In particular, symbolic analysis of infinite systems
based on rewriting techniques analyzed with automata techniques have
already proved successful in a number of areas: communication systems,
parameterized verification, program analysis, timed systems, security
protocols, XML and Web services.  This is the topic of a Dagstuhl
seminar \cite{dagstuhl} and we intend to follow this line of work.
There are also interesting recent works on the verification of Web
services with data (so infinite state), e.g., \cite{hull+05,Vianu+06}.
We will study more systematically verification techniques adapted to
our context that includes data.

Another technology is very relevant, notably because it has data at
its core, namely, datalog and deductive databases \cite{ddb,AHV}.
Observe that, because of distribution, the specification of data on
the Web, so issues such as query evaluation, are recursive by nature:
Data in Peer 1 refers to data in Peer 2, that refers to data in Peer
1. So, we need a logic that handles recursion and datalog is one that
scales to large volumes of data.

\paragraph{Datalog }
Datalog has been very popular in the 80's. In essence, datalog offers
the possibility to define recursive views, i.e., associate names to
recursive query statements. Unfortunately, datalog found no real
application that the simple transitive closure of relations from SQL
could not solve. Thus, even if sometimes silently present in
relational systems via recursive CREATE VIEW statements, datalog has
not been a success. The situation is very different now because of
distribution.  Indeed, we believe that the Web is the killer
application for datalog. To motivate this claim, we next mention three
recent works all relevant here that have in common their reliance on
datalog:
\begin{itemize}
\item Trees: In \cite{lixto}, Gottlob et al specify the extraction of
data from trees in datalog. Here what is useful is the notion of
patterns in datalog and in particular that of tree patterns.

\item Graph: In \cite{H+}, Hellerstein et al compute rooting tables
over the Internet with datalog. Here, what is useful is the recursion
in datalog that permits navigating in a graph by traversing new peers.

\item Trees and graphs: In \cite{AA05}, Abiteboul et al use datalog
to study the diagnosis problem in a network of independent
components. The ``unfolding'' of a computation (in the distributed
Petri net sense) is captured as a tree that is distributed between
different peers.
\end{itemize}

We will study datalog and extensions (in particular, with negation) in
the Web setting.  We will use the language (and extensions) both for
describing and for reasoning about Web computations.

\section{Methodology}

The methodology we adopt is based on a number of
principles that we develop in this section. We also mention problems
that we will address. Clearly, five years is a long time frame and we
expect more issues to be raised during the project.

\paragraph{A model for Web data management}

We will first develop a mathematical model for Web scale data
management.  We use the term model here in a very broad sense,
encompassing issues such as the syntax of the data and typing, but
also the specification of some processing on it (data transformations,
queries, updates) as well as of behavioral properties.

The model should capture data management in typical Web applications
(see examples further).  In a nutshell, the setting consists in a
number of autonomous systems communicating by activating services on
other peers and exchanging information. Locally a peer evaluates
queries and updates, communicates with other peers, and as a result
its state evolves in time. Observe that we typically assume no central
authority and no global knowledge.  Each peer reasons and decides
independently of the other peers, even though they may exchange
information.  Critical issues such as avoiding deadlocks or not
disclosing information to peers without proper credentials have to be
resolved based on local processing only.

A major aspect of Webdam is that the mathematical model will be driven
by the specific domain of interest.  So, in the forthcoming model, we
should be able to describe complex tasks such as:
\begin{enumerate}
\item query processing and optimization, the most common tasks for
data management, including locating the data of interest, and the
management and use of (distributed) access structures such as
distributed hash tables and replication.

\item change management including the management of updates, versions
and the specification of policies for concurrency control.

\item data integration and in particular, dynamic data integration in
the style of mashups (Web applications that combine data from more
than one sources into a single integrated tool).

\item surveillance with functionalities such as complex monitoring
subscriptions.

\item efficient data exchange and integration based on complex mapping
rules between different peers in a static or dynamic (source changing)
environment.

\item diffusion/communication of information based on structured
or unstructured (gossiping) networks.

\item choreography of the activities of several services collaborating
to achieve some particular data (intensive) task.

\end{enumerate}

We already mentioned Active XML as a starting point. Part of the work
will consist in studying logics and algebras for Web data management
going beyond ActiveXML that has been previously mentioned.  In
particular, experience with that language have stressed its
limitations in terms of control.  Recent work with Segoufin and Vianu
has introduced the concept of {\em guarding} formulas to control the
activation of services.  Such features in the spirit of active rules
in databases or standard rules in production systems seem quite
promising. Indeed, they correspond quite closely to features found in
mashup systems. So the challenge will be to integrate such
sophisticated notions of control and simultaneously study restrictions
of the model so that reasoning about applications remains feasible.

\paragraph{Reasoning about Web applications}

We will also develop the necessary tools to reason about Web
applications, e.g. verify temporal properties of a system.  We will
consider both static and dynamic properties, for example:

\begin{enumerate}

\item To optimize queries (similarly for updates), we need to
automatically verify equivalence. This is typically a static property
although in a Web context, one tends to blur the separation between
query optimization (compile time) and processing (runtime) and are
often led to modify an execution plan at runtime.

\item One may wish to determine (at compile time) whether a process is
guaranteed to terminate, or whether it will always satisfy certain
desired properties (no product is ever shipped before payment has been
received).

\item One may want to determine (at runtime) why certain situation has
been reached (diagnosis) and how to recover from an undesired state
(error recovery).  This has to be done automatically since the systems
are typically self administered (self healing).

\item One may want to analyze the log of some run (possibly
distributed between several machines) and detect a posteriori, e.g.,
misuses of the system or rooms for better taking advantage of
resources.

\end{enumerate}
Observe that one cannot separate the static part of the processing
(query optimization) from the dynamic part.  For instance, a mail
order system obeying a certain workflow may involve many calls to
databases (for stock management and billing). And one may want, for
optimization reasons, to ``pack'' several calls to the databases. The
separate optimization of each call would simply miss the point.

To attack the problem, we will combine techniques from databases and
verification, using logic and automata as the cornerstone. We place as
a compulsory requirement that we want to be able to reason in absence
of any centralized authority in a context possibly involving a large
number of peers and huge volumes of data.  A measure of success for
the modeling task is the range of Web management activities we will
be able to describe (data exchange, surveillance, error recovery,
etc.). For reasoning, they are first the nature of the reasoning, both
static and on-line analysis, from optimization, to tuning, error
diagnosis and recovery, automatic administration, etc.

In contrast with theories developed in the relational context, we will
take into account the dynamic nature of the information.  In contract with
the standard works on verification, we will place information at the
center of the picture so that we do not only talk about control but
primarily about data (exchange, transformation, query, update,
etc.)

% Some of the work will consist in pursuing the transfer of
% the relational technology developed for static relations to trees and
% combine it with verification techniques developed in more dynamic
% contexts.  

The problem we will be addressing are complex. Two
main guiding principles for the methodology will make the entire
research program feasible:
\begin{enumerate}

\item Accept severe limitations: The limitations of the relational
model were indispensable to achieve its success. We will similarly
adopt severe limitations. The crux will be to still be able to capture
the essence of Web data management.

\item Keep it simple: Model simplicity is always essential in
mathematics and is a main factor of success in software development,
and this most particularly in the context of the Web (because of the
distribution, the scale, the heterogeneity and autonomy of the
participants).

\end{enumerate}

To illustrate the style of work that we will conduct, we next mention
some first results in that spirit.

Some recent works around ActiveXML \cite{AbiteboulSegoufinVianu08}
trace the border of decidability of reasoning for a specific model for
Web data management.  Temporal properties are specified in a temporal
logic based on Linear Temporal Logic \cite{tl}, that allows specifying
statements such as, {\em eventually} the system will reach the state
{\em delivered} or {\em mailorder-aborted}.  But we are specifically
interested in statements that mention explicitly data and distribution
such as: for each mailorder with a particular ID, we will eventually
reach a state, where there will be a receipt with the same ID at the
financial service of some department or a reject mail will have be
sent to the appropriate customer.  In \cite{AbiteboulSegoufinVianu08},
we use a temporal logic that integrates LTL and XML tree pattern
formulas.  We established the boundaries of decidability and the
complexity of automatic verification for this temporal logic.  From a
positive angle, this work demonstrates that one can isolate models
where desirable dynamic properties can be decided. From a negative
one, the complexity is very high. So, why are we convinced such an
approach is feasible in practice? Because the tasks we are interested
in are fairly simple: fetch data, apply simple transformation rules to
obtain new data or update existing data.  Very strong limitations (to
start, as classical in databases, abandoning Turing completeness) are
acceptable for the ability to reason about programs and in particular
optimize them.

\paragraph{Workflow and business artifacts}
An important aspect of the work will be to also develop a better
understanding of the flow of control in these applications.  The
connections with workflows will in particular have to be investigated.
A workflow is a reliably repeatable pattern of activity enabled by a
systematic organization of resources, defined roles and information
flows, into a work process that can be documented and learned
\cite{wikipedia}.  We have witnessed a mixed success of workflow
systems in industry. They are successful at an atomic level with local
applications described by workflow systems more and more routinely
turned into actual Web services. But the approach is much less
successful in environments where one has to perform the choreography
of many Web services.  In particular, the workflow approach seems to
be poorly adapted to the needs of business applications designers who
think more in terms of rules and constraints than in terms of too
constraining workflows.  A new approach based on ``business
artifacts'' has been proposed \cite{NigamCaswell}. A business artifact
can be seen as a document capturing the information that is produced
during some business activity. The state of the process is obtained
from the content of the artifact and a change of state corresponds to
an update (typically an insertion) to the artifact. This is also the
spirit of the approach we will follow, with the data and its evolution
seen as the core of Web applications and not as the by-product of a
complex workflow.  In that sense, the model we will develop may also
be seen as a theory of business artifacts.  The philosophy here is
that we want to unify the approaches based on data with those based on
control, and thus be able to reason explicitly with applications
involving both data and control.

\paragraph{Datalog}
As already mentioned, we expect datalog to play an important role in
Webdam. So we will study datalog and extensions (in particular, with
negation) in a Web setting.  By considering datalog in a distributed
context, new issues arise. For instance, simply detecting the
termination of a datalog query evaluation that is trivial in a
centralized context, becomes nontrivial and costly.  We illustrate
next three issues where datalog can be useful in our context that we
will investigate:
\begin{enumerate}

\item Query evaluation. We will study the evaluation of datalog
queries in a Web setting, with the extensional data and the
intentional definitions distributed over the network.  Difficulties in
this context (compared to a centralized one) are that the processing
may involve a large number of peers (possibly an unbounded number as
in queries to retrieve music over the Internet) and that we have to
optimize a datalog program that is distributed between many peers with
no one having a global picture of the program.

\item Data evolution.  Techniques have been developed for the
incremental maintenance of datalog queries. These techniques should be
adapted to our context in particular for handling monitoring
functionalities in a Web setting.

\item Run analysis. One can imagine some computation going on the Web
with the participants logging in journals partial distributed traces
of the execution. At any time, one may want to analyze the traces,
e.g. to detect patterns of usage of the system (surveillance), or
explain some unexpected behavior (diagnosis).  Datalog is quite
appropriate to do so.

\end{enumerate}
We expect more applications of datalog to come up during the project. 

In spite of its simplicity and limitations, it is not easy to reason
in general with datalog (e.g., containment is undecidable for
datalog), and even less with extensions such as datalog with negation
that we will have to consider.  However, datalog is a natural setting
for defining even more restricted models where reasoning is possible.

\paragraph{Inspiration and validation}

The road map up to this point may sound abstract. But we will use real
applications as inspiration for our work. We will also use the
applications as testbeds to validate the model and the results through
prototypes.  

We conclude this section by briefly presenting three of the
applications that we will consider in order to validate the
results. They will illustrate the kinds of Web activities we are
targeting.  The first example is of an industrial nature, the second
is about scientific data management, while the last one is about
social networks. The common ground is that all three involve
autonomous systems interacting by exchanging data.

\paragraph{Biological data management}

We focus on biological data but similar concerns may be found for
different kinds of scientific data as well.  There are several major
biological databases (e.g. genetics, proteomics) and thousands of
specialized databases. As a result of the evolution of domain
knowledge, these databases change regularly; the few ontologies that
govern their terminologies and notably the identifiers they use also
change often.  Although the volumes of data are not very large, the
integration of these databases poses real challenges in particular
because of dependencies between them, their highly dynamic nature and
often the presence of many inconsistencies.  In this context, a
particular database typically imports data from a number of
independent sources, restructures it, integrates it with local data,
exports some data itself. The main issues are related to the
management of changes, propagation, synchronization, etc.  The current
data models that are used, typically ad hoc extensions of the
relational model, poorly capture complex aspects such as data
provenance or temporal information and bring very limited responses to
the scientists expectations.  \\
\hspace*{0.5cm} In this application, we will most particularly focus
on change control aspects and the management of data evolution.
      
\paragraph{Supply chain}

Consider the supply chain of a computer manufacturer \cite{dell}.  The
manufacturing system processes continuous flows of orders and has to
cope with issues such as distant suppliers.  The different
participants in the P2P system are the Web servers that are used by
customers, the plants that put together the computers, the banks that
process payments, dispatchers that assign mail orders to plants,
warehouses that act as buffers between plants and suppliers, and
finally the suppliers themselves. Each participant is represented by
an autonomous peer of the system. The data exchanges between
participants are intense. We want to be able to specify such an
application, deploy and monitor it, detect, diagnose and recover from
errors, enrich the application dynamically with complex business
rules, discover and integrate new partners dynamically. 
\\
\hspace*{0.5cm} We are already using this example to test a P2P
monitoring system we have developed \cite{demo:p2pm}.
\\
\hspace*{0.5cm} In this application, we will most particularly test
issues related to the flow of control and novel ideas around business
artifacts.
   
\paragraph{Social Networking in P2P}

A social networking system such as Facebook is based on the management
of information about its users stored in a central repository. It is
rather straightforward to develop new applications using a simple
API. However, these applications are rather limited in scope. We claim
that there are two fundamental flows in this setting. The first one is
that it is technically rather inefficient to centralize all the data
and the control in a system that is bound to become a bottleneck or
end up wasting enormous resources (the Facebook farm). More
importantly, many users are reluctant to give full control over their
data to a provider (Facebook) who can sell it to other businesses and
worse, leave such control to third parties of unknown affiliations. It
is feasible to develop similar systems where personal data is kept in
full control (in some proxy database) by their owners. This becomes a
problem of Web data management in the realm we are addressing.
\\
\hspace*{0.5cm} A user interacts with a system with an interface in
the style of mashups. A proxy handles her data and the interaction
with the community. Note that with such an approach, a user can have
her own data (e.g., phone number, list of trusted friends) shared
between many systems (Myspace, GoogleMail, Flickr, etc.) rather than
replicated and inconsistent on the private servers of these systems.
\\
\hspace*{0.5cm} 
In this last application, we will more particularly test interactions
among a large number of peers. 

Each of the three application has its own specificities. The Social
Network one involves a huge number of peers. The Scientific one may be
quite demanding in terms of quantity of data. The Manufacturing one
comes with high quality of service requirements. But beyond their
specificities, all three belong to a wide range of applications that
are based on data management in a distributed context with autonomous
machines.  More generally, the technology we will develop will be
applicable to a large number of fields, including e-commerce,
e-government, information manufacturing systems, social networking
systems, classical and new telecom services.  In many such
applications, the number of participants, the complexity of their
interactions and the pressure for fast deployment and evolution make
manual solutions infeasible. The clear needs for declarative
specifications and for being able to automatically analyze and reason
about applications are essential motivations for using the forthcoming
Webdam technology.

\paragraph{Acknowledgments}
The Webdam proposal owes a lot to a number of colleagues, in
particular, Omar Benjelloun, Ioana Manolescu, Tova Milo, Neoklis
Polyzotis, Luc Segoufin and Victor Vianu. Also, this proposal has been
influenced by the work in the ANR project Docflow with Anca Muscholl
and Albert Benveniste, as well as by brainstorming around the FET-Open
project proposal Fox lead by Luc Segoufin.  We also wish to thank
Marie-H�l�ne Pautrat for her help in preparing the Webdam proposal.

\end{document}


