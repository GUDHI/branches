\documentclass[11pt,psfig]{book}

\usepackage{latexsym}
% \usepackage{url}
% \usepackage{hyperref}
\usepackage[T1]{fontenc}
\usepackage{graphicx}
% \usepackage[francais]{babel}
\usepackage[latin1]{inputenc}
\usepackage{times}

\textwidth=6in
\textheight=8.5in
\topmargin=0.0in
\oddsidemargin=0.25in
\evensidemargin=0in

\newtheorem{Definition}{Definition}[section]
\newtheorem{PseudoDefinition}[Definition]{Pseudo-Definition}
\newtheorem{Proposition}[Definition]{Proposition}
\newtheorem{Corollary}[Definition]{Corollary}
\newtheorem{Theorem}[Definition]{Theorem}
\newtheorem{Example}[Definition]{Example}
\newtheorem{Lemma}[Definition]{Lemma}
\newtheorem{Notation}[Definition]{Notation}



\begin{document}
%\input{psfig}
%environment.tex

\newcommand{\comment}[1]{

COMMENTS \\ #1 END-COMMENTS

}
% \newcommand{\comment}[1]{}

%This is used to put some text into personal comments.  
%	\comment{blabla ... blabla}
%
%For semi-final versions, 
% 1) un-% the last line, and
% 2) % all other lines


%stuff for lemmas, thms, defs, etc.  for THE BOOK

% in the following bactch of definitions two things happen.
% first, temporaryxxx are environments which start with bold Lemma or Theorem,
%	etc., and use a shared counter.
% second, the environments lemma, thm, etc. are just the same, except
%	unlike raw Latex, the text of the environment is roman instead of
%	italic

%the guys above share one counter, and start over with each section

%note that deff and nott (for `notation') are spelled strangely

% cf. p. 174 of manual

\oddsidemargin=0in
\evensidemargin=0in

\newtheorem{temporarylemma}{Lemma}[section]
\newtheorem{temporarythm}[temporarylemma]{Theorem}
\newtheorem{temporaryprop}[temporarylemma]{Proposition}
\newtheorem{temporaryconj}[temporarylemma]{Conjecture}
\newtheorem{temporarycor}[temporarylemma]{Corollary}
\newtheorem{temporaryalg}[temporarylemma]{Algorithm}
\newtheorem{temporaryproc}[temporarylemma]{Procedure}
\newtheorem{temporaryremark}[temporarylemma]{Remark}
\newtheorem{temporaryexamp}[temporarylemma]{Example}


\newenvironment{lemma}{\begin{temporarylemma} 
	\rm}{\end{temporarylemma}}
\newenvironment{thm}{\begin{temporarythm} 
	\rm}{\end{temporarythm}}
\newenvironment{prop}{\begin{temporaryprop} 
	\rm}{\end{temporaryprop}}
\newenvironment{conj}{\begin{temporaryconj} 
	\rm}{\end{temporaryconj}}
\newenvironment{cor}{\begin{temporarycor} 
	\rm}{\end{temporarycor}}
\newenvironment{alg}{\begin{temporaryalg} 
	\parbox{4in}{  } \rm}{$\Box$ \end{temporaryalg}}
\newenvironment{proc}{\begin{temporaryproc} 
	\rm}{$\Box$ \end{temporaryproc}}
\newenvironment{remark}{\begin{temporaryremark} 
	\rm}{$\Box$ \end{temporaryremark}}


% for examples, we place a horizontal bar above
% and below the example, using the new \sep command
% Unfortunately, one of the horizontal lines might
% be orphaned -- either at the bottom of a page or the top of a page.
% (Various attempts by Rick to remedy this problem failed...)

\newcommand{\sep}{\rule{12.7cm}{0.1mm}}
\newenvironment{examp}{\noindent \sep \begin{temporaryexamp} 
		\rm}{\newline \sep \end{temporaryexamp}}

% the following is for homework exercises; they
% are numbered within chapter, not section

% \newtheorem{temporaryexer}{Exercise}[chapter]
%\newtheorem{tempclubexer}[temporaryexer]{$\clubsuit$Exercise}
%\newtheorem{tempdiamondexer}[temporaryexer]{$\diamondsuit$Exercise}
%\newtheorem{tempheartexer}[temporaryexer]{$\heartsuit$Exercise}
%\newtheorem{tempspadeexer}[temporaryexer]{$\spadesuit$Exercise}
%\newenvironment{exer}{\begin{temporaryexer} \rm}{\end{temporaryexer}}
%\newenvironment{clubexer}{\begin{tempclubexer} \rm}{\end{tempclubexer}}
%\newenvironment{diamondexer}{\begin{tempdiamondexer} \rm}{\end{tempdiamondexer}}
%\newenvironment{heartexer}{\begin{tempheartexer} \rm}{\end{tempheartexer}}
%\newenvironment{spadeexer}{\begin{tempspadeexer} \rm}{\end{tempspadeexer}}

\newenvironment{deff}{\medskip \noindent \bf Definition: \rm}{\medskip}
\newenvironment{nott}{\medskip \noindent \bf Notation: \rm}{\medskip}
\newenvironment{fact}{\medskip \noindent \bf Fact: \rm}{\medskip}
\newenvironment{claim}{\medskip \noindent \bf Claim: \rm}{\medskip}

\newenvironment{proof}{\medskip \noindent \bf Proof: \rm}{$\Box$ \medskip}
\newenvironment{proofnobox}{\medskip \noindent \bf Proof: \rm}{\medskip}
\newenvironment{crux}{\medskip \noindent \bf Crux: \rm}{$\Box$ \medskip}
\newenvironment{cruxnobox}{\medskip \noindent \bf Crux: \rm}{\medskip}



% another useful place in manual is p. 57 and p. 173, 

% --------------------------------------

% some symbols we use

\newcommand{\mvd}{\mbox{$\ \rightarrow \hspace*{-.5cm} \rightarrow \ $}}
% for ``multivalued dependencies''

\newcommand{\jd}{\Join\!\!}
% for ``join dependencies'' -- gets the spacing right --
% however, use: $\models \, \jd[X,Y,Z]$ ...

% --------------------------------------

\newcommand{\reminder}[1]{ [[[ \marginpar{\mbox{$<==$}} #1 ]]] }






\newenvironment{Def} {\begin{Definition} \rm} {\end{Definition}} 
\newenvironment{The} {\begin{Theorem} \rm} {\end{Theorem}} 
\newenvironment{Exa} {\begin{Example} \rm} {\end{Example}} 
\newenvironment{Pro} {\begin{Proposition} \rm} {\end{Proposition}} 
\newenvironment{Cor} {\begin{Corollary} \rm} {\end{Corollary}} 
\newenvironment{Lem} {\begin{Lemma} \rm} {\end{Lemma}} 
\newenvironment{Not} {\begin{Notation} \rm} {\end{Notation}} 

\newcommand{\pp}{\vspace{6 mm}}

\newcommand{\begindef}{\begin{Definition} \rm}
\newcommand{\ben}{\begin{enumerate}}
\newcommand{\een}{\end{enumerate}}
\newcommand{\bit}{\begin{itemize}}
\newcommand{\eit}{\end{itemize}}
\newcommand{\bve}{\begin{verbatim}}
\newcommand{\eve}{\end{verbatim}}
\newcommand{\btab}{\begin{tabbing}}
\newcommand{\etab}{\end{tabbing}}

\newcommand{\sect}{\section}
\newcommand{\ssect}{\subsection}
\newcommand{\sssect}{\subsubsection}

\newcommand{\bigR}{$\cal R$}
\newcommand{\bigS}{$\cal S$}
\newcommand{\bigT}{$\cal T$}
% \newcommand{\SS}{\cal S}

\newcommand{\So}{\setlength{\hspace{-0.2mm}\unitlength}{1mm}\begin{picture}(3.5,5)(0,1.3)\put(1,0){\line(0,1){5}}\put(1.6,0){\line(0,1){5}}\put(1,0){\line(1,0){1.5}}\put(1,5){\line(1,0){1.5}}\end{picture}\hspace{-1mm}}
\newcommand{\Sc}{\setlength{\unitlength}{1mm}\hspace{-2.5mm}\begin{picture}(3.5,5)(0,1.3)\put(3.5,0){\line(0,1){5}}\put(2.9,0){\line(0,1){5}}\put(3.5,0){\line(-1,0){1.5}}\put(3.5,5){\line(-1,0){1.5}}\end{picture}\hspace{0.5mm}}
\newcommand{\EMo}{{\Large\bf\protect[}}
\newcommand{\EMc}{{\Large\bf\protect]}}
\newcommand{\sem}[2]{\So #1\Sc$\!\,_{\mbox{\small #2}}$}

% \newcommand{\mvd}{\mbox{$\ \rightarrow \hspace*{-.5cm} \rightarrow \ $}}

\newcommand{\esp}{\vspace*{1cm}}
\setlength{\parindent}{0.7cm}
\setlength{\parskip}{0.3cm}

\vspace*{1cm}


\begin{center}
\Huge{Webdam} \\
\vspace*{1cm}
\Large{Foundations of Web Data Management}\\
\vspace*{1cm}
\Large{Grant agreement - Annex 1 - Description of Work}\\
\vspace*{1cm}
\Large{ERC Advanced Grant} \\
\Large{7th Framework Programme} 

\end{center}

\vspace*{1cm}
Principal Investigator: Serge Abiteboul 

Host institution: INRIA Saclay--�le-de-France

Full title: Foundations of Web Data Management 

Short name: Webdam

Number: 226513 

Duration in months: 60 

Date of draft: \today

\newcommand{\victor}[1]{ }
\newcommand{\serge}[1]{ }

%% \newcommand{\victor}{{\bf FIX }}
%% \newcommand{\serge}{{\bf Serge }}

% {mh.pautrat@orange.fr}

% \renewcommand{\thesection}{}
% \renewcommand{\thesubsection}{}
% \renewcommand{\thesubsubsection}{}

\newcommand{\url}{}


\newpage

\pagestyle{myheadings}
\markboth{WebDam - Proposal 226513 - Draft - \today}{WebDam - Proposal 226513 - Draft - \today}


% {\begin{center}
\fbox{\parbox{12cm}{\noindent {\bf Abstract: } We propose to develop a
formal model for Web data management. This model will open new
horizons for the development of the Web in a well-principled way,
enhancing its functionality, performance, and
reliability. Specifically, the goal is to develop a universally
accepted formal framework for describing complex and flexible
interacting Web applications featuring notably data exchange, sharing,
integration, querying and updating. We also propose to develop formal
foundations that will enable peers to concurrently reason about global
data management activities, cooperate in solving specific tasks and
support services with desired quality of service.  Although the
proposal addresses fundamental issues, its goal is to serve as the
basis for ground-breaking future software development for Web data
management.  }}\end{center}

% UPDATE A1

% Fondements de la gestion de donn�es du Web

% L'objectif du projet Webdam est de d�velopper un cadre formel
% universellement accept� pour d�crire des applications Web mettant en
% jeu des interactions complexes et flexibles, avec notamment de
% l'�change, du partage, de l'int�gration, de l'interrogation et des
% mises � jour de donn�es. Nous proposons �galement de d�velopper les
% fondements formels qui permettront � des syst�mes autonomes de
% raisonner ensemble sur une gestion globale de donn�es, de coop�rer
% pour r�soudre des t�ches sp�cifiques et procurer des services avec les
% qualit�s de service souhait�es. Bien que ce projet aborde des
% questions fondamentales, son objectif est aussi de servir de base pour
% des d�veloppemeents futurs de logiciels pour la gestion des donn�es du
% Web, faciliter le d�veloppement du Web, am�liorer ses fonctionnalit�s,
% ses performances et sa fiabilit�.}

% \newpage

\tableofcontents

\chapter{The principal investigator}

\section{Curriculum Vit{\ae} }

Serge Abiteboul \\
Email: Serge.Abiteboul@inria.fr \\
Web�: \url{http://www-rocq.inria.fr/\~{ }abitebou}\\
Born: August 25th, 1953, Paris 

Senior researcher at the Institut National de Recherche en
Informatique et Automatique, in the highest ranking, ``classe
exceptionnelle''.

Manager of the Gemo Team on the Management of data and knowledge
distributed over the Web.

Diplomas: PhD from USC, Los Angeles, in 1982 (with Seymour Ginsburg)
and Th�se d'Etat, Universit� Paris-Sud, 1985. 

Professional experience:
\begin{center}
\begin{tabular}{|l|}
\hline
 2002-now: Researcher at INRIA Saclay, France; also Lecturer at
Universit� Paris-Sud 
\\ 2000: Co-Founder of Xyleme \cite{xyleme}
\\ 1997-2003: Part time professor, Ecole Polytechnique, Palaiseau
\\ 1995-1997: Visiting Professor at Computer Science Dept, Stanford University
\\~~~~~~ (Database and Theory groups)
\\ 1990-1995: Part time professor, Ecole Polytechnique, Palaiseau
\\ 1982-2002: Researcher at INRIA Rocquencourt, France
\\ 1979-1982: Teaching/Research assistant at USC, Los Angeles
\\ 1977-1978: Lecturer at ESEAT, Rennes, France (military service)
\\ 1976-77: Research Assistant, Computer Science Dept, Technion, Haifa, Israel
\\
\hline
\end{tabular}
\end{center}

Other appointments:
\begin{itemize}
\item Manager of research teams at Verso INRIA-Rocquencourt team (8
years) and the Gemo INRIA-Saclay one (5 years).

\item Chairman of the Scientific Board of Rocquencourt INRIA Research
  Center (2 years), and of the Scientific Board of Futurs
  (Lilles-Bordeaux-Saclay) Center (4 years). 

\item Steering committees of international conferences including ACM
Pods (chair), ACM Sigmod, LICS, ICDT, EDBT, DOOD. He has been a member
and chaired ACM Sigmod Awards committee. Member of committees of
Agence Nationale de la Recherche. 

\item Consultant for companies including Nasa, O2 Technology,
Xyleme and VCs, Isource.
\end{itemize}

Awards and citations: 
\begin{center}
\begin{tabular}{|l|}
\hline
 ACM SIGMOD Edgar F. Codd Innovations Award, 1998.
\\ Grand Prize in Computer Science of the French Academy of Sciences,
2007.
\\
 ACM SIGMOD Test of Time Award in 2004 (for paper in 1994).
\\
ACM PODS Alberto O. Mendelzon Test-of-Time Award 2008 (for paper in
 1998). 
\\ \hline
h-index of 60 (Harzing P\&P; Jan 2008).
\\ Citeseer, most cited authors in Computer Science, ranked 42 (Jan
2008).  
\\ Citeseer Top 100 all time most cited article (for the Lorel paper -
 1998). 
\\ Citeseer Top 100 most cited article for year 1990, 1996, 1997 (as
 of June 2007).  
\\ ISI Highly cited researcher.
\\ Abiteboul is co-author of 210 entries in DBLP (January 2008). \\
\hline
\end{tabular}
\end{center}

Some main recent software developments include Xyleme commercialized
by Xyleme Inc (a system for the massive storage of XML), ActiveXML in
open source (activexml.net; a system for Web data management); and
Kadop in open source (a P2P system for indexing and querying XML
collections).

Program Committee Chair for international conferences including:
\begin{center}
\begin{tabular}{|l|}
\hline
Very Large Database Conference, General program, Lyon
(2009)
\\ Very Large Database Conference, Infrastructure track,
Berlin (2003)
\\ European Conference on Digital Libraries, Paris (1999)
\\ ACM Symposium on Principles of Database Systems, San
Jose (1995)
\\ International Colloquium on Automata, Languages and
Programming, Jerusalem (1994)
\\ International Conference on Database Theory, Paris (1990)
\\
\hline
\end{tabular}
\end{center}

Abiteboul has been member of more than 70 international program
committees including VLDB, SIGMOD, LICS, PODS, ICALP, W3, SAC, ECDL,
ICDE. He has organized more than a dozen conferences, summer-schools
or workshops including major ones such as the International Conference
on Database Theory (Paris 90) and the European Conference on Digital
Libraries (Paris 99) and participated in the organization of ACM
Sigmod and PODS (Paris 04).

Funding: He has participated in the past in many French or European
grants as well as US grants, NSF and DARPA.  On going grants:
\begin{center}
\begin{tabular}{|l|l|l|l|l|}
\hline
Grant	& Funding scheme	& Participation & Amount &  Duration  \\
\hline \hline
Webcontent & ANR (France) & participant	& 262K & 06-09 \\
Docflow	& ANR (France) & participant	& 190K & 07-09 \\
\hline 
\end{tabular}
\end{center}

Webcontent: The semantic Web Platform. Abiteboul has been the
initiator of that major French project for a platform for Web content
management with major industrial partners (e.g., EADS, Thales),
start-ups (e.g., Exalead) and a dozen of top research groups.

Docflow: Analysis, monitoring, and optimization of Web documents and
services. 

\newpage
\section{Scientific Leadership Profile}

\begin{quotation}
\noindent
Serge Abiteboul, PhD 1982 from the University of Southern California,
is Senior Researcher at INRIA and manages the Gemo team on the
Management of data and knowledge distributed over the Web.  He has
held Professor positions at Stanford U. and the Ecole Polytechnique.
He is one of the co-authors of {\em Foundations of Databases}
\cite{AHV}, the book of reference in database theory.  He received the
1998 ACM SIGMOD Edgar F. Codd Innovations Award, ACM SIGMOD Test of
Time Award in 2004, and the 2007 Computer
Science Prize of the French Academy of Sciences.  He has been program
chair of a number of conferences including ACM PODS-1995, ICALP-1994,
ICDT-90, ECDL-99 and the EDBT99 Summer School, and VLDB-03. He will be
program chair of VLDB09.  He chaired the executive committee of ACM
SIGACT-SOGMOD-SIGART PODS. He co-founded in 2000 a start-up, named
Xyleme.
\end{quotation}

Since his PhD in 1982 from the University of Southern California, he
has been with the Institute National de Recherche en Informatique et
Automatique. He is now Senior Researcher at INRIA, in the highest
ranking, {\em classe exceptionnelle}. He manages the Gemo team on the
Management of data and knowledge distributed over the Web (around 30
people).  Abiteboul has brought substantial advances in computer
science both on theoretical foundations and in software developments.
For several years now, he has been focusing on Web data management
with contributions around XML, distributed query processing, and
verification.  His expertise is most adapted to this proposal focusing
on foundations of Web data management and targeted towards
optimization and automatic verification.

Abiteboul has made major contributions in a number of areas,
with significant scientific and industrial impact. One can notably cite
three areas:
\begin{description}

\item [Database theory] He is considered one of the world leaders in
database theory (only European to ever chair the ACM PODS steering
committee). His book with Hull and Vianu \cite{AHV} is {\em the}
reference in the field. A result with Vianu (the so-called
``Abiteboul-Vianu theorem'') is often quoted in finite model theory
and descriptive complexity.

\item [Object databases] He is the recipient of the prestigious ACM
SIGMOD Edgar F. Codd Innovations Award for his work in object
databases (1998). The only other European who received this award was
Rudolf Bayer in 2001.

\item [Semistructured data] His work on languages for semistructured data at
INRIA then Stanford is seen as a precursor to XQuery, the standard for
XML. His paper on the Lorel language \cite{lorel} is in Citeseer's Top
100 all-time most cited articles in computer science.

\end{description}

More recently, Abiteboul has worked on topics directly related to
Webdam. He is one of the initiators, with Omar Benjelloun and Tova Milo,
of ActiveXML, and with Ioana Manolescu, of ActiveXML algebra. He has
recently initiated a project on the verification and monitoring of
data management tasks in distributed environments with Albert
Benveniste, Anca Muscholl, Luc Segoufin and Victor Vianu.

\paragraph{Technology transfer} Abiteboul has a long record of
technology transfer to industry. In particular, he co-founded in 2000
Xyleme, a company specialized in the management of massive volumes of
XML data (acquired recently by Xitec). Recently, he was awarded the
Grand Prize in Computer Science of the French Academy of Sciences (a
yearly award, first time awarded by the academy) that recompenses
{\em scientific achievements with impact on industry}.
   
\paragraph{Impact measures} According to Citeseer, he is ranked 42nd
(as of Jan 2008) most cited author in computer science. He co-authored
one paper that is in Citeseer Top 100 all time most cited article
(1998). Three others reached Citeseer Top 100 most cited article for
year 1990, 1996, 1997 (as of June 2007).  Measures such as h-index are
rather fragile and depending on the data set.  Google Scholar seems
the most stable for computer science. Unfortunately, it includes self
citations.  Abiteboul h-index is of 60 (Harzing P\&P; Jan 2008),
computed by Quadsearch using GoogleScholar. He is a ``Highly cited
researcher'', according to ISI.
   
\paragraph{Management of science} Abiteboul has been 
  Chairman of the Scientific Board of Rocquencourt INRIA Research
  Center (2 years), and of the Scientific Board of Futurs
  (Lilles-Bordeaux-Saclay) Center (4 years).  In this position, he has
  overseen the evaluation of research teams and the creation of new
  ones. He participated in a number of prospective initiatives notably
  \cite{lowell,delos,2020}. He has served on steering committees of
  international conferences including ACM Pods (chair), ACM Sigmod,
  LICS, ICDT, EDBT, DOOD and Program Chair for the main database
  conferences (VLDB09, VLDB03, PODS95, ICDT90), digital library
  (ECDL99), computer science theory (ICALP94).  He has been a member
  and chaired ACM Sigmod Awards committee. He has organized more than
  a dozen conferences, summer-schools or workshops.

\paragraph{Students} Although his position is only for research, he has
continuously taught, in general at the graduate level. He has been
Invited Professor at Stanford (2 years) and part time Professor at
Ecole Polytechnique, Palaiseau (10 years).  He has also taught classes
in prestigious schools in Europe (e.g., Ecole Normale Sup�rieure,
Oxford U.) and the US (e.g., UC Berkeley, UC San Diego). Some of his
PhD students and post-docs now hold academic positions in France,
India, China, U.K., Finland, Greece, Canada, Israel, Germany, Norway,
Italy, Belgium, USA. Many of his ex-students are very successful in
industry as well.

\newpage
\section{10-year-Track-Record}

Eight most cited papers since 2007 according to Google
Scholar:

\begin{enumerate}
\item Object identity as a query language primitive, 
S. Abiteboul, P.C. Kanellakis, 
Journal of the ACM, 1998.
{\em Cited by 482}

\item Complexity of answering queries using materialized views,
S. Abiteboul, O. Duschka, ACM Conference on Principles
of Database Systems, 1998.  {\em Cited by 277}

\item Detecting changes in XML documents,
G. Cobena, S. Abiteboul, A. Marian, 
 International Conference on  Data Engineering, 2002.
{\em Cited by 235}

\item Extracting schema from semistructured data,
S. Nestorov, S. Abiteboul, R. Motwani,
ACM SIGMOD Conference on the Management of Data, 1998. 
{\em Cited by 225}

\item Regular Path Queries with Constraints,
S. Abiteboul, V. Vianu, 
Journal of Computer and System Sciences, 1999.
{\em Cited by 199}

\item Query flocks: a generalization of association-rule mining, 
J.D. Ullman, S. Abiteboul, C. Clifton, R. Motwani,
ACM SIGMOD Conference on the Management of Data, 1998. 
{\em Cited by 161}

\item Change-Centric Management of Versions in an XML Warehouse,
A. Marian, S. Abiteboul, G. Cobena, L. Mignet,
International Conference on Very Large Databases, 2001 
{\em Cited by 156}

\item On views and XML,
S. Abiteboul,
ACM PODS, 1999. Also in ACM Sigmod Record. 
{\em Cited by 139}
  
\end{enumerate}
ACM PODS, SIGMOD and VLDB are the most selective conferences in
databases. Typically, acceptance is considered as hard or harder than
in journals. Their estimated impacts computed by citeseer are of 1.86
(PODS), 1.74 (SIGMOD) and 1.52 (VLDB)- between top 3.68\% and 8.68\%
publication venues in computer science.

Two recent publications not yet high on Google Scholar:
\begin{enumerate}
\item Adaptive on-line page importance computation\footnote{The
algorithm computes the page rank of Web pages using very limited
resources. In particular, unlike Google PageRank, it does not require
storing the graph of the Web. It was validated on a crawl of one
billion URLs.}, S. Abiteboul, M. Preda, G. Cobena, World-Wide-Web
Conference, 2003.

\item The Active XML project,
S. Abiteboul, T. Milo, O. Benjelloun, 
The International Journal on Very Large Databases, To appear in 2008. 

\end{enumerate}

One book: Data on the Web: From Relations to Semistructured Data and
XML, S Abiteboul, P Buneman, D Suciu, Morgan-Kaufman,  1999.
(translated to Portuguese and Japanese) {\em Cited by 892}.

Invited talks to peer-reviewed, internationally established
conferences: International Symposium on Theoretical Aspects of
Computer Science (2007), European Conference on Object-Oriented
Programming (2006), International Workshop on Abstract State Machines
(2005), International Conference on Flexible Query Answering Systems
(2004), International Conference on Fundamental Approaches to Software
Engineering (part of European Joint Conferences on Theory and Practice
of Software) (2004), International Conference on Advanced Information
Systems Engineering (2003), International Conference on Database and
Expert Systems Applications (2002), International Conference on Web
Information Systems Engineering (2002), European Conference on
Research and Advanced Technology for Digital Libraries (1997)

Organization of scientific events: International Workshop on Semantic
Data and Service Integration, Vienna (2007), European Conference on
Digital Libraries, Paris (1999), Summer School on Extending Database
Technology, La Baule-les-Pins (1999), Dagstuhl School on Foundations
for Information Integration (1999).  Participation in the organization
of the ACM SIGMOD International Conference on Management of Data and
ACM International Conference on Principles of Database Systems (Paris
2004, first time outside of North America), International Conference
on Very Large Databases (Lyon 2009).

Patents:
\begin{enumerate}
\item U.S. Patent, 400652, filed 2003-03-28; System and method for
providing content warehouse; with Amir Milo and Sophie Cluet.

\item U.S. Patent, 940132, filed 2001-08-27; Ranking nodes in a graph, with
Mihai Preda.
\end{enumerate}

\chapter{The research proposal}

\section{Extended synopsis of the project proposal}

\begin{center}
\fbox{\parbox{12cm}{\noindent {\bf Abstract: } We propose to develop a
formal model for Web data management. This model will open new
horizons for the development of the Web in a well-principled way,
enhancing its functionality, performance, and
reliability. Specifically, the goal is to develop a universally
accepted formal framework for describing complex and flexible
interacting Web applications featuring notably data exchange, sharing,
integration, querying and updating. We also propose to develop formal
foundations that will enable peers to concurrently reason about global
data management activities, cooperate in solving specific tasks and
support services with desired quality of service.  Although the
proposal addresses fundamental issues, its goal is to serve as the
basis for ground-breaking future software development for Web data
management.  }}\end{center}

% UPDATE A1

% Fondements de la gestion de donn�es du Web

% L'objectif du projet Webdam est de d�velopper un cadre formel
% universellement accept� pour d�crire des applications Web mettant en
% jeu des interactions complexes et flexibles, avec notamment de
% l'�change, du partage, de l'int�gration, de l'interrogation et des
% mises � jour de donn�es. Nous proposons �galement de d�velopper les
% fondements formels qui permettront � des syst�mes autonomes de
% raisonner ensemble sur une gestion globale de donn�es, de coop�rer
% pour r�soudre des t�ches sp�cifiques et procurer des services avec les
% qualit�s de service souhait�es. Bien que ce projet aborde des
% questions fondamentales, son objectif est aussi de servir de base pour
% des d�veloppemeents futurs de logiciels pour la gestion des donn�es du
% Web, faciliter le d�veloppement du Web, am�liorer ses fonctionnalit�s,
% ses performances et sa fiabilit�.

One can see in computers and the network, a revolution comparable to
that of writing. Writing freed us from the need to memorize
information \cite{Serres07}. The new technology is freeing us (to some
extent) from the need to reason about information and will enable us
to focus our energy to imagination and creativity. The Web with data
deployed on millions of machines flourishes at the core of this new
revolution. 

The Web was first seen as a universal access to an ocean of documents.
It is progressing slowly (by Web-time measure) towards a world-wide
repository of knowledge, i.e., towards the Semantic Web
\cite{TBL01}. We are also observing the birth of Web 2.0
\cite{OReilly05}, that stresses social networks and community
building. Then came proposals for Web 3.0 up to Web Googol.0.  These
initiatives, totally lacking scientific grounding, have ``invented''
concepts already studied in research labs and sometimes even present
in software products.  However, in spite of their obvious weaknesses,
these proposals highlight the fact that the Web is meant to support
functionalities that are beyond a simplistic query/answer paradigm.
Indeed, with the Web as the basis of human sharing of information and
human interaction,
\begin{quotation}
\noindent
{\em Web data management is becoming the cornerstone of human
activities.} 
\end{quotation}

In spite of its importance, the management of data on the Web suffers
from a number of weaknesses that can be blamed to its recent birth and
% BUZZ
its too rapid growth. The goal of the proposal is to develop a
mathematical model for Web data management and formal foundations to
reason about Web applications. Indeed, we propose to develop the
analog of the formal model that has successfully underlied relational
database systems. Building on it, we propose to develop the necessary
reasoning capabilities for controlling data exchange, sharing,
integration, querying and updating, in a Web environment.
    
\paragraph{Time to stop hacking the Web }

For data management, the turn of the century has seen the maturing of
the relational database industry (with notably Oracle, IBM DB2 and MS
SQL Server) and of the corresponding scientific field, at the border
of logic and complexity theory. Since the 60's, industry and academic
research have progressed in tandem, with strong interactions that have
been instrumental in their respective successes. Today, with the
preeminence of the Web, new challenges are facing the data management
arena. New ideas are emerging and transferred into products very
rapidly. The scientific community has difficulties following the pace
and to a large extend, it is lagging behind. We claim that the quick
and dirty solutions, often used today, cannot provide long-term
solutions and do not scale to the Web of the future. We argue that,
perhaps even more than for relational systems, academia has an
essential role to play in shaping the new field. Indeed, the project
we propose tackles key technological issues that present the severe
risk of slowing down Web data management. Solutions even designed by
fantastic hackers are bound to fail due to the complexity of the
problem and its scale. We therefore propose to develop a formal model
that is needed for developing efficient, powerful and reliable
software for Web data management.
   
\paragraph{Limiting the scope: semantics}
A main issue for the Web is semantics. For instance, it is simply not
possible to use some newly discovered resource without understanding
first its semantics. The study of Web semantics leads to fascinating
challenges of an AI nature such as knowledge representation, knowledge
extraction from text, or reasoning with knowledge. Such issues will
not be central here. We are focusing instead on the data layer that
will help support semantic layers. To illustrate the distinction, for
instance, we are not concerned here with how a system analyses a Web
service to understand how to use it (an AI problem) but rather, with
how it will actually take the best advantage of the service to obtain
data (a DB problem). Of course, the boundary between the two domains
is fuzzy and the author of the proposal has worked on a number of
projects at their frontier. However, the techniques they require are
rather different, which justifies limiting the scope of the proposal
to the data management side. Clearly, semantics will still be very
present in the sense that the data management layer should be powerful
enough to satisfy the needs of semantic layers. So, for instance, the
formal model will include the means to describe static
properties of the data (e.g., with tree automata) as well as
behavioral properties (e.g., with temporal logic formulas).

\paragraph{A unifying model} 
An underlying philosophy of the proposal is that we need an unified
formal model for noncentralized data management.  So, we are concerned
here with all distributed applications that {\em possibly} manage huge
volumes of data with {\em possibly} large number of autonomous and
heterogeneous computers, typically without centralized authority.  We
believe that the foundational work that is needed should be applicable
to such a wide range of applications. The prime motivation for using a
single model across all Web applications is that it will enable all
kinds of Web applications to exchange data. Another positive aspect of
genericity, is that it will force us to stay away from being too
directly influenced by particular technological biases, and focus on
rich mathematical notions.  The Webdam proposal is thus extremely
ambitious because it targets the world-wide acceptance of a unique
model for data management on the Web.

Note that we did not use the term ``distributed data management''
because for many it comes with limitations to small numbers of data
servers, often with a centralized authority.  We were also reluctant
to use the term ``peer-to-peer'' because for many its meaning is
restricted to very large numbers of machines and high churn rates. In
Webdam, we address distributed and peer-to-peer data management
scenario, even if each such context may come equipped with its specific
optimization strategies and its specific issues. We prefer the term
``Web data management'' to stress that the Web is the testbed for this
new technology. Observe that we are neither excluding smaller scale
devices (such as RFIDs) nor smaller scale networks (such as home
networks).

Since distributed databases have been around for more than twenty
years, one may wonder at this point whether some such foundations
already exist.  It turns out that this is not the case. While he
taught distributed databases at Stanford, the principal investigator
realized how informal, imprecise and incomplete was the material on
the topic.  For teaching relational databases, one can rely on a clean
and solid formal model \cite{AHV,Ullman}. But when distribution is
considered, the rare existing books, e.g. \cite{OV99}, propose ad hoc
extensions, developed for the purpose of explaining particular aspects
and specific techniques.  A formal model is missing and to develop one
is a main goal of Webdam. Indeed, as a by-product, Webdam should
bring improvements to course materials for distributed databases, and
provide the basis for new courses on Web data management.

The first goal of the proposal is thus to develop a mathematical model
for Web data management. We next discuss the second goal, that is, to
develop formal foundations to reason about Web applications.

\paragraph{Automatic reasoning}

In many Web applications, notably industrial applications, we are
interested in achieving some desired quality of service level. For
instance, we may want to allow several users working on the same data
collection without disrupting each other (concurrency control) or to
support recovery from peer failures. Other examples of a less
classical database nature involve guaranteeing full awareness of
changes occurring in sites of interest (monitoring) or checking that
data only come from trusted sites (provenance). The current style of
development of Web applications makes it infeasible to obtain such
guarantees. Techniques developed for the relational model are not
directly applicable. For instance, concurrently control techniques are
typically assuming a central authority which rarely exists in a Web
context. Two aspects are making the problem even more challenging: (i)
the application design must be extremely flexible to adapt to the
intrinsic anarchist style of the Web and (ii) there is typically no
database administrator in Web applications and often not even any
computer specialist to design and maintain the application.  For
instance, most cooperative Web sites (e.g., based on Wordpress or
similar systems) integrating data from many sources (e.g., RSS feeds),
are developed by non computer experts.  Because of (i), we cannot
impose any a-priori design such as ACID for relational
transactions. And, because of (ii), we have to favor approaches that
do not require actually writing any complex code, with application
developers specifying declaratively requirements, and the code being
generated from them. As a consequence of the two, to still be
manageable, the system has to be able to reason about an application,
analyze the current or future run and possibly adapt based on the
results of the analysis. Automatic reasoning (taken here in a very
large sense) is unavoidable in this setting to support all activities
from querying and optimization, to updates, tuning, recovery,
monitoring, etc. We will precise further on, the kind of reasoning
that we mean.

\paragraph{High-risk, high gain}

What are the challenges and chances for achieving such an ambitious
goal with Webdam?

\paragraph{Not enough theoretical}
There is a risk to stay too close to the current Web technology.  If
we are too influenced by technological details of a non fundamental
nature, we will obtain a too complex model, biased by the current
technology and its limitations. As a consequence, it will be
impossible to lay the appropriate formal foundations.  In recent
applied works, the principal investigator could observe the limits of
ad hoc approaches and the crucial need for solid foundations for Web
data management. He is thus much aware of this pitfall. His experience
and that of a number of talented theoreticians who will be associated
to the endeavor, should allow avoiding it.

\paragraph{Too theoretical}
Of course, there is the somewhat opposite risk of developing beautiful
theoretical techniques with little impact outside of academic circles.
The record of the principal investigator is a strong indication that
this will not be the case. Indeed, he has for the last ten years
always been involved with works with transfer to industry. In
particular, his work around semistructured data has had important
impact notably on standards for XML query languages. Foundational work
here is not a goal for its own sake (which would already have been a
fair motivation) but is meant as a sound basis for future software
development.

\paragraph{All or nothing?}
% We are convinced that we can develop in the time frame desired
% foundations that should serve as the basis for needed new standards.
% Our experience in data management and more specifically on Web data
% management and the results we have already obtained in that field
% substantiate our claim.  
With top quality researchers and the ambition to succeed, Webdam will
serve as the catalyst for excellent European research on the topic.
Even if Webdam does not reach its full goal of providing a
comprehensive, universally accepted framework for Web data management,
it should provide substantial progress towards this goal.

\paragraph{An opportunity for Europe}
Does Europe have a chance to succeed in this strategic field that will
surely be very competitive?  In the early 80ths (when the PI finished
his PhD), Europe was almost absent from data management research. For
data management systems, Europe has regularly closed the gap with the
US and now hosts international quality groups in many countries, one
of them being the Gemo team.  Database theory developed in Europe
starting from a few pioneers, e.g., Paredaens from Belgium and the PI
in France.  One can now say that the center of gravity has shifted
from the US to Europe in this area.  ACM PODS has had 4 European
Program Chairs since 2000 and the past {\em 7} winners of the Best
Paper Awards had one European co-author.  Success can be achieved only
by driving the development of sophisticated mathematical models with
concrete technological goals.  With a strong presence in both database
theory and data management systems, Webdam and more generally Europe
are ideally placed to carry out such a program.

We will see in the next section how this can be achieved using a
methodology based on the following principles:
\begin{enumerate}

\item developing mathematical foundations combining techniques from
databases and verification,

\item building on the results and experience already achieved, 

\item bringing together local talents and researchers from an already
existing network of collaborators, that will be extended during the
project,

\item relying on simplicity and carefully chosen limitations in
the model of computation we use, 

\item getting inspiration from real Web applications and validating
results with prototypes.
\end{enumerate}

To conclude this section, we want to stress that:
\begin{quotation}
\noindent
{\em for obvious economical, social and political reasons, the
management of the data of the Web is of strategic importance. }
\end{quotation}We therefore strongly believe
that it is essential to fund ambitious projects in this area.

%% Because of industry pressure, it is
%% especially difficult to retain research talent in Europe and in
%% academia, in particular young researchers. For instance, 4 top
%% ex-students of the investigator now work for Google and 2 for Yahoo!
%% It is thus critical to host ambitious projects in Europe to motivate
%% the best researchers.  



\newpage

We next discuss the state of the art and the objectives, considering
in turns the two main facets, namely formal model and reasoning. We
then present the main traits of the proposal, briefly sketching the
expected scientific contributions and the milestones. More technical
details on some of these aspects may be found in a CIDR vision paper,
written with Neoklis Polyzotis \cite{AP}. Finally we discuss
resources.

\section{State-of-the-art and objectives}

\subsection{A first scientific shift in data management: trees,
functions}

With the Web, we have shifted from the management of very structured
data (i.e. relations) in centralized databases to semistructured data
(from text to trees and graphs) in autonomous distributed
systems. Even if most Web data still comes from relational databases,
they are typically viewed on the Web as trees, XML or HTML.  A main
aspect of the information is its lack of regularity coming from the
heterogeneity of the Web and from the multiplicity of sources and
authors. It is also most importantly distributed, and dynamic. Another
essential aspect of modern data management is its scale in terms both
of quantity of data (e.g., see the billions of pages indexed by
Google) and of number of peers (e.g. millions of peers for P2P music
sharing).  In Webdam, we will build on the standards of the Web,
namely XML \cite{xml} and Web services
\cite{ws}. These choices are somewhat not negotiable on the Web. They
happen to also be scientifically sound. Indeed, these standards may be
abstracted in a clean and elegant manner using trees and functions.

\paragraph{XML}
XML, the Web standard for data exchange, is based on unranked,
labeled, ordered trees. Depending on the needs, we will sometimes see
the trees as unordered to bring the model closer to standard logics
that are set-oriented as well as to lower the complexity. Labeled
trees are much more appropriate to the Web than relational structures
that are too rigid and constraining. A most natural tool for such
trees is tree automata \cite{tata} that have strong connections with
monadic second-order logic. Standard query languages for XML, XPATH
and XQuery may be seen (with some stretch of imagination) as logics
for such trees.  Theoretical aspects of XML have been actively
investigated during the last few years, with tree automata and monadic
second-order logic playing central roles. The results that were
obtained are important for us and they will serve as basis for our
investigation. They miss the target of being a formal model for Web
data management - our primary concern - for two main reasons. First,
they do not directly address distribution.  Second, they are
tree-based and not graph oriented, as more essential for the
Web. Finally, they address static issues and miss the dynamic nature
and interactivity of Web needs. For instance, tree automata are a
fantastic tool for static centralized trees, whereas we have to manage
evolving distributed graphs.

\paragraph{The playground: the Web and Web services}
One of the key features of the Web are Web services, that is the
possibility to activate a computation on a distant site and obtain
data from it or more generally interact with that data (e.g. apply
updates). In some sense, HTML, the hypergraph structure of the Web, and
Web search engines, may be seen as very limited forms of read-only
services and Web publication as primitive write services. Web services
are now giving birth to a very active area of software
development. Data exchanges between peers are captured by calls to
complex Web services taking into account the {\em logic} of these data
exchanges typically specified intentionally using logical
formulas. Like XML coming with a family of ``standards'' (XML schema,
XQuery, Xpointer, etc.), Web services come with a family of
``standards'', in particular, WSDL for specifying the signatures of
services, BPEL for specifying their sequencing, UDDI for
publishing/discovering services. In this proposal, the focus is on
services for managing information, e.g., query/update services or data
monitoring, that is on data intensive services rather than on services
performing intensive computations as in scientific grids.  More
precisely, we see a Web service here as a function that just
happens to be supported elsewhere, with some side effect there and
eventually returning a piece or a flow of data.

We have already worked for several years on a formal model based on
XML and Web services that will serve as a first step towards the Webdam
model.

\paragraph{ActiveXML}

ActiveXML \cite{axml} is simply XML with embedded Web services, i.e.,
with intentional information. So the fundamental idea is to place views,
i.e. possibly external information, at the center of the picture.  The
notion of documents with function calls is in the spirit of
object-oriented languages.  The combination of trees and functions
opens an array of new questions that are central to Web data
management, some we started addressing, for instance:
\begin{enumerate}
\item Querying intentional data: evaluating queries when part of the
data is elsewhere.
\item Incremental maintenance: maintaining a materialized view when
some functions bring in continuously data.
\item Casting: choosing which functions to call to force such a tree
to match a desired type.
\item Distribution: distributing a tree between several peers in the
style of Ldap to distribute both the information and the processing
load.
\end{enumerate}
Two essential lessons we have learned from the ActiveXML project, are
the following: (i) function calls should be asynchronous and (ii) data
exchange should be based on data streams. But most importantly we have
acquired experience in developing Web applications and on the various
facets of the problem.

The ActiveXML model is more complex than the relational model because
it encompasses in a nutshell most of the various themes of 20 years of
database research, most notably: deductive databases \cite{AHV},
object databases \cite{odmg} and active databases
\cite{WidomCeri96}. This together with the simplicity of the
definition and experiments with real Web applications are, we believe,
indications that we are on the right track. Another indication is that
independent proposals seem to reach similar shores from different
angles. For instance, starting from XQuery, accesses to several
sources are considered in \cite{DXQ}.  Although the approach may seem
very different, similar issues are encountered.

In its current state, the ActiveXML model is too basic, say like the
core relational model of the early days, before dependency and
concurrency control theory matured. As previously mentioned, the
setting is much more complex so more work is needed to achieve the
goal of a general model for Web data management with the proper
accompanying theory. Building on the results already obtained with
ActiveXML, with the experience in designing applications with that
language, we are now ready for a crucial part of the work in Webdam:
developing a simple and solid mathematical model for distributed data
management and building the accompanying theory to obtain the desired
formal foundations.  We will sketch in the next section some
directions to improve the model and questions to solve about this
model.

\subsection{A second shift in data management: deduction}

Let us reconsider the origin of the success of the relational model. A
relational database system is essentially a First-Order Logic machine
placed on everyone's desk to manage data.  More precisely, a
non-specialist can specify some needs, declaratively, in first-order
logic terms. The system then compiles such a ``logical query'' into an
``algebraic query plan'' that is optimized then evaluated. Relational
systems therefore perform automatic reasoning to handle queries and
views, e.g., to rewrite queries into equivalent ones to optimize
them. Reasoning is also present in many other aspects of relational
systems, e.g., dependencies (logical formulas over the data that the
system should enforce), transactions and concurrency control, triggers
(i.e. active rules).  In relational systems, such reasoning is in some
sense ``hard-wired''.  For instance, several transactions are allowed
to access/update simultaneously the same database. The system is in
charge of verifying that they do not interact improperly. To do that,
the system assumes some laws governing the interactions and implements
an algorithm (some reasoning) to check that these laws are not
violated.  The laws are decided in advance (e.g., ACID properties) and
the reasoning is encoded in algorithms, the correctness of which has
been proved in advance (e.g., 2phase-locking).

Similarly, we want an intelligent interface between a human being and
the network that nowadays stores the data we use.  On the Web,
logic is needed for the declarative specification of the application,
but it is also needed to describe the laws governing applications
(the equivalent to the ``hard-wired'' logic of relational systems)
since much more freedom is desired and Web data management cannot live
with inflexible preconceived laws.  The distribution also brings
fundamental differences.  The reasoning is now performed by different,
typically heterogeneous and autonomous systems.  This entails that
reasoning must become a first class citizen in the sense that one peer
may have to delegate some reasoning task to another one and that they
may have to exchange (partial) proofs.  Also, reasoning will typically
be much less hard-wired than in relational systems, since we need to
support complex interactions between peers governed by
application-dependent laws that are not known in advance. As a
consequence, the system must rely on more sophisticated reasoning for
instance to optimize queries in a distributed setting (between
autonomous peers) or perform change control in a very dynamic
distributed environment.

Deduction and reasoning have been around in relational databases,
e.g. with deductive databases. The novel shift is that deduction is
now used within a much wider spectrum of functionalities.

Reasoning about applications is also particularly essential when one
wants to compose Web services to support some complex task
(orchestration) or when several services cooperate towards a
particular goal (choreography).  One may want to verify that a
particular contract is enforced, some quality of service guaranteed.
Current Web service technology is way too limited in terms of
functionalities. Tools for verifying properties of Web services, e.g.
their composability, are too primitive. These issues are not
particular to data management, but as we will see, the fact that the
focus is on data management, does bring fundamental differences.

We briefly consider next relevant existing technologies.

Web knowledge representation languages such as RDF \cite{rdf}, SPARQL
\cite{sparql} or Owl \cite{owl} are essential components of the Web,
in particular, for data integration.  They adress issues that somewhat
complement the core issues we consider.  They often ignore
distribution and when they provide rich means for describing changes
and dynamic behavior, little is known on the automatic verification of
dynamic properties.  Since the developments of knowledge
representation models and systems are important for the semantic Web,
we will have to consider their interaction with the techniques for
data management we envision.  However, research on Web knowledge
representation is not at the core of Webdam.

In databases, reasoning about data and queries has a long history. In
particular, the beautiful theory of dependencies has emerged and query
equivalence and optimization have been studied extensively.  Most
results were obtained for the relational model even if recent
developments have been concerned with tree data and XML.  Results in
this areas are relevant, e.g. the chase, a general technique used in
many contexts for inferring data properties. Unfortunately, most of the
results deal with query processing (most of the time for monotone
queries) and ignore data changes, an essential aspect in our setting.

Information retrieval and search technology are also relevant for the
Web, because of the number of available resources and the difficulty
to choose between them.  For that aspect, we will rely on results
obtained by the ERC Advanced Grant Proposal SeCo on Search Computing.
We are considering collaborations between the two projects. 

For reasoning about Web applications, three other computer science
areas are very relevant: model checking \cite{mc}, automata
\cite{tata} and temporal logic \cite{tl}.  These areas are now very
mature with impressive achievements. In particular, the automatic
verification of temporal properties of (even distributed) programs has
made enormous progress.  Unfortunately, works in these directions
rarely take data into consideration.  Automatic program analysis is
typically limited to finite state systems whereas the presence of data
immediately leads to infinite states.  Researchers in these areas are
aware of the limitations and have considered recently incorporating
data.  For instance, there has been some recent work on integrating
some aspects of data into automata, logics, and model checking, see,
e.g., \cite{Neven&Schwentick&Vianu04,JL07,BMSSD06,B+07}.  One can in
general consider extensions of model checking techniques to infinite
state systems. In particular, symbolic analysis of infinite systems
based on rewriting techniques analyzed with automata techniques have
already proved successful in a number of areas: communication systems,
parameterized verification, program analysis, timed systems, security
protocols, XML and Web services.  This is the topic of a Dagstuhl
seminar \cite{dagstuhl} and we intend to follow this line of work.
There are also interesting recent works on the verification of Web
services with data (so infinite state), e.g., \cite{hull+05,Vianu+06}.
We will study more systematically verification techniques adapted to
our context that includes data.

Another technology is very relevant, notably because it has data at
its core, namely, datalog and deductive databases \cite{ddb,AHV}.
Observe that, because of distribution, the specification of data on
the Web, so issues such as query evaluation, are recursive by nature:
Data in Peer 1 refers to data in Peer 2, that refers to data in Peer
1. So, we need a logic that handles recursion and datalog is one that
scales to large volumes of data.

\paragraph{Datalog }
Datalog has been very popular in the 80's. In essence, datalog offers
the possibility to define recursive views, i.e., associate names to
recursive query statements. Unfortunately, datalog found no real
application that the simple transitive closure of relations from SQL
could not solve. Thus, even if sometimes silently present in
relational systems via recursive CREATE VIEW statements, datalog has
not been a success. The situation is very different now because of
distribution.  Indeed, we believe that the Web is the killer
application for datalog. To motivate this claim, we next mention three
recent works all relevant here that have in common their reliance on
datalog:
\begin{itemize}
\item Trees: In \cite{lixto}, Gottlob et al specify the extraction of
data from trees in datalog. Here what is useful is the notion of
patterns in datalog and in particular that of tree patterns.

\item Graph: In \cite{H+}, Hellerstein et al compute rooting tables
over the Internet with datalog. Here, what is useful is the recursion
in datalog that permits navigating in a graph by traversing new peers.

\item Trees and graphs: In \cite{AA05}, Abiteboul et al use datalog
to study the diagnosis problem in a network of independent
components. The ``unfolding'' of a computation (in the distributed
Petri net sense) is captured as a tree that is distributed between
different peers.
\end{itemize}

We will study datalog and extensions (in particular, with negation) in
the Web setting.  We will use the language (and extensions) both for
describing and for reasoning about Web computations.

\section{Methodology}

The methodology we adopt is based on a number of
principles that we develop in this section. We also mention problems
that we will address. Clearly, five years is a long time frame and we
expect more issues to be raised during the project.

\paragraph{A model for Web data management}

We will first develop a mathematical model for Web scale data
management.  We use the term model here in a very broad sense,
encompassing issues such as the syntax of the data and typing, but
also the specification of some processing on it (data transformations,
queries, updates) as well as of behavioral properties.

The model should capture data management in typical Web applications
(see examples further).  In a nutshell, the setting consists in a
number of autonomous systems communicating by activating services on
other peers and exchanging information. Locally a peer evaluates
queries and updates, communicates with other peers, and as a result
its state evolves in time. Observe that we typically assume no central
authority and no global knowledge.  Each peer reasons and decides
independently of the other peers, even though they may exchange
information.  Critical issues such as avoiding deadlocks or not
disclosing information to peers without proper credentials have to be
resolved based on local processing only.

A major aspect of Webdam is that the mathematical model will be driven
by the specific domain of interest.  So, in the forthcoming model, we
should be able to describe complex tasks such as:
\begin{enumerate}
\item query processing and optimization, the most common tasks for
data management, including locating the data of interest, and the
management and use of (distributed) access structures such as
distributed hash tables and replication.

\item change management including the management of updates, versions
and the specification of policies for concurrency control.

\item data integration and in particular, dynamic data integration in
the style of mashups (Web applications that combine data from more
than one sources into a single integrated tool).

\item surveillance with functionalities such as complex monitoring
subscriptions.

\item efficient data exchange and integration based on complex mapping
rules between different peers in a static or dynamic (source changing)
environment.

\item diffusion/communication of information based on structured
or unstructured (gossiping) networks.

\item choreography of the activities of several services collaborating
to achieve some particular data (intensive) task.

\end{enumerate}

We already mentioned Active XML as a starting point. Part of the work
will consist in studying logics and algebras for Web data management
going beyond ActiveXML that has been previously mentioned.  In
particular, experience with that language have stressed its
limitations in terms of control.  Recent work with Segoufin and Vianu
has introduced the concept of {\em guarding} formulas to control the
activation of services.  Such features in the spirit of active rules
in databases or standard rules in production systems seem quite
promising. Indeed, they correspond quite closely to features found in
mashup systems. So the challenge will be to integrate such
sophisticated notions of control and simultaneously study restrictions
of the model so that reasoning about applications remains feasible.

\paragraph{Reasoning about Web applications}

We will also develop the necessary tools to reason about Web
applications, e.g. verify temporal properties of a system.  We will
consider both static and dynamic properties, for example:

\begin{enumerate}

\item To optimize queries (similarly for updates), we need to
automatically verify equivalence. This is typically a static property
although in a Web context, one tends to blur the separation between
query optimization (compile time) and processing (runtime) and are
often led to modify an execution plan at runtime.

\item One may wish to determine (at compile time) whether a process is
guaranteed to terminate, or whether it will always satisfy certain
desired properties (no product is ever shipped before payment has been
received).

\item One may want to determine (at runtime) why certain situation has
been reached (diagnosis) and how to recover from an undesired state
(error recovery).  This has to be done automatically since the systems
are typically self administered (self healing).

\item One may want to analyze the log of some run (possibly
distributed between several machines) and detect a posteriori, e.g.,
misuses of the system or rooms for better taking advantage of
resources.

\end{enumerate}
Observe that one cannot separate the static part of the processing
(query optimization) from the dynamic part.  For instance, a mail
order system obeying a certain workflow may involve many calls to
databases (for stock management and billing). And one may want, for
optimization reasons, to ``pack'' several calls to the databases. The
separate optimization of each call would simply miss the point.

To attack the problem, we will combine techniques from databases and
verification, using logic and automata as the cornerstone. We place as
a compulsory requirement that we want to be able to reason in absence
of any centralized authority in a context possibly involving a large
number of peers and huge volumes of data.  A measure of success for
the modeling task is the range of Web management activities we will
be able to describe (data exchange, surveillance, error recovery,
etc.). For reasoning, they are first the nature of the reasoning, both
static and on-line analysis, from optimization, to tuning, error
diagnosis and recovery, automatic administration, etc.

In contrast with theories developed in the relational context, we will
take into account the dynamic nature of the information.  In contract with
the standard works on verification, we will place information at the
center of the picture so that we do not only talk about control but
primarily about data (exchange, transformation, query, update,
etc.)

% Some of the work will consist in pursuing the transfer of
% the relational technology developed for static relations to trees and
% combine it with verification techniques developed in more dynamic
% contexts.  

The problem we will be addressing are complex. Two
main guiding principles for the methodology will make the entire
research program feasible:
\begin{enumerate}

\item Accept severe limitations: The limitations of the relational
model were indispensable to achieve its success. We will similarly
adopt severe limitations. The crux will be to still be able to capture
the essence of Web data management.

\item Keep it simple: Model simplicity is always essential in
mathematics and is a main factor of success in software development,
and this most particularly in the context of the Web (because of the
distribution, the scale, the heterogeneity and autonomy of the
participants).

\end{enumerate}

To illustrate the style of work that we will conduct, we next mention
some first results in that spirit.

Some recent works around ActiveXML \cite{AbiteboulSegoufinVianu08}
trace the border of decidability of reasoning for a specific model for
Web data management.  Temporal properties are specified in a temporal
logic based on Linear Temporal Logic \cite{tl}, that allows specifying
statements such as, {\em eventually} the system will reach the state
{\em delivered} or {\em mailorder-aborted}.  But we are specifically
interested in statements that mention explicitly data and distribution
such as: for each mailorder with a particular ID, we will eventually
reach a state, where there will be a receipt with the same ID at the
financial service of some department or a reject mail will have be
sent to the appropriate customer.  In \cite{AbiteboulSegoufinVianu08},
we use a temporal logic that integrates LTL and XML tree pattern
formulas.  We established the boundaries of decidability and the
complexity of automatic verification for this temporal logic.  From a
positive angle, this work demonstrates that one can isolate models
where desirable dynamic properties can be decided. From a negative
one, the complexity is very high. So, why are we convinced such an
approach is feasible in practice? Because the tasks we are interested
in are fairly simple: fetch data, apply simple transformation rules to
obtain new data or update existing data.  Very strong limitations (to
start, as classical in databases, abandoning Turing completeness) are
acceptable for the ability to reason about programs and in particular
optimize them.

\paragraph{Workflow and business artifacts}
An important aspect of the work will be to also develop a better
understanding of the flow of control in these applications.  The
connections with workflows will in particular have to be investigated.
A workflow is a reliably repeatable pattern of activity enabled by a
systematic organization of resources, defined roles and information
flows, into a work process that can be documented and learned
\cite{wikipedia}.  We have witnessed a mixed success of workflow
systems in industry. They are successful at an atomic level with local
applications described by workflow systems more and more routinely
turned into actual Web services. But the approach is much less
successful in environments where one has to perform the choreography
of many Web services.  In particular, the workflow approach seems to
be poorly adapted to the needs of business applications designers who
think more in terms of rules and constraints than in terms of too
constraining workflows.  A new approach based on ``business
artifacts'' has been proposed \cite{NigamCaswell}. A business artifact
can be seen as a document capturing the information that is produced
during some business activity. The state of the process is obtained
from the content of the artifact and a change of state corresponds to
an update (typically an insertion) to the artifact. This is also the
spirit of the approach we will follow, with the data and its evolution
seen as the core of Web applications and not as the by-product of a
complex workflow.  In that sense, the model we will develop may also
be seen as a theory of business artifacts.  The philosophy here is
that we want to unify the approaches based on data with those based on
control, and thus be able to reason explicitly with applications
involving both data and control.

\paragraph{Datalog}
As already mentioned, we expect datalog to play an important role in
Webdam. So we will study datalog and extensions (in particular, with
negation) in a Web setting.  By considering datalog in a distributed
context, new issues arise. For instance, simply detecting the
termination of a datalog query evaluation that is trivial in a
centralized context, becomes nontrivial and costly.  We illustrate
next three issues where datalog can be useful in our context that we
will investigate:
\begin{enumerate}

\item Query evaluation. We will study the evaluation of datalog
queries in a Web setting, with the extensional data and the
intentional definitions distributed over the network.  Difficulties in
this context (compared to a centralized one) are that the processing
may involve a large number of peers (possibly an unbounded number as
in queries to retrieve music over the Internet) and that we have to
optimize a datalog program that is distributed between many peers with
no one having a global picture of the program.

\item Data evolution.  Techniques have been developed for the
incremental maintenance of datalog queries. These techniques should be
adapted to our context in particular for handling monitoring
functionalities in a Web setting.

\item Run analysis. One can imagine some computation going on the Web
with the participants logging in journals partial distributed traces
of the execution. At any time, one may want to analyze the traces,
e.g. to detect patterns of usage of the system (surveillance), or
explain some unexpected behavior (diagnosis).  Datalog is quite
appropriate to do so.

\end{enumerate}
We expect more applications of datalog to come up during the project. 

In spite of its simplicity and limitations, it is not easy to reason
in general with datalog (e.g., containment is undecidable for
datalog), and even less with extensions such as datalog with negation
that we will have to consider.  However, datalog is a natural setting
for defining even more restricted models where reasoning is possible.

\paragraph{Inspiration and validation}

The road map up to this point may sound abstract. But we will use real
applications as inspiration for our work. We will also use the
applications as testbeds to validate the model and the results through
prototypes.  

We conclude this section by briefly presenting three of the
applications that we will consider in order to validate the
results. They will illustrate the kinds of Web activities we are
targeting.  The first example is of an industrial nature, the second
is about scientific data management, while the last one is about
social networks. The common ground is that all three involve
autonomous systems interacting by exchanging data.

\paragraph{Biological data management}

We focus on biological data but similar concerns may be found for
different kinds of scientific data as well.  There are several major
biological databases (e.g. genetics, proteomics) and thousands of
specialized databases. As a result of the evolution of domain
knowledge, these databases change regularly; the few ontologies that
govern their terminologies and notably the identifiers they use also
change often.  Although the volumes of data are not very large, the
integration of these databases poses real challenges in particular
because of dependencies between them, their highly dynamic nature and
often the presence of many inconsistencies.  In this context, a
particular database typically imports data from a number of
independent sources, restructures it, integrates it with local data,
exports some data itself. The main issues are related to the
management of changes, propagation, synchronization, etc.  The current
data models that are used, typically ad hoc extensions of the
relational model, poorly capture complex aspects such as data
provenance or temporal information and bring very limited responses to
the scientists expectations.  \\
\hspace*{0.5cm} In this application, we will most particularly focus
on change control aspects and the management of data evolution.
      
\paragraph{Supply chain}

Consider the supply chain of a computer manufacturer \cite{dell}.  The
manufacturing system processes continuous flows of orders and has to
cope with issues such as distant suppliers.  The different
participants in the P2P system are the Web servers that are used by
customers, the plants that put together the computers, the banks that
process payments, dispatchers that assign mail orders to plants,
warehouses that act as buffers between plants and suppliers, and
finally the suppliers themselves. Each participant is represented by
an autonomous peer of the system. The data exchanges between
participants are intense. We want to be able to specify such an
application, deploy and monitor it, detect, diagnose and recover from
errors, enrich the application dynamically with complex business
rules, discover and integrate new partners dynamically. 
\\
\hspace*{0.5cm} We are already using this example to test a P2P
monitoring system we have developed \cite{demo:p2pm}.
\\
\hspace*{0.5cm} In this application, we will most particularly test
issues related to the flow of control and novel ideas around business
artifacts.
   
\paragraph{Social Networking in P2P}

A social networking system such as Facebook is based on the management
of information about its users stored in a central repository. It is
rather straightforward to develop new applications using a simple
API. However, these applications are rather limited in scope. We claim
that there are two fundamental flows in this setting. The first one is
that it is technically rather inefficient to centralize all the data
and the control in a system that is bound to become a bottleneck or
end up wasting enormous resources (the Facebook farm). More
importantly, many users are reluctant to give full control over their
data to a provider (Facebook) who can sell it to other businesses and
worse, leave such control to third parties of unknown affiliations. It
is feasible to develop similar systems where personal data is kept in
full control (in some proxy database) by their owners. This becomes a
problem of Web data management in the realm we are addressing.
\\
\hspace*{0.5cm} A user interacts with a system with an interface in
the style of mashups. A proxy handles her data and the interaction
with the community. Note that with such an approach, a user can have
her own data (e.g., phone number, list of trusted friends) shared
between many systems (Myspace, GoogleMail, Flickr, etc.) rather than
replicated and inconsistent on the private servers of these systems.
\\
\hspace*{0.5cm} 
In this last application, we will more particularly test interactions
among a large number of peers. 

Each of the three application has its own specificities. The Social
Network one involves a huge number of peers. The Scientific one may be
quite demanding in terms of quantity of data. The Manufacturing one
comes with high quality of service requirements. But beyond their
specificities, all three belong to a wide range of applications that
are based on data management in a distributed context with autonomous
machines.  More generally, the technology we will develop will be
applicable to a large number of fields, including e-commerce,
e-government, information manufacturing systems, social networking
systems, classical and new telecom services.  In many such
applications, the number of participants, the complexity of their
interactions and the pressure for fast deployment and evolution make
manual solutions infeasible. The clear needs for declarative
specifications and for being able to automatically analyze and reason
about applications are essential motivations for using the forthcoming
Webdam technology.

\paragraph{Acknowledgments}
The Webdam proposal owes a lot to a number of colleagues, in
particular, Omar Benjelloun, Ioana Manolescu, Tova Milo, Neoklis
Polyzotis, Luc Segoufin and Victor Vianu. Also, this proposal has been
influenced by the work in the ANR project Docflow with Anca Muscholl
and Albert Benveniste, as well as by brainstorming around the FET-Open
project proposal Fox lead by Luc Segoufin.  We also wish to thank
Marie-H�l�ne Pautrat for her help in preparing the Webdam proposal.


\begin{thebibliography}{5}

\bibitem {AA05} 
Diagnosis of Asynchronous Discrete Event Systems - Datalog to the
Rescue!  
S. Abiteboul, Z. Abrams, S. Haar, T. Milo, 
ACM Conference on Principles of Database Systems, 2005.

\bibitem{sigmod04} 
Lazy Query Evaluation for Active XML, 
S. Abiteboul, O. Benjelloun, B. Cautis, I. Manolescu, T.  Milo, N. Preda,
ACM SIGMOD Conference on the Management of Data, 2004.

\bibitem{axml} 
The Active XML project,
S. Abiteboul, O. Benjelloun, T. Milo, 
VLDB Journal, to appear.  \\
see  \url{http://activexml.net}.

\bibitem{demo:p2pm}
Distributed Monitoring of Peer to Peer Systems"(demo),
S. Abiteboul, P. Bourhis, B. Marinoiu,
International Conference on Data Engineering, 2008.

\bibitem{ABS00} 
Data on the Web, 
S. Abiteboul, P. Buneman, D.  Suciu, 
Morgan Kaufmann Publishers, 2000.

\bibitem{xyleme} 
The Xyleme project.  S. Abiteboul, S. Cluet, G. Ferran, M.-C. Rousset,
Computer Networks 39(3): 225-238 (2002)

\bibitem{AHV} Foundations of databases, 
S. Abiteboul, R. Hull, V. Vianu, 
Addison-Wesley, 1995.

\bibitem{lorel} 
The Lorel Query Language for Semistructured Data, 
S. Abiteboul, D. Quass, J. McHugh, J.
Widom, J. L. Wiener, 
International Journal on Digital Libraries, 1997.

\bibitem {AP}
The Data Ring: Community Content Sharing,
S. Abiteboul, N. Polyzotis,
Conference on Innovative Data Systems Research, 2007.

\bibitem{AbiteboulSegoufinVianu08}
Static Analysis of Active XML Systems,
S. Abiteboul, L. Segoufin, V. Vianu,
ACM Conference on Principles of Database Systems, 2008.

\bibitem{lixto}
Visual Web Information Extraction with Lixto,
R. Baumgartner, S. Flesca, G. Gottlob, 
VLDB Journal, 2001.

\bibitem{hull+05}
Automatic composition of transition-based semantic web services with
messaging, 
D. Berardi, D. Calvanese, G. De Giacomo, R. Hull, M. Mecella,
International Conference on Very Large Databases,
2005.

\bibitem {TBL01}
The Semantic Web,
T. Berners-Lee, James Hendler and Ora Lassila,
Scientific American Magazine, 2001.

\bibitem{BMSSD06}
Two-Variable Logic on Words with Data,
M. Boja\'nczyk et al.,
Conference on Logic in Computer Science,
2006. 

\bibitem{B+07} 
A Generic Framework for Reasoning about Dynamic
Networks of Infinite-State Processes, 
A. Bouajjani, Y. Jurski and M. Sighireanu, 
International Conference on
Tools and algorithms for the construction and analysis of systems,
2007.

\bibitem{odmg} The Object Database Standard: {ODMG-93}, editor
R. Cattell, Morgan Kaufmann, San Mateo, California, 1994.

\bibitem{mc} Model Checking, E. M. Clarke, B.-H.
Schlingloff, Chapter 24, Handbook of Automated Reasoning, Elsevier,
2001.

\bibitem{tata} Tata, Tree Automata Techniques and Applications,
H. Comon, M. Dauchet, R. Gilleron, F. Jacquemard, D. Lugiez, S. Tison
and M. Tommasi, www.grappa.univ-lille3.fr/tata/

\bibitem{delos}
DELOS-NSF Working Group on ``Digital Library Information-Technology
Infrastructures'', 2003. 

\bibitem{tl} Temporal and modal logic, E.A. Emerson, Handbook of
Theoretical Computer Science, Chapter 16, the MIT Press, 1990.

\bibitem{DXQ}
Highly distributed XQuery with DXQ,
M. F. Fern�ndez et al.,  
ACM SIGMOD Conference on the Management of Data,
2007.

\bibitem{JL07}
Alternation-free modal mu-calculus for data trees,
M. Jurdzinski and R. Lazic,
Conference on Logic in Computer Science,
2007.

\bibitem {dell}
Inventory Decisions in Dell's Supply Chain, 
R. Kapuscinski, R. Q. Zhang, P. Carbonneau, R. Moore, B. Reeves,
Interfaces, Vol. 34, No. 3, 2004.

\bibitem{H+} 
Declarative Networking: Language, Execution and Optimization,
B. T. Loo et al.,
ACM SIGMOD Conference on the Management of Data, 2006.

\bibitem{lowell} The Lowell Database Research Self-Assessment Meeting,
2003.  research.microsoft.com/\~{ }gray/lowell/

\bibitem {dagstuhl} Beyond the Finite: New Challenges in Verification
and Semistructured Data, A. Muscholl, R. Ramanujam, M. Rusinowitch,
T. Schwentick, V. Vianu, Dagstuhl Seminar, April 2008.

\bibitem{Neven&Schwentick&Vianu04}
Finite State Machines for Strings Over Infinite Alphabets,
F. Neven, T. Schwentick and V. Vianu,
ACM Transactions on Computational Logic, 2004.

\bibitem{NigamCaswell} Business
artifacts: An approach to operational specification. 
A. Nigam, N. S. Caswell, 
IBM Systems
Journal, 42(3):428-445, 2003.

\bibitem{OReilly05}
What Is Web 2.0, 
Design Patterns and Business Models for the Next Generation of Software,
T. O'Reilly,
http://www.oreillynet.com, 2005.

\bibitem{owl} OWL Web Ontology Language Overview,
http://www.w3.org/TR/owl-features/

\bibitem{OV99}
Principles of Distributed Database Systems, 
T. Ozsu and P. Valduriez,
Prentice Hall,
1999.

\bibitem{ddb}
A Survey of Research on Deductive Database Systems, 
R. Ramakrishnan, J. D. Ullman
Journal of Logic Programming, 1993.

\bibitem{rdf} Resource Description Framework 
http://www.w3.org/RDF/

\bibitem{Serres07} Les nouvelles technologies, M. Serres, INRIA
4Oth anniversary, Lille, 2007.

\bibitem{2020} Towards 2020 Science, 2005. Inspired a special issue of
{\em Nature}, 2020: Future of Computing, March 2006.\\
http://research.microsoft.com/towards2020science/

\bibitem{Ullman} Principles of Database and Knowledge Base Systems,
Volume I and II, J.D. Ullman, Computer Science Press, 1988-89.

\bibitem{Vianu+06}
Verification of communicating data-driven web services,
A. Deutsch, L. Sui, V. Vianu, D. Zhou,
ACM Symposium on Principles of Database Systems,
2006. 

\bibitem {WidomCeri96}
Introduction to Active Database Systems,
J. Widom, S. Ceri, 
in Active Database Systems: Triggers and Rules For Advanced
               Database Processing,
Morgan Kaufmann, 1-41, 
1996.

\bibitem{xml}
The Extensible Markup Language (XML),
http://www.w3.org/XML/

\bibitem{sparql}
SPARQL Query Language for RDF,
http://www.w3.org

\bibitem{wikipedia} Wikipedia, http://www.wikipedia.org/

\bibitem{w3c}
The World Wide Web Consortium (W3C),
http://www.w3.org/

\bibitem{ws}
The W3C Web Services Activity,
http://www.w3.org/2002/ws/

\end{thebibliography}

\newpage
\section{Resources}

\subsection{Budget table} \label{table1}

The budget distribution is as follows:

\begin{center}
\includegraphics[width=400pt]{Table1.jpg}
\end{center}

% \begin{center}
% \begin{tabular}{|l|l|r|}
% \hline
% Type	& Time frame	&  Amount 	\\
% \hline \hline
% && \\ \hline
% Temporary staff  && 1 183 591	\\  \hline
% Phd 1 	& year 1-3	&		105 193\\
% Phd 2 	& year 2-4	&		106 771\\
% Phd 3 	& year 3-5	&		108 372\\
% Research engineer 	& year 2-5	&		220 000\\
% Postdoc & 5 $\times$ 1 year	&		231 667\\
% Invited researchers 	& 3 $\times$ 3 months   &		49 500\\
% Invited researchers 	&  3 $\times$ 10 months	&		198 000\\
% Professor seconded at INRIA & 2 years (full time) & 164 088\\ \hline 
% && 	\\ \hline
% Permanent Staff && 629 426 \\ \hline 
% Serge Abiteboul &75\%	&		424 270\\
% Luc S�goufin &25\%	&		102 578\\
% Iona Manolescu &25\%	&		102 578\\ \hline
% && 	\\ \hline 
% \end{tabular}
% \end{center}

% \begin{center}
% \begin{tabular}{|l|r|}
% \hline \hline
% Personal costs	&	1 813 017 \\ \hline
% Workshop and travels	&	200 000\\ \hline
% Overhead (20\%)		&	402 603	\\ \hline \hline 
% Total eligible cost 	& 	2 415 620 \\ \hline  \hline 
% & \\ \hline \hline
% Requested ERC grant & 2 415 620 \\
% \hline \hline 
% \end{tabular}
% \end{center}

This budget involves some permanent researchers: Serge Abiteboul
(75\%), Ioana Manolescu and Luc Segoufin (for some smaller proportions
of their time because they are involved in other projects as well).
We expect several other researchers and professors of the
world class Data Management team Gemo and of the world class
Verification team Dahu, to also be involved in Webdam directly or in
satellite projects that will emerge from it.  They will be financed by
INRIA and its partners.

The project will involve some PhD students and
postdocs. Unfortunately, these are not easy to find.  As an
indication, in the original proposal we mentioned 3 PhD students and 5
years of postdocs. We also mentioned an engineer and some positions of
professors seconded at INRIA. We also plan to invite world class
researchers to participate in the project.  We will also rely on other
local resources for complementing the Webdam budget.

\subsection{Task table} \label{table2}

Estimated distribution between tasks and planning.

\begin{center}
\includegraphics[width=400pt]{Table2.jpg}
\end{center}

% This budget will be split between four main tasks. The following
% indicates the distribution of resources between these tasks that we
% foresee:

% \begin{center}
% \begin{tabular}{|l|c|l|}
% \hline
% Topic		& Resource share & Time frame 	\\
% \hline \hline
% Model definition &	30\%		& main effort in first 2 years 	\\
% Reasoning	 &		30\% 	& main effort in years 2-4	\\
% System issues 	 &		20\% 	& main effort in years 2-4	\\
% Prototyping 	&		20\% 	& main effort in last 2 years 	\\
% \hline 
% \end{tabular}
% \end{center}
% By system issues, we mean here distributed reasoning on aspects that
% are typically viewed as system aspects such as query optimization,
% concurrency management or self-administration/healing.

Most of these tasks are open ended and the borders between them are
imprecise. We still believe that it is important to keep some form of
balance between them. For instance, the study of a formal model has to
rely on optimization techniques, that need some prototyping and
experimentation to be validated.


\subsection{Reporting} \label{reporting}

There are two scientific reporting periods: months 1 to 30 and months
31 to 60.

Reports will be produced for these periods.

In between, the progress of the project will be reflected in a web
site: {\bf http://webdam.saclay.inria.fr}


\chapter{Research environment}

\section{PI's Host institution} \label{team}

\paragraph{INRIA Saclay--�le-de-France}
INRIA, the French institute for research in computer science and
control, operating under the dual authority of the Ministry of
Research and the Ministry of Industry, is dedicated to fundamental and
applied research in information and communication science and
technology (ICST). The Institute also plays a major role in technology
transfer by fostering training through research, diffusion of
scientific and technical information, development, as well as
providing expert advice and participating in international
programs. By playing a leading role in the scientific community in the
field and being in close contact with industry, INRIA is a major
participant in the development of ICST in France with a long record of
excellency in research.  The institute has often demonstrated in the
past its capacity to identify and strongly support ambitious research
by providing a good infrastructure, and appropriate logistics.  
\\
\hspace*{0.5cm} INRIA created a sixth center in 2002, INRIA Futurs.  In
2008, this center is splitting into 3, Lille, Bordeaux and Saclay. The
hosting institution is INRIA Saclay, set south of Paris.

\paragraph{Digiteo Lab}
There exists a strong policy of the French government to promote
Saclay as a center of excellence with the recent creation of Digiteo
Labs, a joint lab with INRIA, CNRS, University Paris-Sud, Ecole
Polytechnique, Ecole Normal Sup�rieure de Cachan among others.  This
environment is ideal to bring in the best students (e.g., from Ecole
Normale Sup�rieure, Ecole Polytechnique) to graduate programs e.g., at
University Paris-Sud. The environment is also rich in expertise in
many domains of computer science that are very relevant to Webdam,
notably human-machine interfaces (e.g. Beaudouin-Lafon, Mackay),
theorem proving and program verification (e.g., Miller, Palamidessi,
Dowek, Paulin).  Via Digiteo Labs, INRIA is a also partner of the
competitiveness cluster, {\bf System@tic Paris-Region} on telecoms,
security, defense, automotive, transportation, system design and
development tools.

{\em Saclay is a superb environment for Webdam with a large number of
internationally recognized groups in a rather small geographical
perimeter. INRIA and Digiteo Labs will bring valuable support to
Webdam. }

Two INRIA Saclay research teams will provide an excellent environment
for Webdam, namely Gemo and Dahu, in two educational centers of
excellence, respectively, Universit� de Paris-Sud and Ecole Normale
Sup�rieure de Cachan. Abiteboul will share his time between the two
places. The distance between them is 15 km with good public
transportation.

\noindent
{\bf Gemo \& Universit� Paris-Sud} Gemo is a joint team between INRIA
and the Laboratoire de Recherche en Informatique (Universit�
Paris-Sud).  \underline{Serge Abiteboul} has built that group, among
the bests in the world for distributed data and knowledge management,
and is managing it. In particular, ActiveXML has been proposed and
developed in that group.  The group includes a number of excellent
researchers with expertise on distributed data and knowledge
management who will be able to actively participate in Webdam; in
particular \underline{Ioana Manolescu} (distributed query
optimization),\underline{ Fran�ois Goasdou�}, \underline{Laurent
Simon}, \underline{Philippe Chatalic} (distributed reasoning),
\underline{Philippe \- Dague} (Web services and verification).  \\
{\bf Dahu \& ENS Cachan} Dahu is a joint team between INRIA and the
Laboratoire Sp�cification et V�rification (ENS Cachan).
\underline{Luc Segoufin} recently left Gemo to found this small but
top quality team on databases and verification. Abiteboul was
instrumental in creating this spin off of Gemo to build on the
competence of LSV on the verification of critical software and
systems. Besides Luc Segoufin the team includes excellent researchers
in verification; in particular, \underline{St�phane Demri},
\underline{Florent Jacquemard} and \underline{Cristina Sirangelo}.

{\em The two groups, with excellent researchers both in data
management and verification, provide an ideal workforce for
Webdam. }

\paragraph{The extended team}

Abiteboul has established over the years, intense collaborations with
top scientists who will be encouraged to pursue these collaborations
in the Webdam context. We only mention next on-going collaborations
with direct links to Webdam. 
\\
\hspace*{0.5cm} 
In France, some collaboration has been
going on the {\em verification of ActiveXML} with notably Albert
Benveniste and Anca Muscholl (ANR DocFlow project). 
Abiteboul is also involved in new ANR project,
namely Dataring. The PI is Patrick Valduriez. It treats of systems
aspects closely related to the topics of Webdam.  Some synergy is
expected. 
\\
\hspace*{0.5cm} In Europe, Segoufin is the prime investigator in the
Fox grant F7P proposal (short listed; Abiteboul is also involved) on
the {\em Foundations of XML} that involves the best database theory
groups in Europe (and we already mentioned the excellence of Europe on
that topic). If Fox is funded, a lot of synergy should be
expected.  
\\
\hspace*{0.5cm} 
Abiteboul has recently started
collaborating with the Oxford Group (notably \underline{Georg Gottlob}
and Michael Benedikt). (He obtained a Plummer Fellowship to visit that
group part time beginning of 2008.) He also has an on-going
cooperation with Tel Aviv U. (\underline{Tova Milo}) on {\em
workflows} and {\em mashups}.  \\
\hspace*{0.5cm} With the US, we already mentioned some existing
collaboration with UC San Diego (\underline{Victor Vianu}, \underline{Yannis
Papakonstantinou}, \underline{Alin Deutsch}) some of it on {\em reasoning on
distributed data management} and with UC Santa Cruz
(\underline{Neoklis Polyzotis}) on {\em self tuning in P2P
systems}. The UC San Diego team and Gemo have formed for several years
a {\em joint team} with funding for this cooperation both from the
French and US side (NSF).  \underline{Richard Hull} at IBM Research
has a research program on business artifacts that would fit nicely
with the Webdam agenda; he also worked in the past with Abiteboul.

Observe that these topics all participate to the Webdam vision.
Several of these top researchers (Milo, Vianu, Polyzotis) are already
regular visitors of Abiteboul (yearly and often for long periods) and
have already been important contributors to works that form the
starting point of Webdam.  We expect some of the researchers
previously mentioned and their teams to actively collaborate with
Webdam notably via visits to INRIA Saclay.  Those who are underlined
have already agreed to consider working or collaborating with Webdam.
Webdam can immensely benefit from this already existing international
network of top quality researchers and expend from it.

{\em Webdam will bring together local talent and top researchers who
have already been involved with Webdam related works in the past. We
expect more talents to join us during the project, in particular to
bring novel expertise.}

Ethical Issues: no ethical issue applies to this proposal. See Ethical
Issues Table.

\end{document}


